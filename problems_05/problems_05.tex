\documentclass[12pt, a4paper]{article}

\usepackage{import}
\subimport{../common/}{preamble}

\begin{document}

\biLangHeader{5. Адлюстраванні.}{Отображения.}

\begin{problemList}

\problemItemWithCommonPart
{Няхай $f \colon X \to Y$ "--- адвольнае адлюстраванне;
$A_i \subseteq X$ і $B_i \subseteq Y$, дзе $i = 1, 2, \ldots, n$.
Дакажыце наступныя ўласцівасці вобразаў і правобразаў:}
{Пусть $f \colon X \to Y$ "--- произвольное отображение;
$A_i \subseteq X$ и $B_i \subseteq Y$, где $i = 1, 2, \ldots, n$.
Докажите следующие свойства образов и прообразов:}
{%
\begin{belarusianEnumerate}
    \item $f(A_1 \cup A_2 \cup \ldots \cup A_n) = f(A_1) \cup f(A_2) \cup \ldots \cup f(A_n)$;
    \item $f^{-1}(B_1 \cup B_2 \cup \ldots \cup B_n) = f^{-1}(B_1) \cup f^{-1}(B_2) \cup \ldots \cup f^{-1}(B_n)$;
    \item $f(A_1 \cap A_2 \cap \ldots \cap A_n) \subseteq f(A_1) \cap f(A_2) \cap \ldots \cap f(A_n)$;
    \item $f^{-1}(B_1 \cap B_2 \cap \ldots \cap B_n) = f^{-1}(B_1) \cap f^{-1}(B_2) \cap \ldots \cap f^{-1}(B_n)$.
\end{belarusianEnumerate}
}

\smallskip

\problemItemSimple
{Няхай $U$ "--- універсальнае мноства, $S, T \subseteq U$ "--- фіксаваныя падмноствы мноства $U$.
Вызначым адлюстраванне $f \colon 2^U \to 2^U$ як $f(A) = T \cap (S \cup A)$. Знайдзіце $f^{(2)}$.
Высветліце, чаму роўнае $f^{(n)}$ пры $n > 2$.}
{Пусть $U$ "--- универсальное множество, $S, T \subseteq U$ "--- фиксированные подмножества множества $U$.
Определим отображение $f \colon 2^U \to 2^U$ как $f(A) = T \cap (S \cup A)$. Найдите $f^{(2)}$.
Выясните, чему равно $f^{(n)}$ при $n > 2$.}

\bigskip

\problemItemSimple
{%
Адлюстраванні $f, g, h \colon 2^{\mathbbmss{N}} \times 2^{\mathbbmss{N}} \to 2^{\mathbbmss{N}}$
зададзеныя наступным чынам: $f(A, B) = A \cap B$, $g(A, B) = A \cup B$, $h(A, B) = A \oplus B$.
Высветліце, якія з гэтых адлюстраванняў з'яўляюцца ін'ектыўнымі, сюр'ектыўнымі і біектыўнымі.
Вызначыце, якія з указаных ніжэй мностваў не з'яўляюцца канечнымі, і пералічыце элементы канечных мностваў:
$f^{-1}(\emptyset)$, $g^{-1}(\emptyset)$, $h^{-1}(\emptyset)$, $f^{-1}(\{1\})$,
$g^{-1}(\{2\})$, $h^{-1}(\{3\})$, $f^{-1}(\{4, 7\})$,
$g^{-1}(\{8, 12\})$, $h^{-1}(\{5, 9\})$.
}
{%
Отображения $f, g, h \colon 2^{\mathbbmss{N}} \times 2^{\mathbbmss{N}} \to 2^{\mathbbmss{N}}$
заданы следующим образом: $f(A, B) = A \cap B$, $g(A, B) = A \cup B$, $h(A, B) = A \oplus B$.
Выясните, какие из этих отображений являются инъективными, сюръективными и биективными.
Установите, какие из указанных ниже множеств не являются конечными, и перечислите элементы конечных множеств:
$f^{-1}(\emptyset)$, $g^{-1}(\emptyset)$, $h^{-1}(\emptyset)$, $f^{-1}(\{1\})$,
$g^{-1}(\{2\})$, $h^{-1}(\{3\})$, $f^{-1}(\{4, 7\})$,
$g^{-1}(\{8, 12\})$, $h^{-1}(\{5, 9\})$.
}

\bigskip

\problemItemWithCommonPart
{Высветліце, для якіх значэнняў $n \in \mathbbmss{N}$ адлюстраванне
$f_n \colon \mathbbmss{N} \cup \{0\} \to \mathbbmss{N}$ з'яўляецца ін'ектыўным, сюр'ектыўным, біектыўным:}
{Выясните, для каких значений $n \in \mathbbmss{N}$ отображение
$f_n \colon \mathbbmss{N} \cup \{0\} \to \mathbbmss{N}$ является инъективным, сюръективным, биективным:}
{%
\begin{equation*}
    f_n(k) =
    \begin{cases}
    n - k, \quad \text{калі $k < n$;} \\
    n + k, \quad \text{калі $k \ge n$.}
    \end{cases}
\end{equation*}
}

\medskip

\problemItemSimple
{Высветліце, ці з'яўляецца адлюстраванне
$f \colon \mathbbmss{N} \cup \{0\} \to \mathbbmss{Z}$, $f(n) = (-1)^{n + 1}\bigl\lfloor\frac{n + 1}{2}\bigr\rfloor$
біектыўным? Калі адлюстраванне біектыўнае, то знайдзіце адваротнае да~$f$~адлюстраванне.}
{Выясните, является ли отображение
$f \colon \mathbbmss{N} \cup \{0\} \to \mathbbmss{Z}$, $f(n) = (-1)^{n + 1}\bigl\lfloor\frac{n + 1}{2}\bigr\rfloor$
биективным? Если отображение биективно, то найдите обратное к~$f$~отображение.}

\bigskip

\problemItemSimple
{%
Адлюстраванне $f \colon X \to Y$ называецца \textit{абарачальным злева} (\textit{справа}),
калі існуе такое адлюстраванне
$f_l^{-1} \colon Y \to X$ ($f_r^{-1} \colon Y \to X$), што $f_l^{-1} \circ f = e_X$ ($f \circ f_r^{-1} = e_Y$).
Адлюстраванне $f \colon X \to Y$ называецца \textit{абарачальным},
калі існуе такое адлюстраванне $f^{-1} \colon Y \to X$, што $f^{-1} \circ f = e_X$ і $f \circ f^{-1} = e_Y$.
Дакажыце, што:
\begin{belarusianEnumerate}
    \item $f \colon X \to Y$ абарачальнае злева тады і толькі тады, калі $f$ ін'ектыўнае;
    \item $f \colon X \to Y$ абарачальнае справа тады і толькі тады, калі $f$ сюр'ектыўнае;
    \item $f \colon X \to Y$ абарачальнае тады і толькі тады, калі $f$ біектыўнае.
\end{belarusianEnumerate}
}
{%
Отображение $f \colon X \to Y$ называется \textit{обратимым слева} (\textit{справа}),
если существует такое отображение
$f_l^{-1} \colon Y \to X$ ($f_r^{-1} \colon Y \to X$), что $f_l^{-1} \circ f = e_X$ ($f \circ f_r^{-1} = e_Y$).
Отображение $f \colon X \to Y$ называется \textit{обратимым},
если существует такое отображение $f^{-1} \colon Y \to X$, что $f^{-1} \circ f = e_X$ и $f \circ f^{-1} = e_Y$.
Докажите, что:
\begin{russianEnumerate}
    \item $f \colon X \to Y$ обратимо слева тогда и только тогда, когда $f$ инъективно;
    \item $f \colon X \to Y$ обратимо справа тогда и только тогда, когда $f$ сюръективно;
    \item $f \colon X \to Y$ обратимо тогда и только тогда, когда $f$ биективно.
\end{russianEnumerate}
}

\end{problemList}

\end{document}
