\documentclass[12pt,twoside]{article}

\setlength{\textwidth}{166mm}
\setlength{\textheight}{232mm}
\setlength{\topmargin}{-10mm}
\setlength{\headsep}{5mm}
\oddsidemargin=3mm
\evensidemargin=3mm
\setlength{\baselineskip}{18pt}

\usepackage[utf8]{inputenc}
\usepackage[russian]{babel}
\usepackage{amsfonts,amssymb,amsmath}
\usepackage{epsfig}
\usepackage{mathrsfs}
\usepackage{mathabx}
\usepackage{xcolor}

\renewcommand\le{\leqslant}
\renewcommand\ge{\geqslant}

\newcommand{\ruText}[1]{
  {\footnotesize \textcolor{darkgray}{#1} \par}
}

\newcommand{\biLangHeader}[2]{
  \subsection*{%
  	#1 \\%
  	\ruText{#2}%
  }%
}

\newcommand{\quizTitle}[3]{%
\begin{center}
	\textbf{Кантрольная работа па тэме <<#1>> (варыянт #3)} \\
	\ruText{Контрольная работа по теме <<#2>> (вариант #3)}
\end{center}
}

\newcommand{\problemItemSimple}[2]{%
	\item #1 \\%
	\ruText{#2}%
}

\newcommand{\problemItemWithCommonPart}[3]{%
	\item #1 \\%
	\ruText{#2}%
	#3%
}

\newcommand{\problemItemWithCommonPartComplicated}[5]{%
	\item #1 \\%
	\ruText{#2}%
	#3 \\
	\noindent #4 \\%
	\noindent \ruText{#5}%
}

\makeatletter
\def\belarusianLetters#1{
  \expandafter\@belarusianLetters\csname c@#1\endcsname
}
\def\@belarusianLetters#1{
  (%
  \ifcase#1\or а\or б\or в\or г\or д\or е\or ж\or з\or і\or к\or л\or м\fi%
  )
}
\makeatother
\AddEnumerateCounter{\belarusianLetters}{\@belarusianLetters}{Ы}

\newenvironment{problemList}
  {\begin{enumerate}[leftmargin=*,topsep=0pt,itemsep=-1ex,partopsep=1ex,parsep=1ex]}
  {\end{enumerate}}

\newenvironment{belarusianEnumerate}
  {\begin{enumerate}[label=\belarusianLetters*, topsep=-7pt]}
  {\end{enumerate} \textbf{}\vspace{-8pt}}

\AddEnumerateCounter{\asbuk}{\@asbuk}{\cyrm}
\newenvironment{russianEnumerate}
  {\begin{enumerate}[label=(\asbuk*), topsep=-4pt, itemsep=-1ex]}
  {\end{enumerate} \textbf{}\vspace{-11pt}}


% Lines below are to avoid word breaks.
\tolerance=1
\emergencystretch=\maxdimen
\hyphenpenalty=10000
\hbadness=10000

\renewenvironment{itemize}
{\begin{list}
             {\labelitemi}%                     Old parameters:
             {\setlength{\labelwidth}{1.3em}%        1em
              \setlength{\labelsep}{0.7em}%          0.7em
              \setlength{\itemindent}{0em}%          0em
              \setlength{\listparindent}{3em}%       3em
              \setlength{\leftmargin}{2em}%          3em !
              \setlength{\rightmargin}{0em}%         0em
              \setlength{\parsep}{0ex}%              0ex
              \setlength{\topsep}{0.5ex}%            2ex !
              \setlength{\itemsep}{1ex}%             0ex
             }
}
{\end{list}}

\pagestyle{empty}


\begin{document}

\biLangHeader{
  1. Выказванні. Лагічныя аперацыі над выказваннямі. Формулы.
}{
  Высказывания. Логические операции над высказываниями. Формулы.
}

\begin{problemList}

\item Вылучыце ўмову і вынік тэарэмы, сфармулюйце яе з дапамогай звязкі <<Калі $\ldots$, то $\ldots$>>: \\
\ruText{ Выделив условие и заключение теоремы, сформулируйте её посредством связки <<Если $\ldots$, то $\ldots$>>: }

\begin{belarusianEnumerate}

\item Для таго, каб функцыя была дыферэнцавальнай ў некаторай кропцы, неабходна, каб яна была непарыўнай у гэтай кропцы; \\
\ruText{ Для того, чтобы функция была дифференцируемой в некоторой точке, необходимо, чтобы она была непрерывной в этой точке; }

\item Неабходнай уласцівасцю простакутніка з'яўляецца роўнасць яго дыяганаляў; \\
\ruText{ Необходимым свойством прямоугольника является равенство его диагоналей; }

\item Для падзельнасці мнагасклада $f(x)$ на лінейны двусклад $x - a$ дастаткова, каб лік $a$ быў коранем гэтага мнагасклада; \\
\ruText{ Для делимости многочлена $f(x)$ на линейный двучлен $x - a$ достаточно, чтобы число $a$ было корнем этого многочлена; }

\item На 5 дзеляцца тыя цэлыя лікі, якія сканчваюцца лічбамі 0 альбо 5; \\
\ruText{ На 5 делятся те целые числа, которые оканчиваются цифрой 0 или цифрой 5; }

\item Дзве прамыя на плоскасці паралельныя тады, калі яны перпендыкулярныя адной і той жа прамой; \\
\ruText{ Две прямые на плоскости тогда параллельны, когда они перпендикулярны одной и той же прямой; }

\item Камплексныя лікі роўныя, толькі калі роўныя адпаведна іх сапраўдная і ўяўная часткі; \\
\ruText{ Комплексные числа равны, только если равны соответственно их действительные и мнимые части; }

\item Любое квадратнае раўнанне з рэчаіснымі каэфіцыентамі мае не больш за два рэчаісных кораня; \\
\ruText{ Всякое квадратное уравнение с действительными коэффициентами имеет не более двух действительных корней; }

\item З таго, што чатырохкутнік --- ромб, вынікае, што кожная з яго дыяганаляў з'яўлецца воссю сіметрыі; \\
\ruText{ Из того, что четырехугольник --- ромб, следует, что каждая из его диагоналей служит его осью симметрии; }

\item Цотнасць сумы з'яўляецца неабходнай умовай цотнасці кожнага складніка; \\
\ruText{ Четность суммы есть необходимое условие четности каждого слагаемого; }

\item Роўнасць трыкутнікаў з'яўляецца дастатковай умовай іх роўнавялікасці; \\
\ruText{ Равенство треугольников есть достаточное условие их равновеликости; }

\item Для падзельнасці здабытку на нейкі лік дастаткова, каб прынамсі адзін з множнікаў дзяліўся на гэты лік. \\
\ruText{ Для делимости произведения на некоторое число достаточно, чтобы по меньшей мере один из сомножителей делился на это число. }

\end{belarusianEnumerate}

\item
Няхай $A$, $B$ і $C$ абазначаюць, адпаведна, наступныя сказы:
\textit{\guillemotleft Ён чытае коміксы.\guillemotright},
\textit{\guillemotleft Ён любіць навуковую фантастыку.\guillemotright},
\textit{\guillemotleft Ён --- студэнт-інфарматык.\guillemotright}.
Запішыце у сімвалічнай форме выказванне: \textit{\guillemotleft Калі ён чытае коміксы,
то ён любіць навуковую фантастыку і калі ён не чытае коміксы, то ён ---
студэнт-інфарматык.\guillemotright} Запішыце адмаўленне гэтага выразу і прадстаўце яго ў выглядзе формулы,
якая змяшчае толькі аперацыі дыз'юнкцыі, кан'юнкцыі і адмаўлення, прычым адмаўленні могуць распаўсюджвацца
толькі на прапазіцыйныя зменныя.

\noindent
\ruText{ Пусть $A$, $B$ и $C$ обозначают соответственно следующие предложения:
\textit{\guillemotleft Он читает комиксы.\guillemotright},
\textit{\guillemotleft Он любит научную фантастику.\guillemotright},
\textit{\guillemotleft Он студент-информатик.\guillemotright}.
Запишите в символической форме высказывание: \textit{\guillemotleft Если он читает
комиксы, то он любит научную фантастику и если он не читает комиксы, то он ---
студент-информатик.\guillemotright} Запишите отрицание этого выражения и представьте
его в виде формулы, которая содержит только операции дизъюнкции, конъюнкции и
отрицания, причем отрицания распространяются только на пропозиционные переменные. }

\smallskip

\item
Пабудуйце табліцы праўдзівасці наступных формул: \\
\ruText{ Постройте таблицы истинности следующих формул: }

\begin{belarusianEnumerate}

\item $(A \to B) \vee (A \to (A \cdot B))$.

\item $((A \thicksim B) \to \overline{C}) \cdot (A \vee C)$.

\item $(((\overline{A \vee B}) \cdot \overline{C}) \to \overline{B}) \thicksim A$.

\item $((\overline{A} \cdot \overline{B}) \to (\overline{\overline{B} \to \overline{A}}))
\cdot ((A \vee B) \thicksim C)$.

\end{belarusianEnumerate}

\smallskip

\item
Рашыце наступныя лагічныя раўнанні: \\
\ruText{ Решите следующие логические уравнения: }

\begin{belarusianEnumerate}

\item $(A \to C) \cdot (\overline{(B \to C) \to ((A \vee B) \to C)}) = \text{П \textit{(Праўда)}}$.

\item $((\overline{A \cdot B}) \thicksim C) \to (C \vee \overline{A}) = \text{Н \textit{(Няпраўда)}}$.

\item $(\overline{A \to \overline{B}}) \to ((\overline{A \vee (B \thicksim A)}) \to C) = \text{П}$.

\item $((\overline{A \thicksim B}) \cdot (\overline{A \thicksim C})) \to
(\overline{A \thicksim (B \cdot D)}) = \text{Н}$.

\end{belarusianEnumerate}

\smallskip

\item
Дакажыце наступныя раўназначнасці без выкарыстання табліц праўдзівасці: \\
\ruText{ Докажите следующие равносильности без использования таблиц истинности: }

\begin{belarusianEnumerate}

\item $(A \cdot (B \vee \overline{C})) \vee \overline{A} \vee (B \cdot C) \vee
(A \cdot \overline{C}) \equiv \overline{A} \vee B \vee \overline{C}$.

\item $(((((A \to B) \to \overline{A}) \to \overline{B}) \to \overline{C}) \to C)
\equiv C$.

\item $((\overline{(A \cdot B) \to C}) \to (\overline{A \cdot C})) \to ((A \cdot B) \to
(\overline{A \cdot (B \to C)})) \equiv \overline{A} \vee \overline{B} \vee \overline{C}$.

\item $A \to ((A \cdot B) \to (((A \to B) \to B) \cdot C)) \equiv B \to (A \to C)$.

\end{belarusianEnumerate}

\smallskip

\item
Знайдзіце такую формулу $\boldsymbol{\Phi}$, што: \\
\ruText{ Найдите такую формулу $\boldsymbol{\Phi}$, что: }

\begin{belarusianEnumerate}

\item $\vDash ((\boldsymbol{\Phi} \cdot A) \to \overline{B}) \to
((B \to \overline{A}) \to \boldsymbol{\Phi})$;

\item $\vDash (\boldsymbol{\Phi} \to (A \to (B \to C))) \thicksim (\boldsymbol{\Phi} \to (A \to B))$;

\item $\vDash (\boldsymbol{\Phi} \cdot (A \vee (B \to C))) \thicksim \boldsymbol{\Phi}$;

\item $\vDash ((A \vee \boldsymbol{\Phi}) \thicksim A) \thicksim (\overline{A} \to
(B \vee \overline{C}))$.

\end{belarusianEnumerate}

\smallskip

\item
Дакажыце наступныя сцвярджэнні: \\
\ruText{ Докажите следующие утверждения: }

\begin{belarusianEnumerate}

\item калі\, $\vDash \boldsymbol{\mathrm{A}} \vee \boldsymbol{\mathrm{B}}$,\,
$\vDash \overline{\boldsymbol{\mathrm{A}}} \vee \boldsymbol{\mathrm{C}}$,\,
то\, $\vDash \boldsymbol{\mathrm{B}} \vee \boldsymbol{\mathrm{C}}$;

\item калі\, $\vDash \boldsymbol{\mathrm{A}} \to \boldsymbol{\mathrm{B}}$,\,
$\vDash \boldsymbol{\mathrm{A}} \cdot \boldsymbol{\mathrm{C}}$,\, то\,
$\vDash \boldsymbol{\mathrm{B}} \cdot \boldsymbol{\mathrm{C}}$;

\item калі\, $\vDash \boldsymbol{\mathrm{A}} \vee \boldsymbol{\mathrm{B}}$,\,
$\vDash \boldsymbol{\mathrm{A}} \to \boldsymbol{\mathrm{C}}$,\,
$\vDash \boldsymbol{\mathrm{B}} \to \boldsymbol{\mathrm{D}}$,\, то\,
$\vDash \boldsymbol{\mathrm{C}} \vee \boldsymbol{\mathrm{D}}$;

\item калі\, $\vDash \boldsymbol{\mathrm{A}} \cdot \boldsymbol{\mathrm{B}}$,\,
$\vDash \boldsymbol{\mathrm{B}} \thicksim \boldsymbol{\mathrm{C}}$,\, то\,
$\vDash \boldsymbol{\mathrm{D}} \to (\boldsymbol{\mathrm{A}} \cdot \boldsymbol{\mathrm{C}})$.

\end{belarusianEnumerate}

\smallskip

\item
Высветліце, ці справядлівыя наступныя лагічныя вынікі: \\
\ruText{ Выясните, верны ли следующие логические следования: }

\begin{belarusianEnumerate}
\item $\boldsymbol{\mathrm{A}} \to \boldsymbol{\mathrm{B}}$,\,
$\boldsymbol{\mathrm{D}} \to \overline{\boldsymbol{\mathrm{C}}}$,\,
$\boldsymbol{\mathrm{C}} \vee \overline{\boldsymbol{\mathrm{B}}}$\,
$\vDash \boldsymbol{\mathrm{A}} \to \overline{\boldsymbol{\mathrm{D}}}$;

\item $\boldsymbol{\mathrm{A}} \to \boldsymbol{\mathrm{B}}$,\, $((\boldsymbol{\mathrm{A}}
\vee \boldsymbol{\mathrm{D}}) \cdot \boldsymbol{\mathrm{C}}) \to \boldsymbol{\mathrm{E}}$,\,
$\boldsymbol{\mathrm{D}} \to \boldsymbol{\mathrm{C}}$\, $\vDash ((\boldsymbol{\mathrm{A}} \vee \boldsymbol{\mathrm{D}}) \cdot \boldsymbol{\mathrm{B}}) \to \overline{\boldsymbol{\mathrm{E}}}$;

\item $(\boldsymbol{\mathrm{A}} \vee \boldsymbol{\mathrm{B}}) \to (\boldsymbol{\mathrm{C}} \cdot
\boldsymbol{\mathrm{D}})$,\, $(\boldsymbol{\mathrm{D}} \vee \boldsymbol{\mathrm{E}}) \to
\boldsymbol{\mathrm{F}}$\, $\vDash \boldsymbol{\mathrm{A}} \to \boldsymbol{\mathrm{F}}$;

\item $(\boldsymbol{\mathrm{A}} \cdot \boldsymbol{\mathrm{B}}) \to \boldsymbol{\mathrm{C}}$,\,
$(\boldsymbol{\mathrm{C}} \cdot \boldsymbol{\mathrm{D}}) \to \boldsymbol{\mathrm{E}}$,\,
$\overline{\boldsymbol{\mathrm{F}}} \to (\boldsymbol{\mathrm{D}} \cdot \boldsymbol{\mathrm{E}})$\,
$\vDash (\boldsymbol{\mathrm{A}} \cdot \boldsymbol{\mathrm{B}}) \to \boldsymbol{\mathrm{F}}$.

\end{belarusianEnumerate}

\smallskip

\item
Высветліце, ці справядлівыя наступныя сцвярджэнні: \\
\ruText{ Выясните, верны ли следующие утверждения: }

\begin{belarusianEnumerate}

\item калі\, $\boldsymbol{\Gamma} \vDash \boldsymbol{\mathrm{A}}$\, і\, $\boldsymbol{\Gamma}
\vDash \boldsymbol{\mathrm{B}}$,\, то\, $\boldsymbol{\Gamma} \vDash \boldsymbol{\mathrm{A}} \cdot
\boldsymbol{\mathrm{B}}$;

\item $\boldsymbol{\Gamma} \vDash \boldsymbol{\mathrm{A}} \to \boldsymbol{\mathrm{B}}$\, тады
і толькі тады, калі\, $\boldsymbol{\Gamma},\, \overline{\boldsymbol{\mathrm{A}}} \vDash \boldsymbol{\mathrm{B}}$;

\item калі\, $\boldsymbol{\Gamma},\, \boldsymbol{\mathrm{A}} \vDash \boldsymbol{\mathrm{B}}$\, і\,
$\boldsymbol{\Gamma},\, \boldsymbol{\mathrm{A}} \vDash \overline{\boldsymbol{\mathrm{B}}}$,\, то\,
$\boldsymbol{\Gamma} \vDash \overline{\boldsymbol{\mathrm{A}}}$;

\item $\boldsymbol{\Gamma} \vDash \boldsymbol{\mathrm{A}} \to (\boldsymbol{\mathrm{B}}
\to \boldsymbol{\mathrm{C}})$\, тады і толькі тады, калі\, $\boldsymbol{\Gamma} \vDash (\boldsymbol{\mathrm{A}} \cdot \boldsymbol{\mathrm{B}}) \to \boldsymbol{\mathrm{C}}$.

\end{belarusianEnumerate}

(Тут $\boldsymbol{\Gamma}$ --- канечнае мноства формул, магчыма пустое.) \\
\ruText{ (Здесь $\boldsymbol{\Gamma}$ --- конечное множество формул, возможно пустое.) }

\end{problemList}

\end{document}
