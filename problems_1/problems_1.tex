\documentclass[12pt,twoside]{article}

\setlength{\textwidth}{166mm}
\setlength{\textheight}{232mm}
\setlength{\topmargin}{-10mm}
\setlength{\headsep}{5mm}
\oddsidemargin=3mm
\evensidemargin=3mm
\setlength{\baselineskip}{18pt}

\usepackage[utf8]{inputenc}
\usepackage[russian]{babel}
\usepackage{amsfonts,amssymb,amsmath}
\usepackage{epsfig}
\usepackage{mathrsfs}
\usepackage{mathabx}
\usepackage{xcolor}

\renewcommand\le{\leqslant}
\renewcommand\ge{\geqslant}

\newcommand{\ruText}[1]{
  {\footnotesize \textcolor{darkgray}{#1} \par}
}

\newcommand{\biLangHeader}[2]{
  \subsection*{%
  	#1 \\%
  	\ruText{#2}%
  }%
}

\newcommand{\quizTitle}[3]{%
\begin{center}
	\textbf{Кантрольная работа па тэме <<#1>> (варыянт #3)} \\
	\ruText{Контрольная работа по теме <<#2>> (вариант #3)}
\end{center}
}

\newcommand{\problemItemSimple}[2]{%
	\item #1 \\%
	\ruText{#2}%
}

\newcommand{\problemItemWithCommonPart}[3]{%
	\item #1 \\%
	\ruText{#2}%
	#3%
}

\newcommand{\problemItemWithCommonPartComplicated}[5]{%
	\item #1 \\%
	\ruText{#2}%
	#3 \\
	\noindent #4 \\%
	\noindent \ruText{#5}%
}

\makeatletter
\def\belarusianLetters#1{
  \expandafter\@belarusianLetters\csname c@#1\endcsname
}
\def\@belarusianLetters#1{
  (%
  \ifcase#1\or а\or б\or в\or г\or д\or е\or ж\or з\or і\or к\or л\or м\fi%
  )
}
\makeatother
\AddEnumerateCounter{\belarusianLetters}{\@belarusianLetters}{Ы}

\newenvironment{problemList}
  {\begin{enumerate}[leftmargin=*,topsep=0pt,itemsep=-1ex,partopsep=1ex,parsep=1ex]}
  {\end{enumerate}}

\newenvironment{belarusianEnumerate}
  {\begin{enumerate}[label=\belarusianLetters*, topsep=-7pt]}
  {\end{enumerate} \textbf{}\vspace{-8pt}}

\AddEnumerateCounter{\asbuk}{\@asbuk}{\cyrm}
\newenvironment{russianEnumerate}
  {\begin{enumerate}[label=(\asbuk*), topsep=-4pt, itemsep=-1ex]}
  {\end{enumerate} \textbf{}\vspace{-11pt}}


% Lines below are to avoid word breaks.
\tolerance=1
\emergencystretch=\maxdimen
\hyphenpenalty=10000
\hbadness=10000

\renewenvironment{itemize}
{\begin{list}
             {\labelitemi}%                     Old parameters:
             {\setlength{\labelwidth}{1.3em}%        1em
              \setlength{\labelsep}{0.7em}%          0.7em
              \setlength{\itemindent}{0em}%          0em
              \setlength{\listparindent}{3em}%       3em
              \setlength{\leftmargin}{2em}%          3em !
              \setlength{\rightmargin}{0em}%         0em
              \setlength{\parsep}{0ex}%              0ex
              \setlength{\topsep}{0.5ex}%            2ex !
              \setlength{\itemsep}{1ex}%             0ex
             }
}
{\end{list}}

\pagestyle{empty}


\begin{document}

\biLangHeader
{1. Выказванні. Лагічныя аперацыі над выказваннямі. Табліцы праўдзівасці.}
{Высказывания. Логические операции над высказываниями. Таблицы истинности.}

\begin{problemList}
	
\problemItemSimple
{Вылучыце ўмову і вынік тэарэмы, сфармулюйце яе з дапамогай звязкі <<Калі $\ldots$, то $\ldots$>>:}
{Выделив условие и заключение теоремы, сформулируйте её посредством связки <<Если $\ldots$, то $\ldots$>>:}

\begin{belarusianEnumerate}

\problemItemSimple
{Для таго, каб функцыя была дыферэнцавальнай ў некаторай кропцы, неабходна, каб яна была непарыўнай у гэтай кропцы;}
{Для того, чтобы функция была дифференцируемой в некоторой точке, необходимо, чтобы она была непрерывной в этой точке;}

\problemItemSimple
{Неабходнай уласцівасцю простакутніка з'яўляецца роўнасць яго дыяганаляў;}
{Необходимым свойством прямоугольника является равенство его диагоналей;}

\problemItemSimple
{Для падзельнасці мнагасклада $f(x)$ на лінейны двусклад $x - a$ дастаткова, каб лік $a$ быў коранем гэтага мнагасклада;}
{Для делимости многочлена $f(x)$ на линейный двучлен $x - a$ достаточно, чтобы число $a$ было корнем этого многочлена;}

\problemItemSimple
{На 5 дзеляцца тыя цэлыя лікі, якія сканчваюцца лічбамі 0 альбо 5;}
{На 5 делятся те целые числа, которые оканчиваются цифрой 0 или цифрой 5;}

\problemItemSimple
{Дзве прамыя на плоскасці паралельныя тады, калі яны перпендыкулярныя адной і той жа прамой;}
{Две прямые на плоскости тогда параллельны, когда они перпендикулярны одной и той же прямой;}

\problemItemSimple
{Камплексныя лікі роўныя, толькі калі роўныя адпаведна іх сапраўдная і ўяўная часткі;}
{Комплексные числа равны, только если равны соответственно их действительные и мнимые части;}

\problemItemSimple
{Любое квадратнае раўнанне з рэчаіснымі каэфіцыентамі мае не больш за два рэчаісных кораня;}
{Всякое квадратное уравнение с действительными коэффициентами имеет не более двух действительных корней;}

\problemItemSimple
{З таго, што чатырохкутнік --- ромб, вынікае, што кожная з яго дыяганаляў з'яўлецца воссю сіметрыі;}
{Из того, что четырехугольник --- ромб, следует, что каждая из его диагоналей служит его осью симметрии;}

\problemItemSimple
{Цотнасць сумы з'яўляецца неабходнай умовай цотнасці кожнага складніка;}
{Четность суммы есть необходимое условие четности каждого слагаемого;}

\problemItemSimple
{Роўнасць трыкутнікаў з'яўляецца дастатковай умовай іх роўнавялікасці;}
{Равенство треугольников есть достаточное условие их равновеликости;}

\problemItemSimple
{Для падзельнасці здабытку на нейкі лік дастаткова, каб прынамсі адзін з множнікаў дзяліўся на гэты лік.}
{Для делимости произведения на некоторое число достаточно, чтобы по меньшей мере один из сомножителей делился на это число.}

\end{belarusianEnumerate}

\bigskip

\problemItemSimple
{Няхай $A$, $B$ і $C$ абазначаюць, адпаведна, наступныя сказы:
\textit{<<Ён чытае коміксы.>>},
\textit{<<Ён любіць навуковую фантастыку.>>},
\textit{<<Ён --- студэнт-інфарматык.>>}.
Запішыце у сімвалічнай форме выказванне: \textit{<<Калі ён чытае коміксы,
то ён любіць навуковую фантастыку і калі ён не чытае коміксы, то ён ---
студэнт-інфарматык.>>} Запішыце адмаўленне гэтага выразу і прадстаўце яго ў выглядзе формулы,
якая змяшчае толькі аперацыі дыз'юнкцыі, кан'юнкцыі і адмаўлення, прычым адмаўленні могуць распаўсюджвацца
толькі на прапазіцыйныя зменныя.}
{Пусть $A$, $B$ и $C$ обозначают соответственно следующие предложения:
\textit{<<Он читает комиксы.>>},
\textit{<<Он любит научную фантастику.>>},
\textit{<<Он студент-информатик.>>}.
Запишите в символической форме высказывание: \textit{<<Если он читает
комиксы, то он любит научную фантастику и если он не читает комиксы, то он ---
студент-информатик.>>} Запишите отрицание этого выражения и представьте
его в виде формулы, которая содержит только операции дизъюнкции, конъюнкции и
отрицания, причем отрицания распространяются только на пропозиционные переменные.}

\bigskip

\problemItemWithCommonPart
{Пабудуйце табліцы праўдзівасці наступных формул:}
{Постройте таблицы истинности следующих формул:}
{%
\begin{belarusianEnumerate}

\item $(A \to B) \vee (A \to (A \cdot B))$.
\item $((A \thicksim B) \to \overline{C}) \cdot (A \vee C)$.
\item $(((\overline{A \vee B}) \cdot \overline{C}) \to \overline{B}) \thicksim A$.
\item $((\overline{A} \cdot \overline{B}) \to (\overline{\overline{B} \to \overline{A}}))
\cdot ((A \vee B) \thicksim C)$.

\end{belarusianEnumerate}
}

\bigskip

\problemItemWithCommonPart
{Рашыце наступныя лагічныя раўнанні:}
{Решите следующие логические уравнения:}
{%
\begin{belarusianEnumerate}

\item $(A \to C) \cdot (\overline{(B \to C) \to ((A \vee B) \to C)}) = \text{П \textit{(Праўда)}}$.

\item $((\overline{A \cdot B}) \thicksim C) \to (C \vee \overline{A}) = \text{Н \textit{(Няпраўда)}}$.

\item $(\overline{A \to \overline{B}}) \to ((\overline{A \vee (B \thicksim A)}) \to C) = \text{П}$.

\item $((\overline{A \thicksim B}) \cdot (\overline{A \thicksim C})) \to
(\overline{A \thicksim (B \cdot D)}) = \text{Н}$.

\end{belarusianEnumerate}
}

\end{problemList}

\end{document}
