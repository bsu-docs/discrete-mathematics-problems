\usepackage[utf8]{inputenc}
\usepackage[russian]{babel}
\usepackage{amsfonts,amssymb,amsmath}
\usepackage{float}
\usepackage{epsfig}
\usepackage{mathrsfs}
\usepackage{mathabx}
\usepackage{xcolor}
\usepackage{enumitem}
\usepackage{hyperref}
\usepackage{bbm}
\usepackage{geometry}
\usepackage{ifthen}

\geometry{left=2cm, right=1cm, top=1.5cm, bottom=1.5cm}
\newcommand{\ruText}[1]{
  {\scriptsize \textcolor{darkgray}{#1} \par}
}

\newcommand{\biLangHeader}[2]{
  \subsection*{
    {\normalsize #1} \\
    \indent \ruText{#2}
  }
}


% Lines below are to avoid word breaks.
\tolerance=1
\emergencystretch=\maxdimen
\hyphenpenalty=10000
\hbadness=10000

\pagestyle{empty}

\usepackage{titlesec}
\titleformat{\subsection}[display]{\bfseries\filright}{}{}{}



\begin{document}

\biLangHeader{
  1. Выказванні. Лагічныя аперацыі над выказваннямі. Формулы.
}{
  Высказывания. Логические операции над высказываниями. Формулы.
}

\begin{itemize}

\item[1.]

Вылучыце ўмову і вынік тэарэмы, сфармулюйце яе з дапамогай звязкі <<Калі $\ldots$, то $\ldots$>>: \\
\ruText{ Выделив условие и заключение теоремы, сформулируйте её посредством связки <<Если $\ldots$, то $\ldots$>>: }
\begin{itemize}

\item[(а)] Для таго, каб функцыя была дыферэнцавальнай ў некаторай кропцы, неабходна, каб яна была непарыўнай у гэтай кропцы; \\
\ruText{ Для того, чтобы функция была дифференцируемой в некоторой точке, необходимо, чтобы она была непрерывной в этой точке; }

\item[(б)] Неабходнай уласцівасцю простакутніка з'яўляецца роўнасць яго дыяганаляў; \\
\ruText{ Необходимым свойством прямоугольника является равенство его диагоналей; }

\item[(в)] Для падзельнасці мнагасклада $f(x)$ на лінейны двусклад $x - a$ дастаткова, каб лік $a$ быў коранем гэтага мнагасклада; \\
\ruText{ Для делимости многочлена $f(x)$ на линейный двучлен $x - a$ достаточно, чтобы число $a$ было корнем этого многочлена; }

\item[(г)] На 5 дзеляцца тыя цэлыя лікі, якія сканчваюцца лічбамі 0 альбо 5; \\
\ruText{ На 5 делятся те целые числа, которые оканчиваются цифрой 0 или цифрой 5; }

\item[(д)] Дзве простыя на плоскасці паралельныя тады, калі яны перпендыкулярныя адной і той жа простай; \\
\ruText{ Две прямые на плоскости тогда параллельны, когда они перпендикулярны одной и той же прямой; }

\item[(е)] Камплексныя лікі роўныя, толькі калі роўныя адпаведна іх сапраўдная і ўяўная часткі; \\
\ruText{ Комплексные числа равны, только если равны соответственно их действительные и мнимые части; }

\item[(ж)] Любое квадратнае раўнанне з рэчаіснымі каэфіцыентамі мае не больш за два рэчаісных кораня; \\
\ruText{ Всякое квадратное уравнение с действительными коэффициентами имеет не более двух действительных корней; }

\item[(з)] З таго, што чатырохкутнік --- ромб, вынікае, што кожная з яго дыяганаляў з'яўлецца воссю сіметрыі; \\
\ruText{ Из того, что четырехугольник --- ромб, следует, что каждая из его диагоналей служит его осью симметрии; }

\item[(і)] Цотнасць сумы з'яўляецца неабходнай умовай цотнасці кожнага складніка; \\
\ruText{ Четность суммы есть необходимое условие четности каждого слагаемого; }

\item[(к)] Роўнасць трыкутнікаў з'яўляецца дастатковай умовай іх роўнавялікасці; \\
\ruText{ Равенство треугольников есть достаточное условие их равновеликости; }

\item[(л)] Для падзельнасці здабытку на нейкі лік дастаткова, каб прынамсі адзін з множнікаў дзяліўся на гэты лік. \\
\ruText{ Для делимости произведения на некоторое число достаточно, чтобы по меньшей мере один из сомножителей делился на это число. }

\end{itemize}

\item[2.] Няхай $A$, $B$ і $C$ абазначаюць, адпаведна, наступныя сказы:
\textit{\guillemotleft Ён чытае коміксы.\guillemotright},
\textit{\guillemotleft Ён любіць навуковую фантастыку.\guillemotright},
\textit{\guillemotleft Ён --- студэнт-інфарматык.\guillemotright}.
Запішыце у сімвалічнай форме выказванне: \textit{\guillemotleft Калі ён чытае коміксы,
то ён любіць навуковую фантастыку і калі ён не чытае коміксы, то ён ---
студэнт-інфарматык.\guillemotright} Запішыце адмаўленне гэтага выразу і прадстаўце яго ў выглядзе формулы,
якая змяшчае толькі аперацыі дыз'юнкцыі, кан'юнкцыі і адмаўлення, прычым адмаўленні могуць распаўсюджвацца
толькі на прапазіцыйныя зменныя.

\noindent
\ruText{ Пусть $A$, $B$ и $C$ обозначают соответственно следующие предложения:
\textit{\guillemotleft Он читает комиксы.\guillemotright},
\textit{\guillemotleft Он любит научную фантастику.\guillemotright},
\textit{\guillemotleft Он студент-информатик.\guillemotright}.
Запишите в символической форме высказывание: \textit{\guillemotleft Если он читает
комиксы, то он любит научную фантастику и если он не читает комиксы, то он ---
студент-информатик.\guillemotright} Запишите отрицание этого выражения и представьте
его в виде формулы, которая содержит только операции дизъюнкции, конъюнкции и
отрицания, причем отрицания распространяются только на пропозиционные переменные. }

\smallskip

\item[3.] Пабудуйце табліцы праўдзівасці наступных формул: \\
\ruText{Постройте таблицы истинности следующих формул:}
\begin{itemize}
\item[(а)] $(A \to B) \vee (A \to (A \cdot B))$.

\item[(б)] $((A \thicksim B) \to \overline{C}) \cdot (A \vee C)$.

\item[(в)] $(((\overline{A \vee B}) \cdot \overline{C}) \to \overline{B}) \thicksim A$.

\item[(г)] $((\overline{A} \cdot \overline{B}) \to (\overline{\overline{B} \to \overline{A}}))
\cdot ((A \vee B) \thicksim C)$.
\end{itemize}

\smallskip

\item[4.] Рашыце наступныя лагічныя раўнанні: \\
\ruText{ Решите следующие логические уравнения: }
\begin{itemize}
\item[(а)] $(A \to C) \cdot (\overline{(B \to C) \to ((A \vee B) \to C)}) = \text{П}$.

\item[(б)] $((\overline{A \cdot B}) \thicksim C) \to (C \vee \overline{A}) = \text{??}$.

\item[(в)] $(\overline{A \to \overline{B}}) \to ((\overline{A \vee (B \thicksim A)}) \to C) = \text{П}$.

\item[(г)] $((\overline{A \thicksim B}) \cdot (\overline{A \thicksim C})) \to
(\overline{A \thicksim (B \cdot D)}) = \text{??}$.
\end{itemize}

\smallskip

\item[5.] Дакажыце наступныя раўназначнасці без выкарыстання табліц праўдзівасці: \\
\ruText{ Докажите следующие равносильности без использования таблиц истинности: }
\begin{itemize}
\item[(а)] $(A \cdot (B \vee \overline{C})) \vee \overline{A} \vee (B \cdot C) \vee
(A \cdot \overline{C}) \equiv \overline{A} \vee B \vee \overline{C}$.

\item[(б)] $(((((A \to B) \to \overline{A}) \to \overline{B}) \to \overline{C}) \to C)
\equiv C$.

\item[(в)] $((\overline{(A \cdot B) \to C}) \to (\overline{A \cdot C})) \to ((A \cdot B) \to
(\overline{A \cdot (B \to C)})) \equiv \overline{A} \vee \overline{B} \vee \overline{C}$.

\item[(г)] $A \to ((A \cdot B) \to (((A \to B) \to B) \cdot C)) \equiv B \to (A \to C)$.
\end{itemize}

\smallskip

\item[6.] Знайдзіце такую формулу $\boldsymbol{\Phi}$, што: \\
\ruText{ Найдите такую формулу $\boldsymbol{\Phi}$, что: }
\begin{itemize}
\item[(а)] $\vDash ((\boldsymbol{\Phi} \cdot A) \to \overline{B}) \to
((B \to \overline{A}) \to \boldsymbol{\Phi})$;

\item[(б)] $\vDash (\boldsymbol{\Phi} \to (A \to (B \to C))) \thicksim (\boldsymbol{\Phi} \to (A \to B))$;

\item[(в)] $\vDash (\boldsymbol{\Phi} \cdot (A \vee (B \to C))) \thicksim \boldsymbol{\Phi}$;

\item[(г)] $\vDash ((A \vee \boldsymbol{\Phi}) \thicksim A) \thicksim (\overline{A} \to
(B \vee \overline{C}))$.
\end{itemize}

\smallskip

\item[7.] Дакажыце наступныя сцверджанні: \\
\ruText{ Докажите следующие утверждения: }
\begin{itemize}
\item[(а)] калі\, $\vDash \boldsymbol{\mathrm{A}} \vee \boldsymbol{\mathrm{B}}$,\,
$\vDash \overline{\boldsymbol{\mathrm{A}}} \vee \boldsymbol{\mathrm{C}}$,\,
то\, $\vDash \boldsymbol{\mathrm{B}} \vee \boldsymbol{\mathrm{C}}$;

\item[(б)] калі\, $\vDash \boldsymbol{\mathrm{A}} \to \boldsymbol{\mathrm{B}}$,\,
$\vDash \boldsymbol{\mathrm{A}} \cdot \boldsymbol{\mathrm{C}}$,\, то\,
$\vDash \boldsymbol{\mathrm{B}} \cdot \boldsymbol{\mathrm{C}}$;

\item[(в)] калі\, $\vDash \boldsymbol{\mathrm{A}} \vee \boldsymbol{\mathrm{B}}$,\,
$\vDash \boldsymbol{\mathrm{A}} \to \boldsymbol{\mathrm{C}}$,\,
$\vDash \boldsymbol{\mathrm{B}} \to \boldsymbol{\mathrm{D}}$,\, то\,
$\vDash \boldsymbol{\mathrm{C}} \vee \boldsymbol{\mathrm{D}}$;

\item[(г)] калі\, $\vDash \boldsymbol{\mathrm{A}} \cdot \boldsymbol{\mathrm{B}}$,\,
$\vDash \boldsymbol{\mathrm{B}} \thicksim \boldsymbol{\mathrm{C}}$,\, то\,
$\vDash \boldsymbol{\mathrm{D}} \to (\boldsymbol{\mathrm{A}} \cdot \boldsymbol{\mathrm{C}})$.
\end{itemize}

\smallskip

\item[8.] Высветліце, ці справядлівыя наступныя лагічныя вынікі: \\
\ruText{ Выясните, верны ли следующие логические следования: }
\begin{itemize}
\item[(а)] $\boldsymbol{\mathrm{A}} \to \boldsymbol{\mathrm{B}}$,\,
$\boldsymbol{\mathrm{D}} \to \overline{\boldsymbol{\mathrm{C}}}$,\,
$\boldsymbol{\mathrm{C}} \vee \overline{\boldsymbol{\mathrm{B}}}$\,
$\vDash \boldsymbol{\mathrm{A}} \to \overline{\boldsymbol{\mathrm{D}}}$;

\item[(б)] $\boldsymbol{\mathrm{A}} \to \boldsymbol{\mathrm{B}}$,\, $((\boldsymbol{\mathrm{A}}
\vee \boldsymbol{\mathrm{D}}) \cdot \boldsymbol{\mathrm{C}}) \to \boldsymbol{\mathrm{E}}$,\,
$\boldsymbol{\mathrm{D}} \to \boldsymbol{\mathrm{C}}$\, $\vDash ((\boldsymbol{\mathrm{A}} \vee \boldsymbol{\mathrm{D}}) \cdot \boldsymbol{\mathrm{B}}) \to \overline{\boldsymbol{\mathrm{E}}}$;

\item[(в)] $(\boldsymbol{\mathrm{A}} \vee \boldsymbol{\mathrm{B}}) \to (\boldsymbol{\mathrm{C}} \cdot
\boldsymbol{\mathrm{D}})$,\, $(\boldsymbol{\mathrm{D}} \vee \boldsymbol{\mathrm{E}}) \to
\boldsymbol{\mathrm{F}}$\, $\vDash \boldsymbol{\mathrm{A}} \to \boldsymbol{\mathrm{F}}$;

\item[(г)] $(\boldsymbol{\mathrm{A}} \cdot \boldsymbol{\mathrm{B}}) \to \boldsymbol{\mathrm{C}}$,\,
$(\boldsymbol{\mathrm{C}} \cdot \boldsymbol{\mathrm{D}}) \to \boldsymbol{\mathrm{E}}$,\,
$\overline{\boldsymbol{\mathrm{F}}} \to (\boldsymbol{\mathrm{D}} \cdot \boldsymbol{\mathrm{E}})$\,
$\vDash (\boldsymbol{\mathrm{A}} \cdot \boldsymbol{\mathrm{B}}) \to \boldsymbol{\mathrm{F}}$.
\end{itemize}

\smallskip

\item[9.] Высветліце, ці справядлівыя наступныя сцверджанні: \\
\ruText{ Выясните, верны ли следующие утверждения: }
\begin{itemize}
\item[(а)] калі\, $\boldsymbol{\Gamma} \vDash \boldsymbol{\mathrm{A}}$\, і\, $\boldsymbol{\Gamma}
\vDash \boldsymbol{\mathrm{B}}$,\, то\, $\boldsymbol{\Gamma} \vDash \boldsymbol{\mathrm{A}} \cdot
\boldsymbol{\mathrm{B}}$;

\item[(б)] $\boldsymbol{\Gamma} \vDash \boldsymbol{\mathrm{A}} \to \boldsymbol{\mathrm{B}}$\, тады
і толькі тады, калі\, $\boldsymbol{\Gamma},\, \overline{\boldsymbol{\mathrm{A}}} \vDash \boldsymbol{\mathrm{B}}$;

\item[(в)] калі\, $\boldsymbol{\Gamma},\, \boldsymbol{\mathrm{A}} \vDash \boldsymbol{\mathrm{B}}$\, і\,
$\boldsymbol{\Gamma},\, \boldsymbol{\mathrm{A}} \vDash \overline{\boldsymbol{\mathrm{B}}}$,\, то\,
$\boldsymbol{\Gamma} \vDash \overline{\boldsymbol{\mathrm{A}}}$;

\item[(г)] $\boldsymbol{\Gamma} \vDash \boldsymbol{\mathrm{A}} \to (\boldsymbol{\mathrm{B}}
\to \boldsymbol{\mathrm{C}})$\, тады і толькі тады, калі\, $\boldsymbol{\Gamma} \vDash (\boldsymbol{\mathrm{A}} \cdot \boldsymbol{\mathrm{B}}) \to \boldsymbol{\mathrm{C}}$.
\end{itemize}
(Тут $\boldsymbol{\Gamma}$ --- канечнае мноства формул, магчыма пустое.) \\
\ruText{ (Здесь $\boldsymbol{\Gamma}$ --- конечное множество формул, возможно пустое.) }
\end{itemize}

\end{document}
