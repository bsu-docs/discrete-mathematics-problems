\usepackage[utf8]{inputenc}
\usepackage[russian]{babel}
\usepackage{amsfonts,amssymb,amsmath}
\usepackage{float}
\usepackage{epsfig}
\usepackage{mathrsfs}
\usepackage{mathabx}
\usepackage{xcolor}
\usepackage{enumitem}
\usepackage{hyperref}
\usepackage{bbm}
\usepackage{geometry}
\usepackage{ifthen}

\geometry{left=2cm, right=1cm, top=1.5cm, bottom=1.5cm}
\newcommand{\ruText}[1]{
  {\scriptsize \textcolor{darkgray}{#1} \par}
}

\newcommand{\biLangHeader}[2]{
  \subsection*{
    {\normalsize #1} \\
    \indent \ruText{#2}
  }
}


% Lines below are to avoid word breaks.
\tolerance=1
\emergencystretch=\maxdimen
\hyphenpenalty=10000
\hbadness=10000

\pagestyle{empty}

\usepackage{titlesec}
\titleformat{\subsection}[display]{\bfseries\filright}{}{}{}



\begin{document}

\biLangHeader
{10. Размяшчэнні. Перастаноўкі. Спалучэнні.}
{Размещения. Перестановки. Сочетания.}

\biLangHeader
{Размяшчэнні. Перастаноўкі.}
{Размещения. Перестановки.}

\begin{problemList}

\problemItemSimple
{}
{Имеется 25 занумерованных мест. Сколькими способами можно разместить
пятерых человек на этих местах, если каждый из них может занимать
ровно одно место?}

\problemItemSimple
{}
{Учащемуся нужно сдать четыре экзамена на протяжении восьми дней.
Сколькими способами он может это сделать, если в один день разрешается
сдавать не более одного экзамена?}

\problemItemSimple
{}
{На собрании должны выступить пять человек $A$, $B$, $C$, $D$ и $E$,
причем каждый по одному разу. Сколькими способами их можно разместить
в списке ораторов, если: (а) нет ограничений на порядок выступлений;
(б) $B$ выступает сразу за $A$; (в) $B$ не может выступать до того
момента, пока не выступит $A$?}

\problemItemSimple
{}
{Пусть $X = \{1, 2, \ldots, n\}$. Найдите число способов, которыми
можно выбрать подмножества $S$ и $T$ множества $X$ при условии, что:
(а) на выбор $S$ и $T$ нет ограничений; (б) $S \subseteq T$; (в)
$S \cap T = \emptyset$; (г) $S \cap T \ne \emptyset$; (д)
$S \subseteq T$ и $|T|$ -- четное число.}

\problemItemSimple
{}
{Сколькими способами можно рассадить $n$ человек: (а) в ряд; (б) за
круглым столом (два размещения за круглым столом считаются
одинаковыми, если у каждого человека сосед слева один и тот же)?}

\problemItemSimple
{}
{Сколькими способами можно рассадить за круглым столом $n$ мужчин и
$n$ женщин так, чтобы никакие два лица одного пола не сидели рядом?}

\problemItemSimple
{}
{Сколькими способами можно упорядочить множество $\{1, 2, \ldots, n\}$
так, чтобы числа 1, 2 и 3 стояли рядом в порядке возрастания?}

\problemItemSimple
{}
{Сколько существует перестановок множества $\{1, 2, \ldots, n\}$, в
которых два фиксированных элемента $i$ и $j$ не стоят рядом в любом
порядке?}

\problemItemSimple
{}
{Сколькими способами можно упорядочить множество $\{1, 2, \ldots, 2n\}$
так, чтобы каждое четное число имело четный номер?}

\problemItemSimple
{}
{Сколькими способами можно упорядочить множество $\{1, 2, \ldots, n\}$
так, чтобы каждое число, кратное 2, и каждое число, кратное 3, имело
номер, кратный 2 и 3 соответственно?}

\problemItemSimple
{}
{Сколько существует перестановок $(i_1, i_2, \ldots, i_n)$ множества
$\{1, 2, \ldots, n\}$, для которых имеет место неравенство
$i_1 - i_2 > 1$?}

\end{problemList}

\biLangHeader
{}
{Сочетания.}

\begin{problemList}

\problemItemSimple
{}
{На плоскости проведено $n$ прямых $(n \ge 2)$, причем никакие две из
них не параллельны и никакие три не пересекаются в одной точке.
Найдите: (а) число точек пересечения прямых; (б) число треугольников,
которые образуют эти прямые.}

\problemItemSimple
{}
{В скольких точках пересекаются диагонали выпуклого $n$-угольника
$(n \ge 4)$, если никакие три из них не пересекаются в одной точке?}

\problemItemSimple
{}
{В выпуклом $n$-угольнике $(n \ge 4)$ проведены все диагонали, причем
никакие три из них не пересекаются в одной точке. На сколько частей
разделится при этом многоугольник?}

\problemItemSimple
{}
{Монета подброшена 10 раз. Сколько существует способов выпадения
четырех \guillemotleft решек\guillemotright{} и шести \guillemotleft
орлов\guillemotright? Сколько существует способов выпадения не менее
трех \guillemotleft решек\guillemotright?}

\problemItemSimple
{}
{Среди 20 человек 10 являются физиками, а другие 10 -- математиками.
Найдите число способов, которыми можно выбрать четверку людей так,
чтобы в нее вошел по крайней мере один специалист из каждой области.}

\problemItemSimple
{}
{Определите число пятизначных натуральных чисел, десятичная запись
которых содержит ровно две различные цифры.}

\problemItemSimple
{}
{Сколько существует шестизначных натуральных чисел, десятичная запись
которых состоит из трех четных и трех нечетных цифр? Определите число
$2n$-значных $(n \ge 1)$ натуральных чисел, десятичная запись которых
состоит из $n$ четных и $n$ нечетных цифр.}

\problemItemSimple
{}
{Сколькими способами можно распределить 33 различные книги между тремя
людьми $A$, $B$ и $C$ так, чтобы $A$ и $B$ вместе получили книг в два
раза больше, чем $C$?}

\problemItemSimple
{}
{Сколькими способами можно выбрать три различных числа из множества
$\{1, 2, \ldots, 100\}$ так, чтобы их сумма делилась на 3? Тот же
вопрос для множества $\{1, 2, \ldots, n\}$.}

\problemItemSimple
{}
{Сколькими способами можно выбрать три различных числа из множества
$\{1, 2, \ldots, 500\}$ так, чтобы одно из этих чисел было средним
арифметическим двух других? Тот же вопрос для множества
$\{1, 2, \ldots, n\}$.}

\problemItemSimple
{}
{Дана квадратная решетка со сторонами $n$ и $n$, где $n \ge 6$. Найдите
число различных кратчайших путей из точки $O(0; 0)$ в точку $A(n; n)$,
проходящих по сторонам решетки при условии, что путь содержит точку
$B(n - 2; n - 2)$ и не проходит через точку $C(n - 3; 1)$.}

\problemItemSimple
{}
{Колода из $4n$ карт содержит четыре масти и $n$ ($n \ge 5$) карт в
каждой масти, занумерованных числами $1, 2, \ldots, n$. Сколькими
способами можно выбрать пять карт так, чтобы среди них оказались: (а)
лишь карты одинаковой масти; (б) ровно четыре карты одной масти; (в)
как минимум четыре карты одной масти; (г) карты всех мастей; (д) ровно
три карты с одним и тем же номером; (е) две карты с одинаковыми, а
остальные три -- с различными номерами?}

\problemItemSimple
{}
{Сколькими способами можно выбрать 6 карт из колоды, содержащей 52
карты так, чтобы среди них были карты каждой масти?}

\problemItemSimple
{}
{Сколько существует бинарных векторов длины $m + n$, содержащих $m$
единиц и $n$ нулей, в которых никакие две единицы не идут подряд
($m \le n + 1$)?}

\problemItemSimple
{}
{Сколькими способами можно рассадить за круглым столом $m$ мужчин и $n$
женщин ($m \le n$) на $m + n$ занумерованных местах так, чтобы никакие
два мужчины не сидели рядом?}

\problemItemSimple
{}
{Сколькими способами можно составить треугольники, длины сторон которых
являются натуральными числами, если длина каждой стороны больше $n$ и
не больше $2n$?}

\problemItemSimple
{}
{Бросают $n$ одинаковых игральных костей, каждая из которых помечена
очками $1, 2, 3, 4, 5, 6$. Сколькими способами могут выпасть кости?
Во скольких случаях: (а) хотя бы на одной из костей выпадет 6 очков;
(б) ровно на одной из костей выпадет 6 очков; (в) на одной из костей
выпадет 1 очко, а на другой -- 2 очка?}

\problemItemSimple
{}
{Сколько существует $n$-значных ($n \ge 1$) натуральных чисел, в
которых цифры расположены в неубывающем порядке?}

\problemItemSimple
{}
{Функция $f \colon \{1, 2, \ldots, n\} \to \{1, 2, \ldots, n\}$
называется \emph{монотонной}, если $f(x) \le f(y)$ для любых $x < y$.
Определите число монотонных функций указанного вида.}

\problemItemSimple
{}
{Пусть $r$ -- целое неотрицательное число. Найдите число решений в
целых неотрицательных числах (натуральных числах): (а) уравнения
$x_1 + x_2 + \ldots + x_n = r$; (б) неравенства
$x_1 + x_2 + \ldots + x_n \le r$?}

\problemItemSimple
{}
{Сколькими способами можно разместить: (а) $r$ различных шаров по $n$
различным коробкам; (б) $r$ одинаковых шаров по $n$ различным
коробкам?}

\problemItemSimple
{}
{Пусть $r$ -- целое неотрицательное число. Определите число
целочисленных решений неравенства: (а) $|x_1| + |x_2| \le 1000$;
(б) $|x_1| + |x_2| + \ldots + |x_n| \le r$.}

\problemItemSimple
{}
{Найдите число способов, которыми можно разложить число 1728 в
произведение трех натуральных множителей при условии, что разложения,
отличающиеся порядком следования множителей, считаются различными.}

\problemItemSimple
{}
{Сколькими способами три человека могут разделить между собой 6
одинаковых яблок, 1 апельсин, 1 мандарин, 1 лимон, 1 грушу, 1 персик
и 1 абрикос при условии, что: (а) количество плодов, получаемых одним
человеком, не ограничено; (б) каждый получает ровно по 4 плода.}

\problemItemSimple
{}
{Сколькими способами можно выбрать $k$ из $n$ расположенных в ряд
предметов $x_1, x_2, \dots, x_n$ так, чтобы при этом не были выбраны
никакие два соседних предмета ($n \ge 2k - 1$)?}

\problemItemSimple
{}
{На книжной полке в ряд расположено $n$ книг. Сколькими способами из
них можно выбрать $p$ книг так, чтобы между любыми двумя выбранными
книгами, равно как и после $p$-ой (последней) выбранной книги,
располагалось не менее $s$ книг ($p(s + 1) \le n$)?}

\end{problemList}

\end{document}
