\usepackage[utf8]{inputenc}
\usepackage[russian]{babel}
\usepackage{amsfonts,amssymb,amsmath}
\usepackage{float}
\usepackage{epsfig}
\usepackage{mathrsfs}
\usepackage{mathabx}
\usepackage{xcolor}
\usepackage{enumitem}
\usepackage{hyperref}
\usepackage{bbm}
\usepackage{geometry}
\usepackage{ifthen}

\geometry{left=2cm, right=1cm, top=1.5cm, bottom=1.5cm}
\newcommand{\ruText}[1]{
  {\scriptsize \textcolor{darkgray}{#1} \par}
}

\newcommand{\biLangHeader}[2]{
  \subsection*{
    {\normalsize #1} \\
    \indent \ruText{#2}
  }
}


% Lines below are to avoid word breaks.
\tolerance=1
\emergencystretch=\maxdimen
\hyphenpenalty=10000
\hbadness=10000

\pagestyle{empty}

\usepackage{titlesec}
\titleformat{\subsection}[display]{\bfseries\filright}{}{}{}



\begin{document}

\biLangHeader
{10. Размяшчэнні. Перастаноўкі. Спалучэнні.}
{Размещения. Перестановки. Сочетания.}

\biLangHeader
{Размяшчэнні. Перастаноўкі.}
{Размещения. Перестановки.}

\begin{problemList}

\problemItemSimple
{Ёсць 25 занумараваных месцаў. Колькі існуе спосабаў размясціць пяць чалавек на гэтых месцах,
калі кожны з іх можа займаць роўна адно месца?}
{Имеется 25 занумерованных мест. Сколькими способами можно разместить
пятерых человек на этих местах, если каждый из них может занимать
ровно одно место?}

\bigskip

\problemItemSimple
{Навучэнцу трэба здаць чатыры іспыты на працягу васьмі дзён. Колькі існуе спосабаў гэта зрабіць,
калі ў адзін дзень дазваляецца здаваць не больш за адзін іспыт?}
{Учащемуся нужно сдать четыре экзамена на протяжении восьми дней.
Сколькими способами он может это сделать, если в один день разрешается
сдавать не более одного экзамена?}

\bigskip

\problemItemSimple
{На сходзе павінны выступіць пяць чалавек: $A$, $B$, $C$, $D$ і $E$,
прычым кожны па адным разе. Колькі існуе спосабаў размясціць іх у спісе аратараў,
калі: (а) няма абмежаванняў на парадак выступаў; (б) $B$ выступае адразу пасля $A$;
(в) $B$ не можа выступаць, пакуль не выступіць $A$?}
{На собрании должны выступить пять человек: $A$, $B$, $C$, $D$ и $E$,
причем каждый по одному разу. Сколькими способами их можно разместить
в списке ораторов, если: (а) нет ограничений на порядок выступлений;
(б) $B$ выступает сразу за $A$; (в) $B$ не может выступать до того
момента, пока не выступит $A$?}

\bigskip

\problemItemSimple
{Няхай $X = \{1, 2, \ldots, n\}$. Знайдзіце колькасць спосабаў, якімі можна абраць
падмноствы $S$ і $T$ мноства $X$ пры ўмове, што:
(а) на выбар $S$ і $T$ няма абмежаванняў; (б) $S \subseteq T$; (в)
$S \cap T = \emptyset$; (г) $S \cap T \ne \emptyset$; (д)
$S \subseteq T$ і $|T|$ "--- цотны лік.}
{Пусть $X = \{1, 2, \ldots, n\}$. Найдите число способов, которыми
можно выбрать подмножества $S$ и $T$ множества $X$ при условии, что:
(а) на выбор $S$ и $T$ нет ограничений; (б) $S \subseteq T$; (в)
$S \cap T = \emptyset$; (г) $S \cap T \ne \emptyset$; (д)
$S \subseteq T$ и $|T|$ "--- четное число.}

\bigskip

\problemItemSimple
{Колькі існуе спосабаў рассадзіць $n$ чалавек: (а) у шэраг;
(б) за круглым сталом (два размяшчэнні за круглым сталом лічацца аднолькавымі,
калі ў кожнага чалавека сусед злева адзін і той жа)?}
{Сколькими способами можно рассадить $n$ человек: (а) в ряд; (б) за
круглым столом (два размещения за круглым столом считаются
одинаковыми, если у каждого человека сосед слева один и тот же)?}

\bigskip

\problemItemSimple
{Колькі існуе спосабаў рассадзіць за круглым сталом $n$ мужчын і $n$ жанчын так,
каб ніякія дзве асобы аднаго полу не сядзелі побач?}
{Сколькими способами можно рассадить за круглым столом $n$ мужчин и
$n$ женщин так, чтобы никакие два лица одного пола не сидели рядом?}

\bigskip

\problemItemSimple
{Колькі існуе спосабаў упарадкаваць мноства $\{1, 2, \ldots, n\}$ так,
каб лікі 1, 2 і 3 стаялі побач у парадку ўзрастання?}
{Сколькими способами можно упорядочить множество $\{1, 2, \ldots, n\}$
так, чтобы числа 1, 2 и 3 стояли рядом в порядке возрастания?}

\bigskip

\problemItemSimple
{Колькі існуе перастановак мноства $\{1, 2, \ldots, n\}$, у якіх два зафіксаваныя
элементы $i$ і $j$ не стаяць побач у любым парадку?}
{Сколько существует перестановок множества $\{1, 2, \ldots, n\}$, в
которых два фиксированных элемента $i$ и $j$ не стоят рядом в любом
порядке?}

\bigskip

\problemItemSimple
{Колькі існуе спосабаў упарадкаваць мноства $\{1, 2, \ldots, 2n\}$ так,
каб кожны цотны лік меў няцотны парадкавы нумар?}
{Сколькими способами можно упорядочить множество $\{1, 2, \ldots, 2n\}$
так, чтобы каждое четное число имело четный номер?}

\bigskip

\problemItemSimple
{Колькі існуе спосабаў упарадкаваць мноства $\{1, 2, \ldots, n\}$ так,
каб кожны лік, кратны 2, і кожны лік, кратны 3, мелі парадкавыя нумары кратныя,
адпаведна, 2 і 3?}
{Сколькими способами можно упорядочить множество $\{1, 2, \ldots, n\}$
так, чтобы каждое число, кратное 2, и каждое число, кратное 3, имело
номер, кратный 2 и 3 соответственно?}

\bigskip

\problemItemSimple
{Колькі існуе перастановак $(i_1, i_2, \ldots, i_n)$ мноства $\{1, 2, \ldots, n\}$,
для якіх справядлівая няроўнасць $i_1 - i_2 > 1$?}
{Сколько существует перестановок $(i_1, i_2, \ldots, i_n)$ множества
$\{1, 2, \ldots, n\}$, для которых имеет место неравенство
$i_1 - i_2 > 1$?}

\end{problemList}

\biLangHeader
{Спалучэнні.}
{Сочетания.}

\begin{problemList}

\problemItemSimple
{На плоскасці праведзеныя $n$ прамых $(n \ge 2)$, прычым ніякія дзве з іх не паралельныя
і ніякія тры не перакрыжоўваюцца ў адной кропцы. Знайдзіце: (а) колькасць кропак перакрыжавання прамых;
(б) колькасць трыкутнікаў, утвораных гэтымі прамымі.}
{На плоскости проведено $n$ прямых $(n \ge 2)$, причем никакие две из
них не параллельны и никакие три не пересекаются в одной точке.
Найдите: (а) число точек пересечения прямых; (б) число треугольников,
которые образуют эти прямые.}

\bigskip

\problemItemSimple
{У якой колькасці кропак перакрыжоўваюцца дыяганалі выпуклага $n$-кутніка
$(n \ge 4)$, калі ніякія тры з іх не перакрыжоўваюцца ў адной кропцы?}
{В скольких точках пересекаются диагонали выпуклого $n$-угольника
$(n \ge 4)$, если никакие три из них не пересекаются в одной точке?}

\bigskip

\problemItemSimple
{У выпуклым $n$-кутніку $(n \ge 4)$ праведзеныя ўсе дыяганалі, прычым
ніякія тры з іх не перакрыжоўваюцца ў адной кропцы. На колькі частак разаб'ецца
пры гэтым многакутнік?}
{В выпуклом $n$-угольнике $(n \ge 4)$ проведены все диагонали, причем
никакие три из них не пересекаются в одной точке. На сколько частей
разделится при этом многоугольник?}

\bigskip

\problemItemSimple
{Манета была падкінутая 10 разоў. Колькі існуе спосабаў выпадання чатырох
<<рэшак>> і шасці <<арлоў>>?
Колькі існуе спосабаў выпадання не менш за тры <<рэшкі>>?}
{Монета подброшена 10 раз. Сколько существует способов выпадения
четырех <<решек>> и шести <<орлов>>? Сколько существует способов выпадения
не менее трех <<решек>>?}

\bigskip

\problemItemSimple
{Сярод 20 чалавек, 10 "--- фізікі, а іншыя 10 "--- матэматыкі. Знайдзіце колькасць спосабаў,
якімі можна абраць чацвёрку людзей так, каб у яе ўвайшоў прынамсі адзін спецыяліст з кожнай галіны?}
{Среди 20 человек 10 являются физиками, а другие 10 "--- математиками.
Найдите число способов, которыми можно выбрать четверку людей так,
чтобы в нее вошел по крайней мере один специалист из каждой области.}

\bigskip

\problemItemSimple
{Вызначыце колькасць пяцізначных натуральных лікаў, дзесятковы запіс якіх змяшчае роўна дзве адрозныя лічбы?}
{Определите число пятизначных натуральных чисел, десятичная запись
которых содержит ровно две различные цифры.}

\bigskip

\problemItemSimple
{Колькі існуе шасцізначных натуральных лікаў, дзесятковы запіс якіх змяшчае тры цотныя
і тры няцотныя лічбы? Вызначыце колькасць $2n$-значных натуральных лікаў,
дзесятковы запіс якіх змяшчае $n$ цотных і $n$ няцотных лічбаў.}
{Сколько существует шестизначных натуральных чисел, десятичная запись
которых состоит из трех четных и трех нечетных цифр? Определите число
$2n$-значных натуральных чисел, десятичная запись которых
состоит из $n$ четных и $n$ нечетных цифр.}

\bigskip

\problemItemSimple
{Колькі існуе спосабаў размеркаваць 33 адрозныя кнігі паміж трыма людзьмі
$A$, $B$ і $C$ так, каб $A$ і $B$ разам атрымалі кніг у два разы болей, чым $C$?}
{Сколькими способами можно распределить 33 различные книги между тремя
людьми $A$, $B$ и $C$ так, чтобы $A$ и $B$ вместе получили книг в два
раза больше, чем $C$?}

\bigskip

\problemItemSimple
{Колькі існуе спосабаў абраць тры адрозныя лікі з мноства $\{1, 2, \ldots, 100\}$ так,
каб іх сума дзялілася на 3? Тое ж пытанне для мноства $\{1, 2, \ldots, n\}$.}
{Сколькими способами можно выбрать три различных числа из множества
$\{1, 2, \ldots, 100\}$ так, чтобы их сумма делилась на 3? Тот же
вопрос для множества $\{1, 2, \ldots, n\}$.}

\bigskip

\problemItemSimple
{Колькі існуе спосабаў абраць тры адрозныя лікі з мноства $\{1, 2, \ldots, 500\}$ так,
каб адзін з гэтых лікаў быў сярэднім арыфметычным двух іншых?
Тое ж пытанне для мноства $\{1, 2, \ldots, n\}$.}
{Сколькими способами можно выбрать три различных числа из множества
$\{1, 2, \ldots, 500\}$ так, чтобы одно из этих чисел было средним
арифметическим двух других? Тот же вопрос для множества
$\{1, 2, \ldots, n\}$.}

\bigskip

\problemItemSimple
{Дадзеная квадратная рашотка з бакамі $n$ і $n$, дзе $n \ge 6$. Знайдзіце
колькасць усіх магчымых найкарацейшых шляхоў з кропкі $O(0; 0)$ у кропку $A(n; n)$,
якія праходзяць па баках рашоткі пры ўмове, што кожны шлях змяшчае кропку
$B(n - 2; n - 2)$ і не праходзіць праз кропку $C(n - 3; 1)$.}
{Дана квадратная решетка со сторонами $n$ и $n$, где $n \ge 6$. Найдите
число различных кратчайших путей из точки $O(0; 0)$ в точку $A(n; n)$,
проходящих по сторонам решетки при условии, что путь содержит точку
$B(n - 2; n - 2)$ и не проходит через точку $C(n - 3; 1)$.}

\bigskip

\problemItemSimple
{Калода з $4n$ карт змяшчае чатыры масці і $n$ ($n \ge 5$) карт у кожнай масці,
занумараваныя лікамі $1, 2, \ldots, n$. Колькі існуе спосабаў абраць пяць карт так,
каб сярод іх апынуліся: (а) толькі карты аднолькавай масці; (б) роўна чатыры карты
аднолькавай масці; (в) як мінімум чатыры карты адной масці; (г) карты ўсіх масцей;
(д) роўна тры карты з адным і тым жа нумарам; (е) дзве карты з аднолькавымі, а астатнія тры "---
з адрознымі нумарамі?}
{Колода из $4n$ карт содержит четыре масти и $n$ ($n \ge 5$) карт в
каждой масти, занумерованных числами $1, 2, \ldots, n$. Сколькими
способами можно выбрать пять карт так, чтобы среди них оказались: (а)
лишь карты одинаковой масти; (б) ровно четыре карты одной масти; (в)
как минимум четыре карты одной масти; (г) карты всех мастей; (д) ровно
три карты с одним и тем же номером; (е) две карты с одинаковыми, а
остальные три "--- с различными номерами?}

\bigskip

\problemItemSimple
{Колькі існуе спосабаў абраць 6 карт з калоды, якая змяшчае 52 карты, так,
каб сярод іх былі карты кожнай масці?}
{Сколькими способами можно выбрать 6 карт из колоды, содержащей 52
карты так, чтобы среди них были карты каждой масти?}

\bigskip

\problemItemSimple
{Колькі існуе бінарных вектараў даўжыні $m + n$, якія змяшчаюць $m$ адзінак і
$n$ нулёў, у якіх ніякія дзве адзінкі не ідуць запар ($m \le n + 1$)?}
{Сколько существует бинарных векторов длины $m + n$, содержащих $m$
единиц и $n$ нулей, в которых никакие две единицы не идут подряд
($m \le n + 1$)?}

\bigskip

\problemItemSimple
{Колькі існуе спосабаў рассадзіць за круглым сталом $m$ мужчын і $n$ жанчын
($m \le n$) на $m + n$ занумараваных месцах так, каб ніякія два мужчыны не сядзелі побач?}
{Сколькими способами можно рассадить за круглым столом $m$ мужчин и $n$
женщин ($m \le n$) на $m + n$ занумерованных местах так, чтобы никакие
два мужчины не сидели рядом?}

\bigskip

\problemItemSimple
{Колькі існуе спосабаў пабудаваць трыкутнікі, даўжыні бакоў якіх з'яўляюцца натуральнымі лікамі,
калі даўжыня кожнага боку больш за $n$ і не больш за $2n$?}
{Сколькими способами можно составить треугольники, длины сторон которых
являются натуральными числами, если длина каждой стороны больше $n$ и
не больше $2n$?}

\bigskip

\problemItemSimple
{Кідаюць $n$ аднолькавых ігральных касцей, кожная з якіх пазначаная ачкамі $1, 2, 3, 4, 5, 6$.
Колькі існуе разнастайных спосабаў выпасці касцям? У якой колькасці выпадкаў:
(а) хаця б на адной косці выпадзе 6 ачкоў; (б) роўна на адной косці выпадзе 6 ачкоў;
(в) на адной з касцей выпадзе 1 ачко, а на іншай "--- 2 ачкі?}
{Бросают $n$ одинаковых игральных костей, каждая из которых помечена
очками $1, 2, 3, 4, 5, 6$. Сколькими способами могут выпасть кости?
Во скольких случаях: (а) хотя бы на одной из костей выпадет 6 очков;
(б) ровно на одной из костей выпадет 6 очков; (в) на одной из костей
выпадет 1 очко, а на другой "--- 2 очка?}

\bigskip

\problemItemSimple
{Колькі існуе $n$-значных натуральных лікаў, у якіх лічбы размешчаныя ў неспадальным парадку?}
{Сколько существует $n$-значных натуральных чисел, в
которых цифры расположены в неубывающем порядке?}

\bigskip

\problemItemSimple
{Функцыя $f \colon \{1, 2, \ldots, n\} \to \{1, 2, \ldots, n\}$
называецца \emph{манатоннай}, калі $f(x) \le f(y)$ для любых $x < y$.
Вызначыце колькасць манатонных функцый указанага тыпу.}
{Функция $f \colon \{1, 2, \ldots, n\} \to \{1, 2, \ldots, n\}$
называется \emph{монотонной}, если $f(x) \le f(y)$ для любых $x < y$.
Определите число монотонных функций указанного вида.}

\bigskip

\problemItemSimple
{Няхай $r$ "--- цэлы неадмоўны лік. Знайдзіце колькасць рашэнняў у цэлых
неадмоўных ліках: (а) раўнання $x_1 + x_2 + \ldots + x_n = r$;
(б) няроўнасці $x_1 + x_2 + \ldots + x_n \le r$?}
{Пусть $r$ "--- целое неотрицательное число. Найдите число решений в
целых неотрицательных числах: (а) уравнения
$x_1 + x_2 + \ldots + x_n = r$; (б) неравенства
$x_1 + x_2 + \ldots + x_n \le r$?}

\bigskip

\problemItemSimple
{Колькі існуе спосабаў размясціць: (а) $r$ адрозных шароў па $n$
адрозных скрынях; (б) $r$ аднолькавых шароў па $n$ адрозных скрынях?}
{Сколькими способами можно разместить: (а) $r$ различных шаров по $n$
различным коробкам; (б) $r$ одинаковых шаров по $n$ различным коробкам?}

\bigskip

\problemItemWithCommonPart
{Няхай $r$ "--- цэлы неадмоўны лік. Вызначыце колькасць цэлалікавых рашэнняў няроўнасці:}
{Пусть $r$ "--- целое неотрицательное число. Определите число
целочисленных решений неравенства:}
{\begin{belarusianEnumerate}
  \item $|x_1| + |x_2| \le 1000$;
  \item $|x_1| + |x_2| + \ldots + |x_n| \le r$.
\end{belarusianEnumerate}}

\bigskip

\problemItemSimple
{Знайдзіце колькасць спосабаў, якімі можна раскласці лік 1728 у здабытак
трох натуральных множнікаў пры ўмове, што раскладанні, якія адрозніваюцца парадкам
следавання множнікаў, лічацца адрознымі.}
{Найдите число способов, которыми можно разложить число 1728 в
произведение трех натуральных множителей при условии, что разложения,
отличающиеся порядком следования множителей, считаются различными.}

\bigskip

\problemItemSimple
{Колькі існуе спосабаў тром людзям падзяліць паміж сабой 6 аднолькавых яблыкаў,
1 апельсін, 1 мандарын, 1 лімон, 1 грушу, 1 персік і 1 абрыкос пры ўмове, што:
(а) колькасць пладоў, якія атрымае адзін чалавек, не абмежаваная; (б) кожны атрымае
роўна па чатыры плады.}
{Сколькими способами три человека могут разделить между собой 6
одинаковых яблок, 1 апельсин, 1 мандарин, 1 лимон, 1 грушу, 1 персик
и 1 абрикос при условии, что: (а) количество плодов, получаемых одним
человеком, не ограничено; (б) каждый получает ровно по 4 плода.}

\bigskip

\problemItemSimple
{Колькі існуе спосабаў абраць $k$ з $n$ размешчаных у шэраг прадметаў
$x_1, x_2, \dots, x_n$ так, каб пры гэтым не былі абраныя ніякія два
суседнія прадметы ($n \ge 2k - 1$)?}
{Сколькими способами можно выбрать $k$ из $n$ расположенных в ряд
предметов $x_1, x_2, \dots, x_n$ так, чтобы при этом не были выбраны
никакие два соседних предмета ($n \ge 2k - 1$)?}

\bigskip

\problemItemSimple
{На кніжнай паліцы ў шэраг размешчаныя $n$ кніг. Колькі існуе спосабаў выбраць
сярод іх $p$ кніг так, каб паміж любымі дзвюма выбранымі кнігамі, у тым ліку
і пасля $p$-ай (апошняй) выбранай кнігі, змяшчалася не менш за $s$ кніг
($p(s + 1) \le n$)?}
{На книжной полке в ряд расположено $n$ книг. Сколькими способами из
них можно выбрать $p$ книг так, чтобы между любыми двумя выбранными
книгами, равно как и после $p$-ой (последней) выбранной книги,
располагалось не менее $s$ книг ($p(s + 1) \le n$)?}

\end{problemList}

\end{document}
