\documentclass[12pt,twoside]{article}

\setlength{\textwidth}{166mm}
\setlength{\textheight}{232mm}
\setlength{\topmargin}{-10mm}
\setlength{\headsep}{5mm}
\oddsidemargin=3mm
\evensidemargin=3mm
\setlength{\baselineskip}{18pt}

\usepackage[utf8]{inputenc}
\usepackage[russian]{babel}
\usepackage{amsfonts,amssymb,amsmath}
\usepackage{epsfig}
\usepackage{mathrsfs}
\usepackage{mathabx}
\usepackage{xcolor}

\renewcommand\le{\leqslant}
\renewcommand\ge{\geqslant}

\newcommand{\ruText}[1]{
  {\footnotesize \textcolor{darkgray}{#1} \par}
}

\newcommand{\biLangHeader}[2]{
  \subsection*{%
  	#1 \\%
  	\ruText{#2}%
  }%
}

\newcommand{\quizTitle}[3]{%
\begin{center}
	\textbf{Кантрольная работа па тэме <<#1>> (варыянт #3)} \\
	\ruText{Контрольная работа по теме <<#2>> (вариант #3)}
\end{center}
}

\newcommand{\problemItemSimple}[2]{%
	\item #1 \\%
	\ruText{#2}%
}

\newcommand{\problemItemWithCommonPart}[3]{%
	\item #1 \\%
	\ruText{#2}%
	#3%
}

\newcommand{\problemItemWithCommonPartComplicated}[5]{%
	\item #1 \\%
	\ruText{#2}%
	#3 \\
	\noindent #4 \\%
	\noindent \ruText{#5}%
}

\makeatletter
\def\belarusianLetters#1{
  \expandafter\@belarusianLetters\csname c@#1\endcsname
}
\def\@belarusianLetters#1{
  (%
  \ifcase#1\or а\or б\or в\or г\or д\or е\or ж\or з\or і\or к\or л\or м\fi%
  )
}
\makeatother
\AddEnumerateCounter{\belarusianLetters}{\@belarusianLetters}{Ы}

\newenvironment{problemList}
  {\begin{enumerate}[leftmargin=*,topsep=0pt,itemsep=-1ex,partopsep=1ex,parsep=1ex]}
  {\end{enumerate}}

\newenvironment{belarusianEnumerate}
  {\begin{enumerate}[label=\belarusianLetters*, topsep=-7pt]}
  {\end{enumerate} \textbf{}\vspace{-8pt}}

\AddEnumerateCounter{\asbuk}{\@asbuk}{\cyrm}
\newenvironment{russianEnumerate}
  {\begin{enumerate}[label=(\asbuk*), topsep=-4pt, itemsep=-1ex]}
  {\end{enumerate} \textbf{}\vspace{-11pt}}


% Lines below are to avoid word breaks.
\tolerance=1
\emergencystretch=\maxdimen
\hyphenpenalty=10000
\hbadness=10000

\renewenvironment{itemize}
{\begin{list}
             {\labelitemi}%                     Old parameters:
             {\setlength{\labelwidth}{1.3em}%        1em
              \setlength{\labelsep}{0.7em}%          0.7em
              \setlength{\itemindent}{0em}%          0em
              \setlength{\listparindent}{3em}%       3em
              \setlength{\leftmargin}{2em}%          3em !
              \setlength{\rightmargin}{0em}%         0em
              \setlength{\parsep}{0ex}%              0ex
              \setlength{\topsep}{0.5ex}%            2ex !
              \setlength{\itemsep}{1ex}%             0ex
             }
}
{\end{list}}

\pagestyle{empty}


\begin{document}

\biLangHeader
{11. Перастаноўкі з паўтарэннямі.}
{Перестановки с повторениями.}

\begin{problemList}

\problemItemSimple
{%
Вызначыце колькасць адрозных слоў (не абавязкова тых, якія маюць сэнс), якія можна атрымаць
праз перастаноўку літар у слове <<FLOCCINAUCINIHILIPILIFICATION>> так, каб:
\begin{belarusianEnumerate}
    \item сем літар <<I>> не ішлі запар;
    \item ніякія дзве літары <<I>> не ішлі запар;
    \item ніякія тры літары <<I>> не ішлі запар.
\end{belarusianEnumerate}
}
{%
Определите, сколько различных слов (не обязательно имеющих смысл) можно
получить, переставляя буквы в слове <<FLOCCINAUCINIHILIPILIFICATION>>
так, чтобы:
\begin{russianEnumerate}
    \item семь букв <<I>> не следовали подряд;
    \item никакие две буквы <<I>> не следовали подряд;
    \item никакие три буквы <<I>> не следовали подряд.
\end{russianEnumerate}
}

\bigskip

\problemItemSimple
{%
Вызначыце колькасць адрозных слоў (не абавязкова тых, якія маюць сэнс), якія можна атрымаць
праз перастаноўку літар у слове <<МАТЭМАТЫКА>> так, каб:
\begin{belarusianEnumerate}
    \item галосныя і зычныя літары чаргаваліся;
    \item ніякія дзве галосныя літары не ішлі адна за адной.
\end{belarusianEnumerate}
}
{%
Определите, сколько различных слов (не обязательно имеющих смысл) можно
получить, переставляя буквы в слове <<МАТЕМАТИКА>>
так, чтобы:
\begin{russianEnumerate}
    \item гласные и согласные буквы чередовались;
    \item никакие две гласные буквы не следовали подряд.
\end{russianEnumerate}
}

\bigskip

\problemItemSimple
{Колькі існуе васьмізначных натуральных лікаў, складзеных з лічбаў 1, 2, 3 і 4,
у якіх лічбы 3 і 4 сустракаюцца роўна два разы кожная і якія дзеляцца на тры?}
{Сколько имеется восьмизначных натуральных чисел, составленных
из цифр 1, 2, 3 и 4, в которых цифры 3 и 4 встречаются ровно
два раза каждая и которые делятся на три?}

\bigskip

\problemItemSimple
{У скрынцы ляжаць тры блакітныя, тры чырвоныя і чатыры зялёныя шары (шары аднаго колеру
не адрозніваюцца). Восем шароў дасталі са скрынкі, па адным за раз, з улікам парадку.
Вызначыце колькасць спосабаў, якімі можна гэта зрабіць.}
{В коробке лежат три синих, три красных и четыре зеленых шара (шары
одного цвета не различаются). Восемь шаров достали из коробки, по одному
за один раз, учитывая порядок. Определите число способов, которыми
это можно сделать.}

\bigskip

\problemItemSimple
{Вызначыце колькасць спосабаў, якімі можна размеркаваць тры чырвоныя, тры белыя і
тры зялёныя шары па васьмі скрынях так, каб кожная скрыня змяшчала хаця б адзін шар.}
{Определите число способов, которыми можно распределить три красных,
три белых и три зеленых шара по восьми различным коробкам так, чтобы
каждая коробка содержала хотя бы один шар.}

\bigskip

\problemItemSimple
{Вызначыце колькасць пяцізначных натуральных лікаў, якія можна скласці
з лічбаў ліку 75\,\,226\,\,522.}
{Определите число пятизначных натуральных чисел, которые можно составить
из цифр числа 75\,\,226\,\,522.}

\bigskip

\problemItemSimple
{Вызначыце колькасць пяцізначных натуральных лікаў, якія можна скласці
з лічбаў ліку 11\,\,223\,\,334 пры ўмове, што тры лічбы 3 не ідуць адна за адной.}
{Определите число пятизначных натуральных чисел, которые можно составить
из цифр числа 11\,\,223\,\,334 при условии, что три цифры 3 не следуют подряд.}

\bigskip

\problemItemSimple
{Дадзеная прасторавая цэлалікавая рашотка з бакамі $p$, $q$ і $r$.
Знайдзіце агульную колькасць адрозных найкарацейшых шляхоў з кропкі $O(0; 0; 0)$ у кропку
$A(p; q; r)$, якія праходзяць па баках рашоткі. Сфармулюйце і абгрунтуйце абагульненне гэтага
факта на большыя памернасці.}
{Дана пространственная целочисленная решетка со сторонами $p$, $q$ и $r$.
Найдите число различных кратчайших путей из точки $O(0; 0; 0)$ в точку
$A(p; q; r)$, проходящих по сторонам решетки. Установите обобщение этого
факта на высшие размерности.}

\bigskip

\problemItemSimple
{Вызначыце колькасць спосабаў, якімі можна пакласці $mn$ адрозных
паштовак у~$m$~аднолькавых канвертаў так, каб кожны канверт змяшчаў
роўна $n$~паштовак.}
{Определите число способов, которыми можно положить $mn$ различных
открыток в~$m$~одинаковых конвертов так, чтобы каждый конверт содержал
ровно $n$~открыток.}

\bigskip

\problemItemSimple
{Вызначыце колькасць спосабаў, якімі можна саставіць $k$ неўпарадкаваных
пар з~$n$~шахматыстаў, дзе $k \le \lfloor n/2\rfloor$.}
{Определите число способов, которыми можно составить $k$ неупорядоченных
пар из~$n$~шахматистов, где $k \le \lfloor n/2\rfloor$.}

\bigskip

\problemItemSimple
{Знайдзіце каэфіцыент пры $x^3y^2z^4t^5$ у раскладанні $(x - 3y + 2z - 4t + 6)^{19}$.}
{Найдите коэффициент при $x^3y^2z^4t^5$ в разложении $(x - 3y + 2z - 4t + 6)^{19}$.}

\end{problemList}

\end{document}
