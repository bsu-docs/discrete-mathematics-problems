\documentclass[12pt,twoside]{article}

\setlength{\textwidth}{166mm}
\setlength{\textheight}{232mm}
\setlength{\topmargin}{-10mm}
\setlength{\headsep}{5mm}
\oddsidemargin=3mm
\evensidemargin=3mm
\setlength{\baselineskip}{18pt}

\usepackage[utf8]{inputenc}
\usepackage[russian]{babel}
\usepackage{amsfonts,amssymb,amsmath}
\usepackage{epsfig}
\usepackage{mathrsfs}
\usepackage{mathabx}
\usepackage{xcolor}

\renewcommand\le{\leqslant}
\renewcommand\ge{\geqslant}

\newcommand{\ruText}[1]{
  {\footnotesize \textcolor{darkgray}{#1} \par}
}

\newcommand{\biLangHeader}[2]{
  \subsection*{%
  	#1 \\%
  	\ruText{#2}%
  }%
}

\newcommand{\quizTitle}[3]{%
\begin{center}
	\textbf{Кантрольная работа па тэме <<#1>> (варыянт #3)} \\
	\ruText{Контрольная работа по теме <<#2>> (вариант #3)}
\end{center}
}

\newcommand{\problemItemSimple}[2]{%
	\item #1 \\%
	\ruText{#2}%
}

\newcommand{\problemItemWithCommonPart}[3]{%
	\item #1 \\%
	\ruText{#2}%
	#3%
}

\newcommand{\problemItemWithCommonPartComplicated}[5]{%
	\item #1 \\%
	\ruText{#2}%
	#3 \\
	\noindent #4 \\%
	\noindent \ruText{#5}%
}

\makeatletter
\def\belarusianLetters#1{
  \expandafter\@belarusianLetters\csname c@#1\endcsname
}
\def\@belarusianLetters#1{
  (%
  \ifcase#1\or а\or б\or в\or г\or д\or е\or ж\or з\or і\or к\or л\or м\fi%
  )
}
\makeatother
\AddEnumerateCounter{\belarusianLetters}{\@belarusianLetters}{Ы}

\newenvironment{problemList}
  {\begin{enumerate}[leftmargin=*,topsep=0pt,itemsep=-1ex,partopsep=1ex,parsep=1ex]}
  {\end{enumerate}}

\newenvironment{belarusianEnumerate}
  {\begin{enumerate}[label=\belarusianLetters*, topsep=-7pt]}
  {\end{enumerate} \textbf{}\vspace{-8pt}}

\AddEnumerateCounter{\asbuk}{\@asbuk}{\cyrm}
\newenvironment{russianEnumerate}
  {\begin{enumerate}[label=(\asbuk*), topsep=-4pt, itemsep=-1ex]}
  {\end{enumerate} \textbf{}\vspace{-11pt}}


% Lines below are to avoid word breaks.
\tolerance=1
\emergencystretch=\maxdimen
\hyphenpenalty=10000
\hbadness=10000

\renewenvironment{itemize}
{\begin{list}
             {\labelitemi}%                     Old parameters:
             {\setlength{\labelwidth}{1.3em}%        1em
              \setlength{\labelsep}{0.7em}%          0.7em
              \setlength{\itemindent}{0em}%          0em
              \setlength{\listparindent}{3em}%       3em
              \setlength{\leftmargin}{2em}%          3em !
              \setlength{\rightmargin}{0em}%         0em
              \setlength{\parsep}{0ex}%              0ex
              \setlength{\topsep}{0.5ex}%            2ex !
              \setlength{\itemsep}{1ex}%             0ex
             }
}
{\end{list}}

\pagestyle{empty}


\begin{document}

\biLangHeader
{12. Разбіцці мностваў і лікаў.}
{Разбиения множеств и чисел.}

\begin{problemList}

\problemItemWithCommonPart
{Упэўніцеся, што:}
{Проверьте, что:}
{%
\begin{belarusianEnumerate}
    \item $S(n, k) = 0$ для $k > n$; $S(n, 0) = 0$ і $S(n, 1) = S(n, n) = 1$ для $n \ge 1$;
    \item $S(n, 2) = 2^{n - 1} - 1$, $n \ge 2$;
    \item $S(n, n - 1) = C_n^2$, $n \ge 2$;
    \item $S(n, n - 2) = C_n^3 + 3 \cdot C_n^4$, $n \ge 3$;
    \item $S(n, n - 3) = C_n^4 + 10 \cdot C_n^5 + 15 \cdot C_n^6$, $n \ge 4$;
    \item $S(n, k) = k \cdot S(n - 1, k) + S(n - 1, k - 1)$, дзе $n \ge k \ge 1$ "--- цэлыя лікі,
    $S(0, 0) = \\ = S(n, n) = 1$.
\end{belarusianEnumerate}
}

\smallskip

\problemItemSimple
{Сярод 12 дзяцей ёсць дзве пары блізнятаў. Колькі ёсць спосабаў разбіць 12 дзяцей
на 4 групы так, каб блізняты не трапілі ў розныя групы?}
{Среди 12 детей есть две пары близнецов. Сколькими способами можно разбить
12 детей на 4 группы так, чтобы близнецы не попали в разные группы?}

\bigskip

\problemItemWithCommonPart
{Дакажыце, што для натуральных лікаў $n$ і $k$, дзе $n \ge k$,
справядлівая роўнасць:}
{Докажите, что для натуральных чисел $n$ и $k$, где $n \ge k$,
имеет место равенство:}
{$S(n + 1, k) = \sum\limits_{i = 0}^n C_n^i S(n - i, k - 1)$.}

\bigskip

\problemItemSimple
{Няхай $B(n) = \sum\limits_{k = 1}^n S(n, k)$ і $B(0) = 1$. Дакажыце,
што для лікаў $B(n)$ мае месца роўнасць $B(n + 1) = \sum\limits_{k = 0}^n B(k)C_n^k$.
Няхай $X$ "--- канечнае $n$-элементнае мноства. Дакажыце, што колькасць усіх
стасункаў эквівалентнасці на мностве $X$ роўная~$B(n)$.}
{Пусть $B(n) = \sum\limits_{k = 1}^n S(n, k)$ и $B(0) = 1$. Докажите,
что для чисел $B(n)$ имеет место равенство $B(n + 1) = \sum\limits_{k = 0}^n B(k)C_n^k$.
Пусть $X$ "--- конечное $n$-элементное множество. Докажите, что число всех
отношений эквивалентности на множестве $X$ равно~$B(n)$.}

\bigskip

\problemItemWithCommonPart
{Упэўніцеся, што:}
{Проверьте, что:}
{%
\begin{belarusianEnumerate}
    \item $p_k(n) = 0$ для $k > n$; $p_1(n) = p_n(n) = 1$ для $n \ge 1$; $p_{n - 1}(n) = 1$ для $n \ge 2$;
    \item $p_2(n) = \bigl\lfloor \frac{n}{2} \bigr\rfloor$ для $n \ge 2$;
    \item $p_3(n) = \frac{(n - 1)(n - 2)}{12} + \frac{1}{2} + \frac{1}{2} \cdot \bigl\lfloor
    \frac{n - 3}{2} \bigr\rfloor + \frac{1}{3} \cdot \bigl\lfloor \frac{n - 3}{3} \bigr\rfloor -
    \frac{1}{3} \cdot \bigl\lfloor \frac{n - 4}{3}\bigr\rfloor$, $n \ge 4$;
    \item $p_k(n) = p_{k - 1}(n - 1) + p_k(n - k)$, дзе $n \ge k \ge 1$ "--- цэлыя лікі,
    $p_k(n) = 0$ для $k > n$, $p_0(n) = 0$ для $n \ge 1$, $p_0(0) = 1$.
\end{belarusianEnumerate}
}

\end{problemList}

\end{document}
