\documentclass[12pt,twoside]{article}

\setlength{\textwidth}{166mm}
\setlength{\textheight}{232mm}
\setlength{\topmargin}{-10mm}
\setlength{\headsep}{5mm}
\oddsidemargin=3mm
\evensidemargin=3mm
\setlength{\baselineskip}{18pt}

\usepackage[utf8]{inputenc}
\usepackage[russian]{babel}
\usepackage{amsfonts,amssymb,amsmath}
\usepackage{epsfig}
\usepackage{mathrsfs}
\usepackage{mathabx}
\usepackage{xcolor}

\renewcommand\le{\leqslant}
\renewcommand\ge{\geqslant}

\newcommand{\ruText}[1]{
  {\footnotesize \textcolor{darkgray}{#1} \par}
}

\newcommand{\biLangHeader}[2]{
  \subsection*{%
  	#1 \\%
  	\ruText{#2}%
  }%
}

\newcommand{\quizTitle}[3]{%
\begin{center}
	\textbf{Кантрольная работа па тэме <<#1>> (варыянт #3)} \\
	\ruText{Контрольная работа по теме <<#2>> (вариант #3)}
\end{center}
}

\newcommand{\problemItemSimple}[2]{%
	\item #1 \\%
	\ruText{#2}%
}

\newcommand{\problemItemWithCommonPart}[3]{%
	\item #1 \\%
	\ruText{#2}%
	#3%
}

\newcommand{\problemItemWithCommonPartComplicated}[5]{%
	\item #1 \\%
	\ruText{#2}%
	#3 \\
	\noindent #4 \\%
	\noindent \ruText{#5}%
}

\makeatletter
\def\belarusianLetters#1{
  \expandafter\@belarusianLetters\csname c@#1\endcsname
}
\def\@belarusianLetters#1{
  (%
  \ifcase#1\or а\or б\or в\or г\or д\or е\or ж\or з\or і\or к\or л\or м\fi%
  )
}
\makeatother
\AddEnumerateCounter{\belarusianLetters}{\@belarusianLetters}{Ы}

\newenvironment{problemList}
  {\begin{enumerate}[leftmargin=*,topsep=0pt,itemsep=-1ex,partopsep=1ex,parsep=1ex]}
  {\end{enumerate}}

\newenvironment{belarusianEnumerate}
  {\begin{enumerate}[label=\belarusianLetters*, topsep=-7pt]}
  {\end{enumerate} \textbf{}\vspace{-8pt}}

\AddEnumerateCounter{\asbuk}{\@asbuk}{\cyrm}
\newenvironment{russianEnumerate}
  {\begin{enumerate}[label=(\asbuk*), topsep=-4pt, itemsep=-1ex]}
  {\end{enumerate} \textbf{}\vspace{-11pt}}


% Lines below are to avoid word breaks.
\tolerance=1
\emergencystretch=\maxdimen
\hyphenpenalty=10000
\hbadness=10000

\renewenvironment{itemize}
{\begin{list}
             {\labelitemi}%                     Old parameters:
             {\setlength{\labelwidth}{1.3em}%        1em
              \setlength{\labelsep}{0.7em}%          0.7em
              \setlength{\itemindent}{0em}%          0em
              \setlength{\listparindent}{3em}%       3em
              \setlength{\leftmargin}{2em}%          3em !
              \setlength{\rightmargin}{0em}%         0em
              \setlength{\parsep}{0ex}%              0ex
              \setlength{\topsep}{0.5ex}%            2ex !
              \setlength{\itemsep}{1ex}%             0ex
             }
}
{\end{list}}

\pagestyle{empty}


\begin{document}

\biLangHeader
{13. Метад уключэнняў і выключэнняў.}
{Метод включения и исключения.}

\begin{problemList}

\problemItemSimple
{У выніку апытання пэўнай колькасці людзей высветлілі, што 45 з іх
ведаюць ангельскую мову, 31 "--- нямецкую мову, 52 "--- французскую мову.
Тры мовы ведаюць 8 чалавек, ангельскую і французскую "--- 28, ангельскую і нямецкую "--- 16,
нямецкую і французскую "--- 20. Колькі людзей было апытана, калі толькі адзін чалавек не ведае
аніводнай мовы? Колькі з апытаных ведаюць толькі нямецкую мову?}
{В результате опроса некоторого количества человек выяснили, что 45
из них знают английский язык, 31 "--- немецкий язык, 52 "--- французский язык.
Три языка знают 8 человек, английский и французский "--- 28, английский и
немецкий "--- 16, немецкий и французский "--- 20. Сколько людей было опрошено,
если только один человек не знает ни одного языка? Сколько из опрошенных
знают только немецкий язык?}

\bigskip

\problemItemSimple
{Колькі існуе цэлых лікаў ад 1 да 1000, якія не дзеляцца ані на 3, ані на 5, ані на 7?}
{Сколько существует целых чисел от 1 до 1000, которые не делятся ни
на 3, ни на 5, ни на 7?}

\bigskip

\problemItemSimple
{Колькі існуе перастановак элементаў мноства $X$, $|X| = n$, у якіх:
а) тры зафіксаваныя элементы $a$, $b$, $c$ з $X$ не ідуць адзін за адным (у любым парадку);
б) ніякія два з зафіксаваных элементаў $a$, $b$, $c$ мноства $X$ не стаяць побач?}
{Сколько существует перестановок из элементов множества $X$, $|X| = n$,
в которых: a) фиксированные три элемента $a$, $b$, $c$ из $X$ не располагаются
подряд (в любом порядке); б) никакие два из фиксированных элементов $a$, $b$, $c$
множества $X$ не стоят рядом?}

\bigskip

\problemItemSimple
{У ліфт увайшлі 7 чалавек. Колькі для іх існуе спосабаў выйсці на чатырох верхніх паверхах дома так,
каб на кожным паверсе выйшаў хаця б адзін чалавек?}
{В лифт вошли 7 человек. Сколькими способами они могут выйти на четырёх
верхних этажах дома так, чтобы на каждом этаже вышел хотя бы один человек?}

\bigskip

\problemItemSimple
{Няхай зададзеныя непустыя мноствы $X$ і $Y$, $|X| = k$, $|Y| = m$.
Зменная $x_i \in X$ функцыі $f\colon X^n \to Y$, якая залежыць ад $n$ зменных,
называецца \emph{фіктыўнай}, калі $$f(x_1, \ldots, x_{i - 1}, x', x_{i + 1}, \ldots, x_n) =
f(x_1, \ldots, x_{i - 1}, x'', x_{i + 1}, \ldots, x_n)$$ для любых $x', x'' \in X$.
Знайдзіце колькасць функцый $f\colon X^n \to Y$, якія не маюць фіктыўных зменных.}
{Пусть заданы непустые множества $X$ и $Y$, $|X| = k$, $|Y| = m$.
Переменная $x_i \in X$ функции $f\colon X^n \to Y$, зависящей от $n$ переменных,
называется \emph{фиктивной}, если $$f(x_1, \ldots, x_{i - 1}, x', x_{i + 1}, \ldots, x_n) =
f(x_1, \ldots, x_{i - 1}, x'', x_{i + 1}, \ldots, x_n)$$ для любых $x', x'' \in X$.
Найдите число функций $f\colon X^n \to Y$, которые не имеют фиктивных переменных.}

\bigskip

\problemItemSimple
{Дадзеныя $n$ набораў, кожны з якіх складаецца з $q$ аднолькавых элементаў, прычым
элементы розных набораў адрозніваюцца. Колькі існуе спосабаў размясціць у шэраг
усе $nq$ элементаў указаных набораў так, каб ніякія $q$ аднолькавых элементаў
не ішлі запар?}
{Даны $n$ наборов, каждый из которых состоит из $q$ одинаковых элементов, причём
элементы различных наборов различны. Сколькими способами можно разместить в ряд
все $nq$ элементов указанных наборов так, чтобы никакие $q$ одинаковых элементов
не располагались подряд?}

\newpage

\problemItemSimple
{%
Колькі лікаў можна ўтварыць перастаноўкай лічбаў ліку $1~234~114~546$ так, каб:
\begin{belarusianEnumerate}
    \item ніякія тры аднолькавыя лічбы не ішлі запар;
    \item ніякія дзве аднолькавыя лічбы не знаходзіліся побач?
\end{belarusianEnumerate}
}
{%
Сколько чисел можно образовать перестановкой цифр числа $1~234~114~546$ так, чтобы
\begin{russianEnumerate}
    \item никакие три одинаковые цифры не располагались подряд;
    \item никакие две одинаковые цифры не располагались рядом?
\end{russianEnumerate}
}

\bigskip

\problemItemSimple
{Паслядоўнасць даўжыні 6, складзеная з лічбаў, называецца \emph{шчаслівай},
калі сума трох першых лічбаў паслядоўнасці роўная суме трох апошніх лічбаў.
Знайдзіце колькасць шчаслівых паслядоўнасцяў.}
{Последовательность длины 6, составленная из цифр, называется \emph{счастливой},
если сумма трёх первых цифр последовательности равна сумме трёх её последних цифр.
Найдите число счастливых последовательностей.}

\bigskip

\problemItemSimple
{Ёсць $2n$ прадметаў першага тыпу, $2n$ прадметаў другога тыпу і $2n$ прадметаў трэцяга тыпу.
Знайдзіце колькасць спосабаў, згодна з якімі можна падзяліць усе $6n$ прадметаў пароўну паміж
двума людзьмі.}
{Имеется $2n$ предметов одного типа, $2n$ предметов второго типа и $2n$ предметов
третьего типа. Найдите число способов, согласно которым можно разделить все $6n$
предметов поровну между двумя людьми.}

\bigskip

\problemItemSimple
{Знайдзіце колькасць $n$-значных натуральных лікаў, састаўленых роўна з $k$
адрозных лічбаў ($1 \le k \le 10$).}
{Найдите число $n$-значных натуральных чисел, составленных в точности из
$k$ различных цифр ($1 \le k \le 10$).}

\end{problemList}

\end{document}
