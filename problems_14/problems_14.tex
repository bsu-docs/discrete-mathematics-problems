\usepackage[utf8]{inputenc}
\usepackage[russian]{babel}
\usepackage{amsfonts,amssymb,amsmath}
\usepackage{float}
\usepackage{epsfig}
\usepackage{mathrsfs}
\usepackage{mathabx}
\usepackage{xcolor}
\usepackage{enumitem}
\usepackage{hyperref}
\usepackage{bbm}
\usepackage{geometry}
\usepackage{ifthen}

\geometry{left=2cm, right=1cm, top=1.5cm, bottom=1.5cm}
\newcommand{\ruText}[1]{
  {\scriptsize \textcolor{darkgray}{#1} \par}
}

\newcommand{\biLangHeader}[2]{
  \subsection*{
    {\normalsize #1} \\
    \indent \ruText{#2}
  }
}


% Lines below are to avoid word breaks.
\tolerance=1
\emergencystretch=\maxdimen
\hyphenpenalty=10000
\hbadness=10000

\pagestyle{empty}

\usepackage{titlesec}
\titleformat{\subsection}[display]{\bfseries\filright}{}{}{}



\begin{document}

\biLangHeader
{14.}
{Рекуррентные соотношения}

\begin{problemList}

\problemItemSimple
{}
{Найдите число подмножеств множества $\{1, 2, \ldots, 11\}$, которые
не содержат:
\begin{belarusianEnumerate}
  \item никаких двух подряд идущих чисел;
  \item никаких трёх подряд идущих чисел.
\end{belarusianEnumerate}}

\problemItemSimple
{}
{Найдите число способов, которыми можно наклеить на почтовый конверт
марки на сумму 40 копеек, используя марки стоимостью в 5, 10, 15 и 20 копеек,
при этом расположив их в одну линию (порядок расположения марок учитывается).}

\problemItemSimple
{}
{\emph{Числа Фибоначчи} $F(n)$, $n \ge 1$, определяются формулами
$F(1) = 1$, $F(2) = 1$, $F(n) = F(n - 1) + F(n - 2)$ при $n \ge 3$. Выразите
через числа Фибоначчи число последовательностей $(x_1, x_2, \ldots, x_n)$,
состоящих из нулей и единиц, для которых
$x_1 \le x_2 \ge x_3 \le x_4 \ge x_5 \le \ldots\,\,$.}

\problemItemSimple
{}
{Флаг состоит из $n \ge 1$ горизонтальных полосок. Каждая полоска
имеет один из трёх цветов (красный, белый, синий). Никакие две соседние полоски
не имеют одинакового цвета, а также крайние полоски окрашены в разные цвета.
Найдите число возможных флагов.}

\problemItemSimple
{}
{Рассматриваются все последовательности $(x_1, x_2, \ldots, x_n)$,
состоящие из чисел $0, 1, 2, 3$. Найдите число последовательностей,
содержащих чётное число нулей.}

\problemItemSimple
{}
{Определите число паролей длины 7, состоящих из букв английского
алфавита $\{a, b, c, \ldots, z\}$, в которых буквы $a$, $b$ и $c$ не
стоят рядом в любом порядке. (Буквы в пароле могут повторяться).}

\problemItemWithCommonPart
{}
{Найдите общее решение линейных однородных рекуррентных соотношений с
постоянными коэффициентами:}
{\begin{belarusianEnumerate}
  \item $f(n + 3) - 9f(n + 2) + 26f(n + 1) - 24f(n) = 0$, $n = 1, 2, \ldots\, $;
  \item $f(n + 3) - 4f(n + 2) - 3f(n + 1) + 18f(n) = 0$, $n = 1, 2, \ldots\, $;
  \item $f(n + 4) - 2f(n + 3) + 5f(n + 2) - 8f(n + 1) + 4f(n) = 0$, $n = 1, 2, \ldots\,\, $.
\end{belarusianEnumerate}}

\end{problemList}

\end{document}
