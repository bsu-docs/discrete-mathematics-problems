\documentclass[12pt,twoside]{article}

\setlength{\textwidth}{166mm}
\setlength{\textheight}{232mm}
\setlength{\topmargin}{-10mm}
\setlength{\headsep}{5mm}
\oddsidemargin=3mm
\evensidemargin=3mm
\setlength{\baselineskip}{18pt}

\usepackage[utf8]{inputenc}
\usepackage[russian]{babel}
\usepackage{amsfonts,amssymb,amsmath}
\usepackage{epsfig}
\usepackage{mathrsfs}
\usepackage{mathabx}
\usepackage{xcolor}

\renewcommand\le{\leqslant}
\renewcommand\ge{\geqslant}

\newcommand{\ruText}[1]{
  {\footnotesize \textcolor{darkgray}{#1} \par}
}

\newcommand{\biLangHeader}[2]{
  \subsection*{%
  	#1 \\%
  	\ruText{#2}%
  }%
}

\newcommand{\quizTitle}[3]{%
\begin{center}
	\textbf{Кантрольная работа па тэме <<#1>> (варыянт #3)} \\
	\ruText{Контрольная работа по теме <<#2>> (вариант #3)}
\end{center}
}

\newcommand{\problemItemSimple}[2]{%
	\item #1 \\%
	\ruText{#2}%
}

\newcommand{\problemItemWithCommonPart}[3]{%
	\item #1 \\%
	\ruText{#2}%
	#3%
}

\newcommand{\problemItemWithCommonPartComplicated}[5]{%
	\item #1 \\%
	\ruText{#2}%
	#3 \\
	\noindent #4 \\%
	\noindent \ruText{#5}%
}

\makeatletter
\def\belarusianLetters#1{
  \expandafter\@belarusianLetters\csname c@#1\endcsname
}
\def\@belarusianLetters#1{
  (%
  \ifcase#1\or а\or б\or в\or г\or д\or е\or ж\or з\or і\or к\or л\or м\fi%
  )
}
\makeatother
\AddEnumerateCounter{\belarusianLetters}{\@belarusianLetters}{Ы}

\newenvironment{problemList}
  {\begin{enumerate}[leftmargin=*,topsep=0pt,itemsep=-1ex,partopsep=1ex,parsep=1ex]}
  {\end{enumerate}}

\newenvironment{belarusianEnumerate}
  {\begin{enumerate}[label=\belarusianLetters*, topsep=-7pt]}
  {\end{enumerate} \textbf{}\vspace{-8pt}}

\AddEnumerateCounter{\asbuk}{\@asbuk}{\cyrm}
\newenvironment{russianEnumerate}
  {\begin{enumerate}[label=(\asbuk*), topsep=-4pt, itemsep=-1ex]}
  {\end{enumerate} \textbf{}\vspace{-11pt}}


% Lines below are to avoid word breaks.
\tolerance=1
\emergencystretch=\maxdimen
\hyphenpenalty=10000
\hbadness=10000

\renewenvironment{itemize}
{\begin{list}
             {\labelitemi}%                     Old parameters:
             {\setlength{\labelwidth}{1.3em}%        1em
              \setlength{\labelsep}{0.7em}%          0.7em
              \setlength{\itemindent}{0em}%          0em
              \setlength{\listparindent}{3em}%       3em
              \setlength{\leftmargin}{2em}%          3em !
              \setlength{\rightmargin}{0em}%         0em
              \setlength{\parsep}{0ex}%              0ex
              \setlength{\topsep}{0.5ex}%            2ex !
              \setlength{\itemsep}{1ex}%             0ex
             }
}
{\end{list}}

\pagestyle{empty}


\begin{document}

\biLangHeader
{14.}
{Рекуррентные соотношения}

\begin{problemList}

\problemItemSimple
{}
{Найдите число подмножеств множества $\{1, 2, \ldots, 11\}$, которые
не содержат:
\begin{belarusianEnumerate}
  \item никаких двух подряд идущих чисел;
  \item никаких трёх подряд идущих чисел.
\end{belarusianEnumerate}}

\problemItemSimple
{}
{Найдите число способов, которыми можно наклеить на почтовый конверт
марки на сумму 40 копеек, используя марки стоимостью в 5, 10, 15 и 20 копеек,
при этом расположив их в одну линию (порядок расположения марок учитывается).}

\problemItemSimple
{}
{\emph{Числа Фибоначчи} $F(n)$, $n \ge 1$, определяются формулами
$F(1) = 1$, $F(2) = 1$, $F(n) = F(n - 1) + F(n - 2)$ при $n \ge 3$. Выразите
через числа Фибоначчи число последовательностей $(x_1, x_2, \ldots, x_n)$,
состоящих из нулей и единиц, для которых
$x_1 \le x_2 \ge x_3 \le x_4 \ge x_5 \le \ldots\,\,$.}

\problemItemSimple
{}
{Флаг состоит из $n \ge 1$ горизонтальных полосок. Каждая полоска
имеет один из трёх цветов (красный, белый, синий). Никакие две соседние полоски
не имеют одинакового цвета, а также крайние полоски окрашены в разные цвета.
Найдите число возможных флагов.}

\problemItemSimple
{}
{Рассматриваются все последовательности $(x_1, x_2, \ldots, x_n)$,
состоящие из чисел $0, 1, 2, 3$. Найдите число последовательностей,
содержащих чётное число нулей.}

\problemItemSimple
{}
{Определите число паролей длины 7, состоящих из букв английского
алфавита $\{a, b, c, \ldots, z\}$, в которых буквы $a$, $b$ и $c$ не
стоят рядом в любом порядке. (Буквы в пароле могут повторяться).}

\problemItemWithCommonPart
{}
{Найдите общее решение линейных однородных рекуррентных соотношений с
постоянными коэффициентами:}
{\begin{belarusianEnumerate}
  \item $f(n + 3) - 9f(n + 2) + 26f(n + 1) - 24f(n) = 0$, $n = 1, 2, \ldots\, $;
  \item $f(n + 3) - 4f(n + 2) - 3f(n + 1) + 18f(n) = 0$, $n = 1, 2, \ldots\, $;
  \item $f(n + 4) - 2f(n + 3) + 5f(n + 2) - 8f(n + 1) + 4f(n) = 0$, $n = 1, 2, \ldots\,\, $.
\end{belarusianEnumerate}}

\end{problemList}

\end{document}
