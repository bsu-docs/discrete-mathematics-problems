\usepackage[utf8]{inputenc}
\usepackage[russian]{babel}
\usepackage{amsfonts,amssymb,amsmath}
\usepackage{float}
\usepackage{epsfig}
\usepackage{mathrsfs}
\usepackage{mathabx}
\usepackage{xcolor}
\usepackage{enumitem}
\usepackage{hyperref}
\usepackage{bbm}
\usepackage{geometry}
\usepackage{ifthen}

\geometry{left=2cm, right=1cm, top=1.5cm, bottom=1.5cm}
\newcommand{\ruText}[1]{
  {\scriptsize \textcolor{darkgray}{#1} \par}
}

\newcommand{\biLangHeader}[2]{
  \subsection*{
    {\normalsize #1} \\
    \indent \ruText{#2}
  }
}


% Lines below are to avoid word breaks.
\tolerance=1
\emergencystretch=\maxdimen
\hyphenpenalty=10000
\hbadness=10000

\pagestyle{empty}

\usepackage{titlesec}
\titleformat{\subsection}[display]{\bfseries\filright}{}{}{}



\begin{document}

\biLangHeader
{15.}
{Производящие функции}

\begin{problemList}

\problemItemWithCommonPart
{}
{Найдите замкнутые формы производящих функций следующих последовательностей:}
{\begin{belarusianEnumerate}
  \item $f(n) = n\alpha^n$, $n = 0, 1, 2, \ldots\, $;
  \item $f(n) = n^2$, $n = 0, 1, 2, \ldots\,\, $.
\end{belarusianEnumerate}}

\problemItemWithCommonPart
{}
{Выразите производящую функцию последовательности $F(n)$ через производящие
функции последовательностей $f(n)$ и $g(n)$, если:}
{\begin{belarusianEnumerate}
  \item $F(n) = \sum_{i = 0}^n f(i)$, $n = 0, 1, 2, \ldots\, $;
  \item $F(n) = \sum_{i = 0}^n f(i)g(n - i)$, $n = 0, 1, 2, \ldots\,\, $.
\end{belarusianEnumerate}}

\problemItemSimple
{}
{Четыре человека поочерёдно бросают игральную кость. Найдите число способов,
которыми может выпасть 14 очков?}

\problemItemSimple
{}
{Найдите число способов выбора одиннадцати объектов из совокупности объектов
пяти типов, если необходимо выбрать не более двух объектов первых трёх типов и
неограниченное количество объектов остальных двух типов.}

\problemItemSimple
{}
{В коробке находятся 2 белых, 3 красных, 8 зелёных и 9 оранжевых шаров.
Найдите число способов которыми можно выбрать 12 шаров, если среди выбранных шаров
есть красный, число выбранных зелёных шаров --- чётное, а число оранжевых шаров ---
нечётное.}

\problemItemSimple
{}
{Найдите число целочисленных решений уравнения $x_1 + x_2 + x_3 = 40$ при
следующих ограничениях на переменные: $4 \le x_1 \le 15$, $9 \le x_2 \le 18$,
$5 \le x_3 \le 16$.}

\problemItemSimple
{}
{Используя экспоненциальную производящую функцию, найдите количество
натуральных чисел, состоящих из $n$ цифр, все цифры в которых нечётные, а цифры 1 и 3
присутствуют в них не менее одного раза.}

\problemItemSimple
{}
{Используя экспоненциальную производящую функцию, найдите число способов
размещения 25 объектов четырёх типов по порядку, если число объектов первого
типа --- чётное, второго типа --- нечётное, и хотя бы по одному объекту третьего и
четвёртого типов.}

\problemItemWithCommonPart
{}
{Используя производящие функции, решите следующие рекуррентные соотношения:}
{\begin{belarusianEnumerate}
  \item $f(0) = 1$ і $f(n) = 2f(n - 1) + 2^n$ для $n = 1, 2, \ldots\, $;
  \item $f(0) = 5$ і $f(n) = nf(n - 1) + 2n$ для $n = 1, 2, \ldots\,\, $.
\end{belarusianEnumerate}}

\end{problemList}

\end{document}
