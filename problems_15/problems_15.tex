\documentclass[12pt,twoside]{article}

\setlength{\textwidth}{166mm}
\setlength{\textheight}{232mm}
\setlength{\topmargin}{-10mm}
\setlength{\headsep}{5mm}
\oddsidemargin=3mm
\evensidemargin=3mm
\setlength{\baselineskip}{18pt}

\usepackage[utf8]{inputenc}
\usepackage[russian]{babel}
\usepackage{amsfonts,amssymb,amsmath}
\usepackage{epsfig}
\usepackage{mathrsfs}
\usepackage{mathabx}
\usepackage{xcolor}

\renewcommand\le{\leqslant}
\renewcommand\ge{\geqslant}

\newcommand{\ruText}[1]{
  {\footnotesize \textcolor{darkgray}{#1} \par}
}

\newcommand{\biLangHeader}[2]{
  \subsection*{%
  	#1 \\%
  	\ruText{#2}%
  }%
}

\newcommand{\quizTitle}[3]{%
\begin{center}
	\textbf{Кантрольная работа па тэме <<#1>> (варыянт #3)} \\
	\ruText{Контрольная работа по теме <<#2>> (вариант #3)}
\end{center}
}

\newcommand{\problemItemSimple}[2]{%
	\item #1 \\%
	\ruText{#2}%
}

\newcommand{\problemItemWithCommonPart}[3]{%
	\item #1 \\%
	\ruText{#2}%
	#3%
}

\newcommand{\problemItemWithCommonPartComplicated}[5]{%
	\item #1 \\%
	\ruText{#2}%
	#3 \\
	\noindent #4 \\%
	\noindent \ruText{#5}%
}

\makeatletter
\def\belarusianLetters#1{
  \expandafter\@belarusianLetters\csname c@#1\endcsname
}
\def\@belarusianLetters#1{
  (%
  \ifcase#1\or а\or б\or в\or г\or д\or е\or ж\or з\or і\or к\or л\or м\fi%
  )
}
\makeatother
\AddEnumerateCounter{\belarusianLetters}{\@belarusianLetters}{Ы}

\newenvironment{problemList}
  {\begin{enumerate}[leftmargin=*,topsep=0pt,itemsep=-1ex,partopsep=1ex,parsep=1ex]}
  {\end{enumerate}}

\newenvironment{belarusianEnumerate}
  {\begin{enumerate}[label=\belarusianLetters*, topsep=-7pt]}
  {\end{enumerate} \textbf{}\vspace{-8pt}}

\AddEnumerateCounter{\asbuk}{\@asbuk}{\cyrm}
\newenvironment{russianEnumerate}
  {\begin{enumerate}[label=(\asbuk*), topsep=-4pt, itemsep=-1ex]}
  {\end{enumerate} \textbf{}\vspace{-11pt}}


% Lines below are to avoid word breaks.
\tolerance=1
\emergencystretch=\maxdimen
\hyphenpenalty=10000
\hbadness=10000

\renewenvironment{itemize}
{\begin{list}
             {\labelitemi}%                     Old parameters:
             {\setlength{\labelwidth}{1.3em}%        1em
              \setlength{\labelsep}{0.7em}%          0.7em
              \setlength{\itemindent}{0em}%          0em
              \setlength{\listparindent}{3em}%       3em
              \setlength{\leftmargin}{2em}%          3em !
              \setlength{\rightmargin}{0em}%         0em
              \setlength{\parsep}{0ex}%              0ex
              \setlength{\topsep}{0.5ex}%            2ex !
              \setlength{\itemsep}{1ex}%             0ex
             }
}
{\end{list}}

\pagestyle{empty}


\begin{document}

\biLangHeader
{15. Утваральныя функцыі.}
{Производящие функции.}

\begin{problemList}

\problemItemWithCommonPart
{Знайдзіце замкнёныя формы ўтваральных функцый наступных паслядоўнасцяў:}
{Найдите замкнутые формы производящих функций следующих последовательностей:}
{%
\begin{belarusianEnumerate}
    \item $f(n) = n\alpha^n$, $n = 0, 1, 2, \ldots\, $;
    \item $f(n) = n^2$, $n = 0, 1, 2, \ldots\,\, $.
\end{belarusianEnumerate}
}

\smallskip

\problemItemWithCommonPart
{Выразіце ўтваральную функцыю паслядоўнасці $F(n)$ праз утваральныя функцыі
паслядоўнасцяў $f(n)$ і $g(n)$, калі:}
{Выразите производящую функцию последовательности $F(n)$ через производящие
функции последовательностей $f(n)$ и $g(n)$, если:}
{%
\begin{belarusianEnumerate}
    \item $F(n) = \sum \limits _{i = 0}^n f(i)$, $n = 0, 1, 2, \ldots\, $;
    \item $F(n) = \sum \limits _{i = 0}^n f(i)g(n - i)$, $n = 0, 1, 2, \ldots\,\, $.
\end{belarusianEnumerate}
}

\medskip

\problemItemSimple
{Чатыры чалавекі па чарзе кідаюць ігральную косць. Знайдзіце колькасць спосабаў,
якімі можа выпасці 14 ачкоў?}
{Четыре человека поочерёдно бросают игральную кость. Найдите число способов,
которыми может выпасть 14 очков?}

\bigskip

\problemItemSimple
{Знайдзіце колькасць спосабаў выбраць адзінаццаць аб'ектаў з сукупнасці аб'ектаў
пяці тыпаў, калі неабходна абраць не больш за два аб'екты першых трох тыпаў і неабмежаваную
колькасць аб'ектаў астатніх трох тыпаў.}
{Найдите число способов выбора одиннадцати объектов из совокупности объектов
пяти типов, если необходимо выбрать не более двух объектов первых трёх типов и
неограниченное количество объектов остальных двух типов.}

\bigskip

\problemItemSimple
{У скрынцы знаходзяцца 2 белыя, 3 чырвоныя, 8 зялёных і 9 аранжавых шароў.
Знайдзіце колькасць спосабаў, якімі можна выбраць 12 шароў, калі сярод абраных шароў
ёсць чырвоны, колькасць абраных зялёных шароў "--- цотная, а колькасць аранжавых шароў "--- няцотная.}
{В коробке находятся 2 белых, 3 красных, 8 зелёных и 9 оранжевых шаров.
Найдите число способов, которыми можно выбрать 12 шаров, если среди выбранных шаров
есть красный, число выбранных зелёных шаров "--- чётное, а число оранжевых шаров "--- нечётное.}

\bigskip

\problemItemWithCommonPart
{Знайдзіце колькасць цэлалікавых рашэнняў раўнання $x_1 + x_2 + x_3 = 40$ пры наступных
абмежаваннях на зменныя:}
{Найдите число целочисленных решений уравнения $x_1 + x_2 + x_3 = 40$ при
следующих ограничениях на переменные:}
{%
    $$
    \begin{cases}
    4 \le x_1 \le 15, \\ 9 \le x_2 \le 18, \\ 5 \le x_3 \le 16.
    \end{cases}
    $$
}

\newpage

\problemItemSimple
{З дапамогай экспаненцыяльнай утваральнай функцыі знайдзіце колькасць натуральных лікаў,
якія складаюцца з $n$ лічбаў, усе лічбы ў якіх няцотныя, а лічбы 1 і 3 прысутнічаюць у іх
не менш за адзін раз.}
{Используя экспоненциальную производящую функцию, найдите количество
натуральных чисел, состоящих из $n$ цифр, все цифры в которых нечётные, а цифры 1 и 3
присутствуют в них не менее одного раза.}

\bigskip

\problemItemSimple
{З дапамогай экспаненцыяльнай утваральнай функцыі знайдзіце колькасць спосабаў упарадкаванага размяшчэння
25 аб'ектаў чатырох тыпаў, калі колькасць аб'ектаў першага тыпу "--- цотная, другога тыпу "--- няцотная,
і прысутнічаюць хаця б па адным аб'екце трэцяга і чацвёртага тыпаў.}
{Используя экспоненциальную производящую функцию, найдите число способов
размещения 25 объектов четырёх типов по порядку, если число объектов первого
типа "--- чётное, второго типа "--- нечётное, и хотя бы по одному объекту третьего и
четвёртого типов.}

\bigskip

\problemItemWithCommonPart
{З дапамогай утваральнай функцыі рашыце наступныя рэкурэнтныя судачыненні:}
{Используя производящие функции, решите следующие рекуррентные соотношения:}
{%
\begin{belarusianEnumerate}
    \item $f(0) = 1$ і $f(n) = 2f(n - 1) + 2^n$ для $n = 1, 2, \ldots\, $;
    \item $f(0) = 5$ і $f(n) = nf(n - 1) + 2n$ для $n = 1, 2, \ldots\,\, $.
\end{belarusianEnumerate}
}

\end{problemList}

\end{document}
