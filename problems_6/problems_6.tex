\documentclass[12pt,twoside]{article}

\setlength{\textwidth}{166mm}
\setlength{\textheight}{232mm}
\setlength{\topmargin}{-10mm}
\setlength{\headsep}{5mm}
\oddsidemargin=3mm
\evensidemargin=3mm
\setlength{\baselineskip}{18pt}

\usepackage[utf8]{inputenc}
\usepackage[russian]{babel}
\usepackage{amsfonts,amssymb,amsmath}
\usepackage{epsfig}
\usepackage{mathrsfs}
\usepackage{mathabx}
\usepackage{xcolor}

\renewcommand\le{\leqslant}
\renewcommand\ge{\geqslant}

\newcommand{\ruText}[1]{
  {\footnotesize \textcolor{darkgray}{#1} \par}
}

\newcommand{\biLangHeader}[2]{
  \subsection*{%
  	#1 \\%
  	\ruText{#2}%
  }%
}

\newcommand{\quizTitle}[3]{%
\begin{center}
	\textbf{Кантрольная работа па тэме <<#1>> (варыянт #3)} \\
	\ruText{Контрольная работа по теме <<#2>> (вариант #3)}
\end{center}
}

\newcommand{\problemItemSimple}[2]{%
	\item #1 \\%
	\ruText{#2}%
}

\newcommand{\problemItemWithCommonPart}[3]{%
	\item #1 \\%
	\ruText{#2}%
	#3%
}

\newcommand{\problemItemWithCommonPartComplicated}[5]{%
	\item #1 \\%
	\ruText{#2}%
	#3 \\
	\noindent #4 \\%
	\noindent \ruText{#5}%
}

\makeatletter
\def\belarusianLetters#1{
  \expandafter\@belarusianLetters\csname c@#1\endcsname
}
\def\@belarusianLetters#1{
  (%
  \ifcase#1\or а\or б\or в\or г\or д\or е\or ж\or з\or і\or к\or л\or м\fi%
  )
}
\makeatother
\AddEnumerateCounter{\belarusianLetters}{\@belarusianLetters}{Ы}

\newenvironment{problemList}
  {\begin{enumerate}[leftmargin=*,topsep=0pt,itemsep=-1ex,partopsep=1ex,parsep=1ex]}
  {\end{enumerate}}

\newenvironment{belarusianEnumerate}
  {\begin{enumerate}[label=\belarusianLetters*, topsep=-7pt]}
  {\end{enumerate} \textbf{}\vspace{-8pt}}

\AddEnumerateCounter{\asbuk}{\@asbuk}{\cyrm}
\newenvironment{russianEnumerate}
  {\begin{enumerate}[label=(\asbuk*), topsep=-4pt, itemsep=-1ex]}
  {\end{enumerate} \textbf{}\vspace{-11pt}}


% Lines below are to avoid word breaks.
\tolerance=1
\emergencystretch=\maxdimen
\hyphenpenalty=10000
\hbadness=10000

\renewenvironment{itemize}
{\begin{list}
             {\labelitemi}%                     Old parameters:
             {\setlength{\labelwidth}{1.3em}%        1em
              \setlength{\labelsep}{0.7em}%          0.7em
              \setlength{\itemindent}{0em}%          0em
              \setlength{\listparindent}{3em}%       3em
              \setlength{\leftmargin}{2em}%          3em !
              \setlength{\rightmargin}{0em}%         0em
              \setlength{\parsep}{0ex}%              0ex
              \setlength{\topsep}{0.5ex}%            2ex !
              \setlength{\itemsep}{1ex}%             0ex
             }
}
{\end{list}}

\pagestyle{empty}


\begin{document}

\biLangHeader{6. Прынцып Дзірыхле.}{Прынцып Дзірыхле.}

\begin{problemList}
	
\problemItemSimple
{Дакажыце, што ў любым мностве з 52 цэлых лікаў знойдуцца прынамсі два лікі, сума ці розніца якіх дзеліцца на 100.}
{Докажите, что в любом множестве из 52 целых чисел найдутся по крайней мере два числа, сумма или разность которых делится на 100.}

\bigskip

\problemItemSimple
{Кропка $(x, y, z) \in \mathbbmss{R}^3$ называецца \emph{цэлай}, калі $x, y, z \in \mathbbmss{Z}$. Дакажыце, што сярод дзевяці цэлых кропак знойдуцца прынамсі дзве кропкі, для якіх сярэдзіна адрэзка з канцамі ў гэтых кропках таксама з'яўляецца цэлай кропкай.}
{Точка $(x, y, z) \in \mathbbmss{R}^3$ называется \emph{целой}, если $x, y, z \in \mathbbmss{Z}$. Докажите, что среди девяти целых точек найдутся по крайней мере две точки, для которых середина отрезка с концами в этих точках также является целой точкой.}

\bigskip

\problemItemSimple
{Дакажыце, што любое падмноства $S \subset \{1, 2, 3, \ldots, 200\}$ магутнасці $|S| = 101$ змяшчае прынамсі два ўзаема простыя лікі $x$ и $y$, то-бок. $\text{НАД}(x, y) = 1$.}
{Докажите, что любое подмножество $S \subset \{1, 2, 3, \ldots, 200\}$ мощности $|S| = 101$ содержит по крайней мере два взаимно простых числа $x$ и $y$, т.~е. $\text{НОД}(x, y) = 1$.}

\bigskip

\problemItemSimple
{Дакажыце, што любое падмноства $S \subset \{1, 2, 3, \ldots, 200\}$ магутнасці $|S| = 101$ змяшчае прынамсі два такія элементы $x$ і $y$, што альбо $x \mid y$, альбо $y \mid x$.}
{Докажите, что любое подмножество $S \subset \{1, 2, 3, \ldots, 200\}$ мощности $|S| = 101$ содержит по крайней мере два таких элемента $x$ и $y$, что либо $x \mid y$, либо $y \mid x$.}

\bigskip

\problemItemSimple
{Няхай $m$ --- адвольны няцотны натуральны лік. Дакажыце, што існуе такі натуральны лік $n$, што $m \mid 2^n - 1$.}
{Пусть $m$ --- произвольное нечетное натуральное число. Докажите, что существует такое натуральное число $n$, что $m \mid 2^n - 1$.}

\bigskip

\problemItemSimple
{Кожны дзень на працягу чатырохтыднёвага адпачынку адпачывальнік гуляў прынамсі адну партыю ў шахматы, Агульны лік згуляных партый не перавышае 40. Дакажыце, што знойдзецца прамежак часу, які складаецца з паслядоўных дзён, на працягу якіх было згуляна роўна 15 партый.}
{Каждый день на протяжении четырехнедельного отпуска отдыхающий играл
по крайней мере одну партию в шахматы. Общее число сыгранных партий не
превышает 40. Докажите, что найдется промежуток времени, состоящий из
последовательных дней, в течение которых было сыграно ровно 15 партий.}
	
\end{problemList}

\end{document}