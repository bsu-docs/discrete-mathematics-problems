\documentclass[12pt,twoside]{article}

\setlength{\textwidth}{166mm}
\setlength{\textheight}{232mm}
\setlength{\topmargin}{-10mm}
\setlength{\headsep}{5mm}
\oddsidemargin=3mm
\evensidemargin=3mm
\setlength{\baselineskip}{18pt}

\usepackage[utf8]{inputenc}
\usepackage[russian]{babel}
\usepackage{amsfonts,amssymb,amsmath}
\usepackage{epsfig}
\usepackage{mathrsfs}
\usepackage{mathabx}
\usepackage{xcolor}

\renewcommand\le{\leqslant}
\renewcommand\ge{\geqslant}

\newcommand{\ruText}[1]{
  {\footnotesize \textcolor{darkgray}{#1} \par}
}

\newcommand{\biLangHeader}[2]{
  \subsection*{%
  	#1 \\%
  	\ruText{#2}%
  }%
}

\newcommand{\quizTitle}[3]{%
\begin{center}
	\textbf{Кантрольная работа па тэме <<#1>> (варыянт #3)} \\
	\ruText{Контрольная работа по теме <<#2>> (вариант #3)}
\end{center}
}

\newcommand{\problemItemSimple}[2]{%
	\item #1 \\%
	\ruText{#2}%
}

\newcommand{\problemItemWithCommonPart}[3]{%
	\item #1 \\%
	\ruText{#2}%
	#3%
}

\newcommand{\problemItemWithCommonPartComplicated}[5]{%
	\item #1 \\%
	\ruText{#2}%
	#3 \\
	\noindent #4 \\%
	\noindent \ruText{#5}%
}

\makeatletter
\def\belarusianLetters#1{
  \expandafter\@belarusianLetters\csname c@#1\endcsname
}
\def\@belarusianLetters#1{
  (%
  \ifcase#1\or а\or б\or в\or г\or д\or е\or ж\or з\or і\or к\or л\or м\fi%
  )
}
\makeatother
\AddEnumerateCounter{\belarusianLetters}{\@belarusianLetters}{Ы}

\newenvironment{problemList}
  {\begin{enumerate}[leftmargin=*,topsep=0pt,itemsep=-1ex,partopsep=1ex,parsep=1ex]}
  {\end{enumerate}}

\newenvironment{belarusianEnumerate}
  {\begin{enumerate}[label=\belarusianLetters*, topsep=-7pt]}
  {\end{enumerate} \textbf{}\vspace{-8pt}}

\AddEnumerateCounter{\asbuk}{\@asbuk}{\cyrm}
\newenvironment{russianEnumerate}
  {\begin{enumerate}[label=(\asbuk*), topsep=-4pt, itemsep=-1ex]}
  {\end{enumerate} \textbf{}\vspace{-11pt}}


% Lines below are to avoid word breaks.
\tolerance=1
\emergencystretch=\maxdimen
\hyphenpenalty=10000
\hbadness=10000

\renewenvironment{itemize}
{\begin{list}
             {\labelitemi}%                     Old parameters:
             {\setlength{\labelwidth}{1.3em}%        1em
              \setlength{\labelsep}{0.7em}%          0.7em
              \setlength{\itemindent}{0em}%          0em
              \setlength{\listparindent}{3em}%       3em
              \setlength{\leftmargin}{2em}%          3em !
              \setlength{\rightmargin}{0em}%         0em
              \setlength{\parsep}{0ex}%              0ex
              \setlength{\topsep}{0.5ex}%            2ex !
              \setlength{\itemsep}{1ex}%             0ex
             }
}
{\end{list}}

\pagestyle{empty}


\begin{document}

\biLangHeader
{7. Бінарныя стасункі. Стасункі эквівалентнасці.}
{Бинарные отношения. Отношения эквивалентности.}

\begin{problemList}

\problemItemWithCommonPart
{Знайдзіце $D_R$, $E_R$, $R^{-1}$, $R \circ R$, $R \circ R^{-1}$, $R^{-1} \circ R$ для наступных бінарных стасункаў: }
{Найдите $D_R$, $E_R$, $R^{-1}$, $R \circ R$, $R \circ R^{-1}$, $R^{-1} \circ R$ для следующих бинарных отношений:}
{%
\begin{belarusianEnumerate}
	
\item $R = \{(x, y) \colon x,y \in \mathbbmss{N}\,\,\text{і}\,\,x|y\}$;
\item $R = \{(x, y) \colon x,y \in \mathbbmss{N}\,\,\text{і}\,\,y|x\}$;
\item $R = \{(x, y) \colon x,y \in \mathbbmss{R}\,\,\text{і}\,\,x + y \le 0\}$;
\item $R = \{(x, y) \colon x,y \in \mathbbmss{R}\,\,\text{і}\,\,2x \ge 3y\}$.
	
\end{belarusianEnumerate}
}

\bigskip

\problemItemWithCommonPart
{Няхай $R$, $R_1$, $R_2$ --- бінарныя стасункі, вызначаныя на пары мностваў $A$ і $B$; $S$, $T$ --- бінарныя стасункі, вызначаныя на пары мностваў  $B$ і $C$. Дакажыце, што:}
{Пусть $R$, $R_1$, $R_2$ --- бинарные отношения, определенные на паре
множеств $A$ и $B$; $S$, $T$ --- бинарные отношения, определенные на
паре множеств $B$ и $C$. Докажите, что:}
{%}
\begin{belarusianEnumerate}
	
\item $(R^{-1})^{-1} = R$;
\item $\overline{R^{-1}} = (\overline{R})^{-1}$;
\item $(R_1 \cup R_2)^{-1} = R_1^{-1} \cup R_2^{-1}$;
\item $(R_1 \cap R_2)^{-1} = R_1^{-1} \cap R_2^{-1}$;
\item $R \circ (S \cup T) = (R \circ S) \cup (R \circ T)$;
\item $(R \circ S)^{-1} = S^{-1} \circ R^{-1}$.
	
\end{belarusianEnumerate}
}

\bigskip

\problemItemSimple
{Высветліце, для якіх бінарных стасункаў $R$, вызначаных на пары мностваў $A$ і $B$ выконваецца судачыненне $R^{-1} = \overline{R}$.}
{Выясните, для каких бинарных отношений $R$, определенных на паре
множеств $A$ и $B$, выполняется соотношение $R^{-1} = \overline{R}$.}

\bigskip

\problemItemSimple
{%
Няхай $R \subseteq A^2$ і $E = \{(a, a) \colon a \in A\}$ --- дыяганаль мноства $A$. Дакажыце, што:
\begin{belarusianEnumerate}
	\item $R$ рэфлексіўнае тады і толькі тады, калі $E \subseteq R$;
	\item $R$ сіметрычнае тады і толькі тады, калі $R^{-1} = R$;
	\item $R$ транзітыўнае тады і толькі тады, калі $R \circ R \subseteq R$;
	\item $R$ антысіметрычнае тады і толькі тады, калі $R \cap R^{-1} \subseteq E$.
\end{belarusianEnumerate}
% <workaround>
\textbf{}
\vspace{-1.5em}
% </workaround>
}
{%
Пусть $R \subseteq A^2$ и $E = \{(a, a) \colon a \in A\}$ --- диагональ множества $A$. Докажите, что:
\begin{russianEnumerate}
	\item $R$ рефлексивно тогда и только тогда, когда $E \subseteq R$;
	\item $R$ симметрично тогда и только тогда, когда $R^{-1} = R$;
	\item $R$ транзитивно тогда и только тогда, когда $R \circ R \subseteq R$;
	\item $R$ антисимметрично тогда и только тогда, когда $R \cap R^{-1} \subseteq E$.
\end{russianEnumerate}
}

\bigskip

\problemItemSimple
{Дакажыце, што сіметрычны і антысіметрычны бінарны стасунак $R$ на мностве $A$ з'яўляецца трынзітыўным на гэтым мностве.}
{Докажите, что симметричное и антисимметричное бинарное отношение $R$ на множестве $A$ является транзитивным на этом множестве.}

\bigskip

\problemItemWithCommonPart
{Усталюйце, ці з'яўляецца кожны з пералічаных ніжэй стасункаў на мностве $A$ стасункам эквівалентнасці. Для кожнага стасунка эквівалентнасці знайдзіце класы эквівалентнасці.}
{Установите, является ли каждое из перечисленных ниже отношений на множестве $A$ отношением эквивалентности. Для каждого отношения эквивалентности найдите классы эквивалентности.}
{%
\begin{belarusianEnumerate}
	
\item $A = \mathbbmss{Z}$ і $R = \{(a, b) \colon a + b = 0\}$;
\item $A = \mathbbmss{Z}$ і $R = \{(a, b) \colon a + b \,\,\text{цотна}\}$;
\item $A = \mathbbmss{Z}$ і $R = \{(a, b) \colon a^2 = b^2\}$;
\item $A = \mathbbmss{Z}$ і $R = \{(a, b) \colon a^3 = b^3\}$;
\item $A = 2^{\{a, b, c, d\}}$ і $R = \{(X, Y) \colon |X| = |Y|\}$;
\item $A = \mathbbmss{Z}$ і $R = \{(a, b) \colon \exists k \in \mathbbmss{Z} \,\, (a - b = 5k)\}$.
	
\end{belarusianEnumerate}
}

\bigskip

\problemItemSimple
{Няхай $A = \{1, 2, 3, 4, 5, 6, 7\}$, $B = \{x, y, z\}$ и $f \colon A \to B$ --- сюр'ектыўная функцыя выгляду $f = \{(1, x), (2, z), (3, x), (4, y), (5, z), (6, y), (7, x)\}$.
Вызначым бінарны стасунак $R$ на мностве $A$ наступныхм чынам: $aRb$ тады і толькі тады, калі $f(a) = f(b)$. Дакажыце, што $R$ ---  стасунак эквівалентнасці, і знайдзіце класы эквівалентнасці.}
{Пусть $A = \{1, 2, 3, 4, 5, 6, 7\}$, $B = \{x, y, z\}$ и $f \colon A \to B$ ---
сюръективная функция вида $f = \{(1, x), (2, z), (3, x), (4, y), (5, z), (6, y), (7, x)\}$.
Определим бинарное отношение $R$ на множестве $A$ следующим образом: $aRb$ тогда и только тогда, когда $f(a) = f(b)$. Докажите, что $R$ --- отношение эквивалентности, и найдите классы эквивалентности.}
	
\end{problemList}

\end{document}