\usepackage[utf8]{inputenc}
\usepackage[russian]{babel}
\usepackage{amsfonts,amssymb,amsmath}
\usepackage{float}
\usepackage{epsfig}
\usepackage{mathrsfs}
\usepackage{mathabx}
\usepackage{xcolor}
\usepackage{enumitem}
\usepackage{hyperref}
\usepackage{bbm}
\usepackage{geometry}
\usepackage{ifthen}

\geometry{left=2cm, right=1cm, top=1.5cm, bottom=1.5cm}
\newcommand{\ruText}[1]{
  {\scriptsize \textcolor{darkgray}{#1} \par}
}

\newcommand{\biLangHeader}[2]{
  \subsection*{
    {\normalsize #1} \\
    \indent \ruText{#2}
  }
}


% Lines below are to avoid word breaks.
\tolerance=1
\emergencystretch=\maxdimen
\hyphenpenalty=10000
\hbadness=10000

\pagestyle{empty}

\usepackage{titlesec}
\titleformat{\subsection}[display]{\bfseries\filright}{}{}{}



\begin{document}

\biLangHeader
{9. Асноўныя правілы камбінаторыкі.}
{Основные правила комбинаторики.}

\begin{problemList}
	
\problemItemSimple
{A}
{Группа состоит из 23 студентов, среди которых 5 отличников. Сколькими способами можно выбрать в группе команду из трёх человек так, чтобы в неё вошли по крайней мере два отличника?}	

\problemItemSimple
{A}
{Найдите число 7-буквенных паролей, состоящих из символов алфавита $\{a, b, \dots, z \}$, которые: (а)~состоят из неповторяющихся букв; (б)~состоят из неповторяющихся букв и буквы $a$ и $b$ не стоят рядом?}

\problemItemSimple
{А}
{Сколько существует $n$-значных натуральных чисел ($n \ge 1$), которые: (а)~делятся на 5; (б)~читаются одинаково слева направо и справа налево?}

\problemItemSimple
{А}
{Сколько существует натуральных чисел, меньших 10000, десятичная запись которых содержит по крайней мере одну цифру 1?}

\problemItemSimple
{А}
{Сколько существует четырёхзначных натуральных чисел, цифры которых не повторяются и принадлежат множеству $\{0, 1, 2, 3, 4, 5\}$? Сколько из них чётных чисел?}

\problemItemSimple
{А}
{Сколько <<слов>>, состоящих из $k$ букв каждое, можно составить из 32 букв, если допускаются повторения, но никакие две соседние буквы не должны совпадать?}

\problemItemSimple
{А}
{Номер карточки состоит из последовательности трёх букв из $\{a, b, \dots, z \}$, которая следует за последовательностью из трёх цифр из $\{0, 1, \dots, 9 \}$. Сколько имеется номеров, если одновременное использование цифры <<0>> и буквы <<о>> запрещено?}

\problemItemSimple
{А}
{Сколько имеется восьмизначных натуральных чисел, составленных из цифр 1, 2, 3, 4, в которых цифры 3 и 4 встречаются ровно два раза каждая и которые делятся на 4?}

\problemItemSimple
{А}
{Сколько существует $n$-значных натуральных чисел ($n \ge 1$), которые делятся на 4 и образованы с помощью цифр 0, 1, 2, 3, 4, 5?} 

\problemItemSimple
{А}
{Пусть заданы непустые множества $X$ и $Y$, $|X|=m$, $|Y|=n$. Сколько существует: (а)~различных отображений $f : X \rightarrow Y$; (б)~различных инъективных отображений $f : X \rightarrow Y$; (в)~различных биективных отображений $f : X \rightarrow Y$?}

\problemItemSimple
{А}
{Пусть $p_{1}, p_{2}, \dots, p_{k}$ --- различные простые числа, $\alpha_{1}, \alpha_{2}, \dots \alpha_{k} \in \mathbbmss{N}$. Сколько различных натуральных делителей имеет число $p_{1} ^{\alpha_{1}} \cdot p_{2} ^{\alpha_{2}} \cdots p_{k} ^{\alpha_{k}}$?}

\problemItemSimple
{А}
{Сколько существует упорядоченных пар $(a,b)$ натуральных чисел $a$ и $b$, для которых $\text{НОК}(a,b) = 2^{3} \cdot 5^{7}\cdot 11^{13}$?}

\problemItemSimple
{А}
{Пусть $p_{1}, p_{2}, \dots, p_{k}$ --- различные простые числа, $\alpha_{1}, \alpha_{2}, \dots \alpha_{k} \in \mathbbmss{N}$. Сколько существует упорядоченных пар $(a,b)$ натуральных чисел $a$ и $b$, для которых $\text{НОК}(a,b) =p_{1} ^{\alpha_{1}} \cdot p_{2} ^{\alpha_{2}} \cdots p_{k} ^{\alpha_{k}}$?}

\problemItemSimple
{А}
{Определите число:~(а) бинарных матриц размерности $m \times n$;~(б) бинарных матриц размерности $m \times n$, в которых строки попарно различны;~(в) бинарных матриц размерности $m \times n$, у которых в каждой строке и каждом столбце содержится чётное число единиц?}

\problemItemSimple
{А}
{Определите число подмножеств $n$-элементного ($n \ge 1$) множества.}

\end{problemList}

\end{document}