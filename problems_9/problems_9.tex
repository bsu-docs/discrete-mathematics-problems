\documentclass[12pt,twoside]{article}

\setlength{\textwidth}{166mm}
\setlength{\textheight}{232mm}
\setlength{\topmargin}{-10mm}
\setlength{\headsep}{5mm}
\oddsidemargin=3mm
\evensidemargin=3mm
\setlength{\baselineskip}{18pt}

\usepackage[utf8]{inputenc}
\usepackage[russian]{babel}
\usepackage{amsfonts,amssymb,amsmath}
\usepackage{epsfig}
\usepackage{mathrsfs}
\usepackage{mathabx}
\usepackage{xcolor}

\renewcommand\le{\leqslant}
\renewcommand\ge{\geqslant}

\newcommand{\ruText}[1]{
  {\footnotesize \textcolor{darkgray}{#1} \par}
}

\newcommand{\biLangHeader}[2]{
  \subsection*{%
  	#1 \\%
  	\ruText{#2}%
  }%
}

\newcommand{\quizTitle}[3]{%
\begin{center}
	\textbf{Кантрольная работа па тэме <<#1>> (варыянт #3)} \\
	\ruText{Контрольная работа по теме <<#2>> (вариант #3)}
\end{center}
}

\newcommand{\problemItemSimple}[2]{%
	\item #1 \\%
	\ruText{#2}%
}

\newcommand{\problemItemWithCommonPart}[3]{%
	\item #1 \\%
	\ruText{#2}%
	#3%
}

\newcommand{\problemItemWithCommonPartComplicated}[5]{%
	\item #1 \\%
	\ruText{#2}%
	#3 \\
	\noindent #4 \\%
	\noindent \ruText{#5}%
}

\makeatletter
\def\belarusianLetters#1{
  \expandafter\@belarusianLetters\csname c@#1\endcsname
}
\def\@belarusianLetters#1{
  (%
  \ifcase#1\or а\or б\or в\or г\or д\or е\or ж\or з\or і\or к\or л\or м\fi%
  )
}
\makeatother
\AddEnumerateCounter{\belarusianLetters}{\@belarusianLetters}{Ы}

\newenvironment{problemList}
  {\begin{enumerate}[leftmargin=*,topsep=0pt,itemsep=-1ex,partopsep=1ex,parsep=1ex]}
  {\end{enumerate}}

\newenvironment{belarusianEnumerate}
  {\begin{enumerate}[label=\belarusianLetters*, topsep=-7pt]}
  {\end{enumerate} \textbf{}\vspace{-8pt}}

\AddEnumerateCounter{\asbuk}{\@asbuk}{\cyrm}
\newenvironment{russianEnumerate}
  {\begin{enumerate}[label=(\asbuk*), topsep=-4pt, itemsep=-1ex]}
  {\end{enumerate} \textbf{}\vspace{-11pt}}


% Lines below are to avoid word breaks.
\tolerance=1
\emergencystretch=\maxdimen
\hyphenpenalty=10000
\hbadness=10000

\renewenvironment{itemize}
{\begin{list}
             {\labelitemi}%                     Old parameters:
             {\setlength{\labelwidth}{1.3em}%        1em
              \setlength{\labelsep}{0.7em}%          0.7em
              \setlength{\itemindent}{0em}%          0em
              \setlength{\listparindent}{3em}%       3em
              \setlength{\leftmargin}{2em}%          3em !
              \setlength{\rightmargin}{0em}%         0em
              \setlength{\parsep}{0ex}%              0ex
              \setlength{\topsep}{0.5ex}%            2ex !
              \setlength{\itemsep}{1ex}%             0ex
             }
}
{\end{list}}

\pagestyle{empty}


\begin{document}

\biLangHeader
{9. Асноўныя правілы камбінаторыкі.}
{Основные правила комбинаторики.}

\begin{problemList}

\problemItemSimple
{Група складаецца з 23 студэнтаў, сярод якіх 5 выдатнікаў. Якой колькасцю спосабаў можна выбраць у групе каманду з трох чалавек так, каб у яе ўвайшлі прынамсі 2 выдатнікі?}
{Группа состоит из 23 студентов, среди которых 5 отличников. Сколькими способами можно выбрать в группе команду из трёх человек так, чтобы в неё вошли по крайней мере два отличника?}

\bigskip

\problemItemSimple
{Знайдзіце колькасць 7-літарных пароляў, якія складаюцца з сімвалаў алфавіта $\{a, b, \dots, z \}$, якія: (а)~складаюцца з адрозных літар; (б)~складаюцца з адрозных літар і літары $a$ і $b$ не стаяць побач?}
{Найдите число 7-буквенных паролей, состоящих из символов алфавита $\{a, b, \dots, z \}$, которые: (а)~состоят из неповторяющихся букв; (б)~состоят из неповторяющихся букв и буквы $a$ и $b$ не стоят рядом?}

\bigskip

\problemItemSimple
{Колькі існуе $n$-значных натуральных лікаў, якія: (а)~дзеляцца на 5; (б)~чытаюцца аднолькава злева направа і справа налева?}
{Сколько существует $n$-значных натуральных чисел, которые: (а)~делятся на 5; (б)~читаются одинаково слева направо и справа налево?}

\bigskip

\problemItemSimple
{Колькі існуе натуральных лікаў, меншых за 10000, дзесятковы запіс якіх змяшчае прынамсі адну лічбу 1?}
{Сколько существует натуральных чисел, меньших 10000, десятичная запись которых содержит по крайней мере одну цифру 1?}

\bigskip

\problemItemSimple
{Колькі існуе чатырохзначных натуральных лікаў, лічбы якіх не паўтараюцца і належаць мноству $\{0, 1, 2, 3, 4, 5\}$? Колькі сярод іх цотных лікаў?}
{Сколько существует четырёхзначных натуральных чисел, цифры которых не повторяются и принадлежат множеству $\{0, 1, 2, 3, 4, 5\}$? Сколько из них чётных чисел?}

\bigskip

\problemItemSimple
{Колькі <<слоў>>, якія складаюцца з $k$ літар кожнае, можна скласці з 32 літар, калі дапускаюцца паўтарэнні, але ніякія дзве суседнія літары не павінны супадаць?}
{Сколько <<слов>>, состоящих из $k$ букв каждое, можно составить из 32 букв, если допускаются повторения, но никакие две соседние буквы не должны совпадать?}

\bigskip

\problemItemSimple
{Нумар карткі складаецца з паслядоўнасці трох літар з  $\{a, b, \dots, z \}$, якая ідзе за паслядоўнасцю з трох лічбаў з $\{0, 1, \dots, 9 \}$. Колькі ёсць нумароў, калі адначасовае ўжыванне лічбы <<0>> і літары <<о>> забаронена?}
{Номер карточки состоит из последовательности трёх букв из $\{a, b, \dots, z \}$, которая следует за последовательностью из трёх цифр из $\{0, 1, \dots, 9 \}$. Сколько имеется номеров, если одновременное использование цифры <<0>> и буквы <<о>> запрещено?}

\bigskip

\problemItemSimple
{Колькі ёсць васьмізначных натуральных лікаў, складзеных з лічбаў 1, 2, 3, 4, у якіх лічбы 3 і 4 сустракаюцца роўна два разы кожная і якія дзеляцца на 4?}
{Сколько имеется восьмизначных натуральных чисел, составленных из цифр 1, 2, 3, 4, в которых цифры 3 и 4 встречаются ровно два раза каждая и которые делятся на 4?}

\bigskip

\problemItemSimple
{Колькі існуе $n$-значных натуральных лікаў, якія дзеляцца на 4 і якія ўтвораныя з дапамогай лічбаў 0, 1, 2, 3, 4, 5?}
{Сколько существует $n$-значных натуральных чисел, которые делятся на 4 и образованы с помощью цифр 0, 1, 2, 3, 4, 5?}

\bigskip

\problemItemSimple
{Няхай зададзеныя непустыя мноствы $X$ і $Y$, $|X|=m$, $|Y|=n$. Колькі існуе: (а)~адлюстраванняў $f : X \rightarrow Y$; (б)~ін'ектыўных адлюстраванняў $f : X \rightarrow Y$; (в)~біектыўных адлюстраванняў $f : X \rightarrow Y$?}
{Пусть заданы непустые множества $X$ и $Y$, $|X|=m$, $|Y|=n$. Сколько существует: (а)~различных отображений $f : X \rightarrow Y$; (б)~различных инъективных отображений $f : X \rightarrow Y$; (в)~различных биективных отображений $f : X \rightarrow Y$?}

\bigskip

\problemItemSimple
{Няхай $p_{1}, p_{2}, \dots, p_{k}$ "--- адрозныя простыя лікі, $\alpha_{1}, \alpha_{2}, \dots \alpha_{k} \in \mathbbmss{N}$. Колькі адрозных натуральных дзельнікаў мае лік $p_{1} ^{\alpha_{1}} \cdot p_{2} ^{\alpha_{2}} \cdots p_{k} ^{\alpha_{k}}$?}
{Пусть $p_{1}, p_{2}, \dots, p_{k}$ "--- различные простые числа, $\alpha_{1}, \alpha_{2}, \dots \alpha_{k} \in \mathbbmss{N}$. Сколько различных натуральных делителей имеет число $p_{1} ^{\alpha_{1}} \cdot p_{2} ^{\alpha_{2}} \cdots p_{k} ^{\alpha_{k}}$?}

\bigskip

\problemItemSimple
{Колькі існуе ўпарадкаваных пар $(a,b)$ натуральных лікаў $a$ і $b$, для якіх $\text{НАК}(a,b) = 2^{3} \cdot 5^{7}\cdot 11^{13}$?}
{Сколько существует упорядоченных пар $(a,b)$ натуральных чисел $a$ и $b$, для которых $\text{НОК}(a,b) = 2^{3} \cdot 5^{7}\cdot 11^{13}$?}

\bigskip

\problemItemSimple
{Няхай $p_{1}, p_{2}, \dots, p_{k}$ "--- адрозныя простыя лікі, $\alpha_{1}, \alpha_{2}, \dots \alpha_{k} \in \mathbbmss{N}$. Колькі існуе ўпадаркаваных пар $(a,b)$  натуральных лікаў $a$ і $b$, для якіх $\text{НАК}(a,b) = p_{1} ^{\alpha_{1}} \cdot p_{2} ^{\alpha_{2}} \cdots p_{k} ^{\alpha_{k}}$?}
{Пусть $p_{1}, p_{2}, \dots, p_{k}$ "--- различные простые числа, $\alpha_{1}, \alpha_{2}, \dots \alpha_{k} \in \mathbbmss{N}$. Сколько существует упорядоченных пар $(a,b)$ натуральных чисел $a$ и $b$, для которых $\text{НОК}(a,b) = p_{1} ^{\alpha_{1}} \cdot p_{2} ^{\alpha_{2}} \cdots p_{k} ^{\alpha_{k}}$?}

\bigskip

\problemItemSimple
{Вызначыце колькасць: (а)~бінарных матрыц памеру $m \times n$; (б)~бінарных матрыц памеру $m \times n$, у якіх радкі папарна адрозныя; (в)~бінарных матрыц памеру $m \times n$, у якіх у кожным радку і кожным слупку знаходзіцца цотная колькасць адзінак?}
{Определите число: (а)~бинарных матриц размерности $m \times n$; (б)~бинарных матриц размерности $m \times n$, в которых строки попарно различны; (в)~бинарных матриц размерности $m \times n$, у которых в каждой строке и каждом столбце содержится чётное число единиц?}

\bigskip

\problemItemSimple
{Вызначыце колькасць падмностваў $n$-элементнага ($n \ge 1$) мноства.}
{Определите число подмножеств $n$-элементного ($n \ge 1$) множества.}

\end{problemList}

\end{document}
