\documentclass[12pt, a4paper]{article}

\usepackage{import}
\subimport{../common/}{preamble}

\begin{document}

\quizTitle
{Логіка~выказванняў.~Мноствы.~Адлюстраванні.}
{Логика высказываний.~Множества.~Отображения.}
{1}

\begin{problemList}

\problemItemSimple
{Што можна сказаць наконт праўдзівасці выказвання $\overline{A} \sim B$, калі вядома, што~$A~\sim~B = \mbox{П}$?}
{Что можно сказать об истинностном значении высказывания $\overline{A} \sim B$, если известно, что~$A \sim B = \mbox{И}$?}

\bigskip

\problemItemSimple
{Дакажыце раўназначнасць логікі выказванняў $(A \to B) \to B \equiv A \vee B$, карыстаючыся раўназначнымі пераўтварэннямі.}
{Докажите равносильность логики высказываний $(A \to B) \to B \equiv A \vee B$, используя равносильные преобразования.}

\bigskip

\problemItemSimple
{Высветліце, ці з'яўляецца формула логікі выказванняў $\overline{A \vee B} \cdot A$ таўталогіяй, супярэчнасцю, здзяйсняльнай формулай.}
{Выясните, является ли формула логики высказываний $\overline{A \vee B} \cdot A$ тавтологией, противоречием, выполнимой формулой.}

\bigskip

\problemItemSimple
{Дакажыце, што лагічны вынік $(A \to B), (C \to D), (A \vee C) \models (B \vee D)$ справядлівы.}
{Докажите, что логическое следствие $(A \to B), (C \to D), (A \vee C) \models (B \vee D)$ верно.}

\bigskip

\problemItemSimple
{Калі Барыс не прыйшоў на сход, то адсутнічае і Аляксей. Калі Барыс прыйшоў на сход, то прысутнічаюць Аляксей і Валерый.
Ці абавязкова прысутнічае на сходзе Аляксей, калі Валерыя там няма?}
{Если Борис не пришёл на собрание, то отсутствует и Алексей. Если Борис пришёл на собрание, то присутствуют Алексей и Валерий.
Обязательно ли присутствует на собрании Алексей, если Валерия там нет?}

\bigskip

\problemItemSimple
{Дакажыце тэарэтыка-мноственную тоеснасць $(A \cup B) \cap (A \cup \overline{B}) = A$, карыстаючыся раўназначнымі пераўтварэннямі.}
{Докажите теоретико-множественное тождество $(A \cup B) \cap (A \cup \overline{B}) = A$, используя равносильные преобразования.}

\bigskip

\problemItemSimple
{Даследуйце адлюстраванне $f: \mathbbmss{R} \mapsto \mathbbmss{R}$, $f(x) = |x|$ на ін'ектыўнасць, сюр'ектыўнасць і біектыўнасць. }
{Исследуйте отображение $f: \mathbbmss{R} \mapsto \mathbbmss{R}$, $f(x) = |x|$ на инъективность, сюръективность и биективность.}

\bigskip

\problemItemSimple
{Прывядзіце прыклад двух біектыўных адлюстраванняў з $\mathbbmss{R}$ у $\mathbbmss{R}$,
сума і здабытак якіх не з'яўляюцца біектыўнымі адлюстраваннямі з $\mathbbmss{R}$ у $\mathbbmss{R}$.}
{Приведите пример двух биективных отображений из $\mathbbmss{R}$ в $\mathbbmss{R}$,
сумма и произведение которых не являются биективными отображениями из $\mathbbmss{R}$ в $\mathbbmss{R}$.}

\end{problemList}

\newpage

\quizTitle
{Логіка~выказванняў.~Мноствы.~Адлюстраванні.}
{Логика высказываний.~Множества.~Отображения.}
{2}

\begin{problemList}

\problemItemSimple
{Што можна сказаць наконт праўдзівасці выказвання$A \sim \overline{B}$, калі вядома, што~$A~\sim~B = \mbox{П}$?}
{Что можно сказать об истинностном значении высказывания $A \sim \overline{B}$, если известно, что~$A \sim B = \mbox{И}$?}

\bigskip

\problemItemSimple
{Дакажыце раўназначнасць логікі выказванняў $\overline{B} \to (A \to B) \equiv \overline{A} \vee B$,
карыстаючыся раўназначнымі пераўтварэннямі.}
{Докажите равносильность логики высказываний $\overline{B} \to (A \to B) \equiv \overline{A} \vee B$,
используя равносильные преобразования.}

\bigskip

\problemItemSimple
{Высветліце, ці з'яўляецца формула логікі выказванняў $(\overline{A} \to B) \cdot \overline{A} \cdot \overline{B}$
таўталогіяй, супярэчнасцю, здзяйсняльнай формулай.}
{Выясните, является ли формула логики высказываний $(\overline{A} \to B) \cdot \overline{A} \cdot \overline{B}$
тавтологией, противоречием, выполнимой формулой.}

\bigskip

\problemItemSimple
{Дакажыце, што лагічны вынік $(A \to B), (C \to D), \overline{B}, \overline{D} \models (\overline{A} \cdot \overline{C})$ справядлівы.}
{Докажите, что логическое следствие $(A \to B), (C \to D), \overline{B}, \overline{D} \models (\overline{A} \cdot \overline{C})$ верно.}

\bigskip

\problemItemSimple
{Калі Барыс не прыйшоў на сход, то адсутнічае і Аляксей. Калі Барыс прыйшоў на сход, то прысутнічаюць Аляксей і Валерый.
Ці абавязкова прысутнічае на сходзе Валерый, калі Аляксей прыйшоў на сход?}
{Если Борис не пришёл на собрание, то отсутствует и Алексей. Если Борис пришёл на собрание, то присутствуют Алексей и Валерий.
Обязательно ли присутствует на собрании Валерий, если Алексей находится на собрании?}

\bigskip

\problemItemSimple
{Дакажыце тэарэтыка-мноственную тоеснасць $(A \cap B) \cup (A \cap \overline{B}) = A$, карыстаючыся раўназначнымі пераўтварэннямі.}
{Докажите теоретико-множественное тождество $(A \cap B) \cup (A \cap \overline{B}) = A$, используя равносильные преобразования.}

\bigskip

\problemItemSimple
{Даследуйце адлюстраванне $f: \mathbb{R} \setminus \{1\} \mapsto \mathbb{R}$, $f(x) = \frac{x+1}{x-1}$
на ін'ектыўнасць, сюр'ектыўнасць і біектыўнасць. }
{Исследуйте отображение $f: \mathbb{R} \setminus \{1\} \mapsto \mathbb{R}$, $f(x) = \frac{x+1}{x-1}$
на инъективность, сюръективность и биективность.}

\bigskip

\problemItemSimple
{Прывядзіце прыклад двух небіектыўных адлюстраванняў, кампазіцыя якіх з'яўляецца біектыўным адлюстраваннем.}
{Приведите пример двух небиективных отображений, композиция которых является биективным отображением.}

\end{problemList}

\end{document}
