\documentclass[11pt, a4paper]{article}
\usepackage{import}

\subimport{../common/}{preamble}
\geometry{left=1cm, top=1cm, bottom=1cm}

\begin{document}
    
\quizTitle{Логіка прэдыкатаў. Камбінаторыка.}
{Логика предикатов. Комбинаторика.}
{1}

\begin{problemList}

\problemItemSimple
{Знайдзіце вобласть праўдзівасці прэдыката $P(x, y) = (x \cdot y = 8)$, вызначанага на мностве натуральных лікаў.}
{Найдите область истинности предиката $P(x, y) = (x \cdot y = 8)$, определенного на множестве натуральных чисел.}

\bigskip

\problemItemSimple
{Надайце формуле логіцы прэдыкатаў $\forall x \forall y P(x, y, z)$ наступную інтэрпрэтацыю: $P(x, y, z) = (x + y = z)$, $z = 3$ і $x, y, z \in \mathbbmss{N}$.
Высветліце праўдзівае значэнне выказвання, якое атрымліваецца ў выніку такой інтэрпрэтацыі.}
{Придайте формуле логики предикатов $\forall x \forall y P(x, y, z)$ следующую интерпретацию: $P(x, y, z) = (x + y = z)$, $z = 3$ и $x, y, z \in \mathbbmss{N}$.
Выясните истинностное значение высказывания, которое получается в результате такой интерпретации.}

\bigskip

\problemItemSimple
{Дакажыце, што наступная формула логікі прэдыкатаў $\forall x\;P(x) \sim \overline{\exists x\;\overline{P(x)}}$ з'яўляецца таўталогіяй.}
{Докажите, что следующая формула логики предикатов $\forall x\;P(x) \sim \overline{\exists x\;\overline{P(x)}}$ является тавтологией.}

\bigskip

\problemItemSimple
{У скрынцы ляжаць тры розныя чырвоныя шары, чатыры аднолькавыя сінія шары і пяць розных зялёных шароў.
Колькімі спосабамі можна выцягнуць адзін шар? Колькі існуе спосабаў выцягнуць адзін чырвоны шар,
а пасля за ім выцягнуць адзін сіні шар?}
{В ящике лежат три разных красных шара, четыре одинаковых синих шара и пять разных зеленых шаров.
Сколькими способами можно извлечь один шар?
Сколькими способами можно извлечь один красный шар, а вслед за ним извлечь один синий шар?}

\bigskip

\problemItemSimple
{Шааліньскі манах Лі Нь усталяваў сабе на айфон telegram і актыўна ім карыстаецца.
Нядаўна ён даведаўся, што можна замацаваць любыя 5 чатаў уверсе, таму вырашыў выбраць гутаркі, якія не трэба будзе доўга шукаць.
Лі падпісаны на 10 каналаў, знаходзіцца ў 40 групавых чатах і вядзе 70 асабістых перапісак.
Ён жадае замацаваць 1 канал, 2 групы і 2 асабістыя чаты, прычым у вызначаным парадку.
Колькімі спосабамі ён можа гэта зрабіць?}
{Шаолиньский монах Ли Нь установил себе на айфон telegram и активно им пользуется.
Недавно он узнал, что можно закрепить любые 5 чатов вверху, поэтому решил выбрать чаты, которые не нужно будет долго искать.
Ли подписан на 10 каналов, находится в 40 групповых чатах и ведёт 70 личных переписок.
Он хочет закрепить 1 канал, 2 группы и 2 личных чата, причём в определённом порядке.
Сколькими способами он может это сделать?}

\bigskip

\problemItemSimple
{Студэнту Рому прыходзіцца жыць на стыпендыю, і для гэтага ён есць шмат таннай ежы хуткага прыгатавання.
Кожны месяц Рома набывае 30 пачкаў лапшы <<Ролтан>> з трыма смакамі:
<<з грыбамі>>, <<з курыцай>> і <<з ялавічынай>> "--- а таксама 20 пачкаў пюрэ <<з курыцай>> і <<з мясным смакам>>.
Колькі адрозных магчымасцяў закупіць 50 адзінак тавару ёсць у Ромы, калі ў краме можна набыць любы набор?}
{Студенту Роме приходится жить на стипендию, и для этого он ест много дешёвой еды быстрого приготовления.
Каждый месяц Рома покупает 30 пачек лапши <<Роллтон>> с тремя вкусами:
<<с грибами>>, <<с курицей>> и <<с говядиной>> "--- а также 20 пачек пюре <<с курицей>> и <<с мясным вкусом>>.
Сколько различных возможностей закупить 50 единиц товара есть у Ромы, если в магазине можно приобрести любой набор?}

\bigskip

\problemItemSimple
{Школьная каманда па спартыўным праграмаванні, якая складаецца з 9 чалавек з папарна адрознымі імёнамі, сабралася паабедаць у сталовай.
Яны вырашылі сесці за вялікі круглы стол. Вядома, што Паша не лічыць Сяргея годным быць у камандзе, 
таму побач з ім не сядзе, а Машы і Тані падабаецца адзін і той жа хлопец, і яны таксама не хочуць сядзець побач.
Колькі існуе адрозных спосабаў рассадзіць школьнікаў, каб усе былі задаволеныя
(два размяшчэнні за круглым сталом лічацца аднолькавымі,
калі ў кожнага чалавека сусед злева адзін і той жа).}
{Школьная команда по спортивному программированию, состоящая из 9 человек с попарно различными именами, собралась пообедать в столовой.
Они решили сесть за большой кргулый стол. Известно, что Паша считает Сергея недостойным команды, поэтому рядом с ним не сядет,
а Маше и Тане нравится один и тот же один парень, и они тоже не хотят сидеть рядом.
Сколько существует различных способов рассадить школьников, чтобы все были довольны
(два размещения за круглым столом считаются одинаковыми, если у каждого человека сосед слева один и тот же)?}

\bigskip

\problemItemSimple
{Паталагаанатам Віктар дужа любіць сваю працу, таму ў <<Counter-Strike: Global Offensive>> ён гуляе толькі з ботамі.
У выходны дзень Віктар вырашыў зладзіць марафон і згуляць на 8 розных мапах запар.
Выбіраць ён плануе з 18 мап са старонкі сайта liquipedia.net, прычым, калі ў гэты дзень ён будзе гуляць і на мапе Dust 2, і на мапе Inferno,
то ў спісе выбраных мап Inferno павінна ісці раней, чым Dust 2.
Колькі адрозных спосабаў выбраць 8 мап і згуляць на іх у вызначаным парадку ёсць у Віктара?}
{Патологоанатом Виктор очень любит свою работу, поэтому в <<Counter-Strike: Global Offensive>> он играет только с ботами.
В выходной день Виктор решил устроить марафон и сыграть на 8 различных картах подряд.
Выбирать он планирует из 18 карт со страницы сайта liquipedia.net, причём, если в этот день он будет играть и на карте Dust 2, и на карте Inferno,
то в списке выбранных карт Inferno должна идти раньше, чем Dust 2.
Сколько различных способов выбрать 8 карт и сыграть на них в определённом порядке есть у Виктора?}

\end{problemList}

\newpage

\quizTitle{Логіка прэдыкатаў. Камбінаторыка.}
{Логика предикатов. Комбинаторика.}
{2}
 
\begin{problemList}

\problemItemSimple
{Знайдзіце вобласть праўдзівасці прэдыката $P(x, y) = (x - y = 8)$, вызначанага на мностве натуральных лікаў.}
{Найдите область истинности предиката $P(x, y) = (x - y = 8)$, определенного на множестве натуральных чисел.}

\medskip

\problemItemSimple
{Надайце формуле логіцы прэдыкатаў $\forall x \forall y P(x, y, z)$ наступную інтэрпрэтацыю: $P(x, y, z) = (x - y = z)$, $z = 9$ і $x, y, z \in \mathbbmss{R}$.
Высветліце праўдзівае значэнне выказвання, якое атрымліваецца ў выніку такой інтэрпрэтацыі.}
{Придайте формуле логики предикатов $\forall x \forall y P(x, y, z)$ следующую интерпретацию: $P(x, y, z) = (x - y = z)$, $z = 9$ и $x, y, z \in \mathbbmss{R}$.
Выясните истинностное значение высказывания, которое получается в результате такой интерпретации.}

\medskip

\problemItemSimple
{Дакажыце, што наступная формула логікі прэдыкатаў $\overline{\forall x\;P(x)} \sim \exists x\;\overline{P(x)}$ з'яўляецца таўталогіяй.}
{Докажите, что следующая формула логики предикатов $\overline{\forall x\;P(x)} \sim \exists x\;\overline{P(x)}$ является тавтологией.}

\medskip

\problemItemSimple
{У скрынцы ляжаць тры розныя чырвоныя шары, чатыры розныя сінія шары і пяць розных зялёных шароў.
Колькімі спосабамі можна выцягнуць адзін шар? Колькі існуе спосабаў выцягнуць адзін чырвоны шар,
а пасля за ім выцягнуць адзін сіні шар?}
{В ящике лежат три разных красных шара, четыре разных синих шара и пять разных зеленых шаров.
Сколькими способами можно извлечь один шар?
Сколькими способами можно извлечь один красный шар, а вслед за ним извлечь один синий шар?}

\medskip

\problemItemSimple
{Халасцяка Дзяніса паклікалі на вечарыну, на якую, акрамя яго, прыйшло 19 незнаёмых яму людзей: 11 хлопцаў і 8 дзяўчын.
Дзяніс хоча пазнаёміцца і паразмаўляць з усімі дзяўчынамі, але, як чалавек з добрым выхаваннем, ён павінен пазнаёміцца з усімі.
Тады Дзяніс вырашыў, што першае і апошняе знаёмствы на гэтай тусоўцы будуць з дзяўчынай, а парадак астатніх знаёмстваў яму не важны.
Колькімі спосабамі Дзяніс можа выбраць парадак, у якім ён будзе знаёміцца з людзьмі на вечарыне?}
{Холостяка Дениса позвали на вечеринку, на которую, кроме него, пришло 19 незнакомых ему людей: 11 парней и 8 девушек.
Денис хочет познакомиться и пообщаться со всеми девушками, но, как воспитанный человек, он должен познакомиться со всеми.
Тогда Денис решил, что первое и последнее знакомства на этой тусовке будут с девушкой, а порядок остальных знакомств ему не важен.
Сколькими способами Денис может выбрать порядок, в котором он будет знакомиться с людьми на вечеринке?}

\medskip

\problemItemSimple
{У татуіроўшчыцы Любы шмат вольнага часу, і яна гуляе ў шахматы анлайн.
На сайце lichess.org у яе 70 сяброў, сярод якіх 42 маюць рэйтынг, вышэйшы за рэйтынг Любы, а ў астатніх 28 рэйтынг ніжэй.
У нейкі дзень Люба вырашыла арганізаваць турнір з 24 чалавек, у якім прымуць удзел толькі яна са сваімі сябрамі.
Турнір будзе праводзіцца па швейцарскай сістэме, па якой у першым матчы яна будзе гуляць з выпадковым праціўнікам.
Люба гуляе лепш, калі турнір пачынаецца з перамогі, таму яна хоча ўзяць не больш за 11 сяброў з вышэйшым за яе рэйтынгам.
Колькімі адрознымі спосабамі Люба можа выбраць удзельнікаў турніру?}
{У татуировщицы Любы много свободного времени, и она играет в шахматы онлайн.
На сайте lichess.org у неё 70 друзей, среди которых 42 имеют рейтинг выше, чем рейтинг Любы, а у остальных 28 рейтинг ниже.
В какой-то день Люба решила организовать турнир из 24 человек, в котором примут участие только она со своими друзьями.
Турнир будет проводиться по швейцарской системе, по которой в первом матче она будет играть со случайным противником.
Люба играет лучше, если турнир начинается с победы, поэтому она хочет взять не более 11 друзей с большим относительно Любы рейтингом.
Сколькими различными способами Люба может выбрать участников турнира?}

\medskip

\problemItemSimple
{Хіпстар Ягор разумее, у які час жыве, таму не марнуе ніводнай секунды.
Таму ён, седзячы ў туалеце, павышае свой далягляд, чытаючы цікавыя артыкулы, разгадваючы сканворды і г.д.
На гэтым тыдні Ягор вырашыў разгадваць японскія красворды, кожны дзень па часопісе.
У шапіку ён набыў 16 розных часопісаў, сярод якіх ёсць яго любімыя <<То яма, то канава>> і <<Сікоку-сікоку>>.
Колькі адрозных спосабаў ёсць у Ягора, каб выбраць 7 часопісаў і скласці іх у стос, пры гэтым, калі ў стосе адначасова аказаліся ягоныя любімыя часопісы, то <<То яма, то канава>> павінна знаходзіцца вышэй, чым <<Сікоку-сікоку>>?}
{Хипстер Егор понимает, в какое время живёт, поэтому не тратит ни одной секунды зря.
Поэтому он, сидя в туалете, повышает свой кругозор, читая интересные статьи, разгадывая сканворды и т.д.
На этой неделе Егор решил разгадывать японские кроссворды, каждый день по журналу.
В киоске он купил 16 разных журналов, среди которых ёсць его любимые <<То яма, то канава>> и <<Сикоку-сикоку>>.
Сколько различных способов есть у Егора, чтобы выбрать 7 журналов и сложить их в стопку, при этом,
если в стопке одновременно оказались его любимые журналы, то <<То яма, то канава>> должна находиться выше, чем <<Сикоку-сикоку>>?}

\medskip

\problemItemSimple
{Баскетбольны youtube-блогер Mike Korzemba прыдумаў наступную ідэю відэа, якое будзе называцца <<$23 > 33$?>>.
Ідэя ў тым, каб праверыць, якая каманда пераможа ў сімуляцыі NBA 2k19: каманда лепшых гульцоў, якія калісьці насілі нумар 23, або каманда з гульцоў пад 33 нумарам.
За каманду <<23>> будуць гуляць Дрэйманд Грын, ЛеБрон Джэймс і Майкл Джордан,
а за каманду <<33>> "--- Лары Бёрд, Карым Абдул-Джабар, Скоці Піпен і Алонза Моўрнінг.
Каманды павінны складацца з 10 чалавек, таму гульцы будуць <<кланіраваныя>>, каб у выніку іх хапіла для каманды.
Колькі існуе адрозных спосабаў правесці матчы, дзе 2 спосабы лічацца аднолькавымі,
калі абедзве каманды ў гэтых спосабаў супадаюць? Пазіцыі гульцоў не ўлічваюцца.}
{Баскетбольный youtube-блогер Mike Korzemba придумал следующую идею для видео, которое будет называться <<$23 > 33$?>>.
Идея состоит в том, чтобы проверить, какая команда победит в симуляции NBA 2k19: команда лучших игроков, когда-либо носивших номер 23, или команда игравших под 33 номером.
За команду <<23>> будут играть Дрэймонд Грин, ЛеБрон Джеймс и Майкл Джордан,
а за команду <<33>> "--- Ларри Бёрд, Карим Абдул-Джаббар, Скотти Пиппен и Алозо Моурнинг.
Команды должны состоять из 10 человек, поэтому игроки будут <<клонированы>>, чтобы в результате их хватило для команды.
Сколькими различными способами можно провести матчи, где 2 способа считаются одинаковыми,
если обе команды у этих способов совпадают? Позиции игроков не учитываются.}

\end{problemList}

\end{document}