\documentclass[12pt,twoside]{article}

\setlength{\textwidth}{166mm}
\setlength{\textheight}{232mm}
\setlength{\topmargin}{-10mm}
\setlength{\headsep}{5mm}
\oddsidemargin=3mm
\evensidemargin=3mm
\setlength{\baselineskip}{18pt}

\usepackage[utf8]{inputenc}
\usepackage[russian]{babel}
\usepackage{amsfonts,amssymb,amsmath}
\usepackage{epsfig}
\usepackage{mathrsfs}
\usepackage{mathabx}
\usepackage{xcolor}

\renewcommand\le{\leqslant}
\renewcommand\ge{\geqslant}

\newcommand{\ruText}[1]{
  {\footnotesize \textcolor{darkgray}{#1} \par}
}

\newcommand{\biLangHeader}[2]{
  \subsection*{%
  	#1 \\%
  	\ruText{#2}%
  }%
}

\newcommand{\quizTitle}[3]{%
\begin{center}
	\textbf{Кантрольная работа па тэме <<#1>> (варыянт #3)} \\
	\ruText{Контрольная работа по теме <<#2>> (вариант #3)}
\end{center}
}

\newcommand{\problemItemSimple}[2]{%
	\item #1 \\%
	\ruText{#2}%
}

\newcommand{\problemItemWithCommonPart}[3]{%
	\item #1 \\%
	\ruText{#2}%
	#3%
}

\newcommand{\problemItemWithCommonPartComplicated}[5]{%
	\item #1 \\%
	\ruText{#2}%
	#3 \\
	\noindent #4 \\%
	\noindent \ruText{#5}%
}

\makeatletter
\def\belarusianLetters#1{
  \expandafter\@belarusianLetters\csname c@#1\endcsname
}
\def\@belarusianLetters#1{
  (%
  \ifcase#1\or а\or б\or в\or г\or д\or е\or ж\or з\or і\or к\or л\or м\fi%
  )
}
\makeatother
\AddEnumerateCounter{\belarusianLetters}{\@belarusianLetters}{Ы}

\newenvironment{problemList}
  {\begin{enumerate}[leftmargin=*,topsep=0pt,itemsep=-1ex,partopsep=1ex,parsep=1ex]}
  {\end{enumerate}}

\newenvironment{belarusianEnumerate}
  {\begin{enumerate}[label=\belarusianLetters*, topsep=-7pt]}
  {\end{enumerate} \textbf{}\vspace{-8pt}}

\AddEnumerateCounter{\asbuk}{\@asbuk}{\cyrm}
\newenvironment{russianEnumerate}
  {\begin{enumerate}[label=(\asbuk*), topsep=-4pt, itemsep=-1ex]}
  {\end{enumerate} \textbf{}\vspace{-11pt}}


% Lines below are to avoid word breaks.
\tolerance=1
\emergencystretch=\maxdimen
\hyphenpenalty=10000
\hbadness=10000

\renewenvironment{itemize}
{\begin{list}
             {\labelitemi}%                     Old parameters:
             {\setlength{\labelwidth}{1.3em}%        1em
              \setlength{\labelsep}{0.7em}%          0.7em
              \setlength{\itemindent}{0em}%          0em
              \setlength{\listparindent}{3em}%       3em
              \setlength{\leftmargin}{2em}%          3em !
              \setlength{\rightmargin}{0em}%         0em
              \setlength{\parsep}{0ex}%              0ex
              \setlength{\topsep}{0.5ex}%            2ex !
              \setlength{\itemsep}{1ex}%             0ex
             }
}
{\end{list}}

\pagestyle{empty}


\begin{document}

\quizTitle{Логіка прэдыкатаў. Камбінаторыка.}
{Логика предикатов. Комбинаторика.}
{1}

\begin{problemList}

\problemItemSimple
{Ф}
{Получите высказывание $\sqrt[3]{27} = 3$ как значение некоторого предиката.}

\problemItemSimple
{Ф}
{Запишите предикат $K(x, y) = (x \leq y)$ с помощью предикатов $P(x, y) = (x < y)$, $Q(x, y) = (x = y)$ и логических операций. Все предикаты определены на множестве вещественных чисел.}

\problemItemSimple
{Ф}
{Найдите и изобразите на плоскости область истинности предиката $Q(x, y) = (x + y > 4 \wedge x - y = 2)$, определённого на множестве вещественных чисел.}

\problemItemSimple
{Ф}
{На полке стоят десять томов Пушкина, четыре тома Лермонтова и шесть томов Гоголя. Сколькими способами можно выбрать с полки одну книгу?}

\problemItemSimple
{Ф}
{Коля любит коллекционировать диски с музыкой. У него есть полные дискографии Avril Lavigne, Lady Gaga и Powerwolf, в которых по 6, 6 и 7 альбомов соответственно, которые нужно расставить на полке так, чтобы альбомы каждого исполнителя стояли подряд. Сколькими способами это можно сделать?}

\problemItemSimple
{Ф}
{Офисный работник Саня заскучал и решил побросать скомканные бумажки в урны. Он взял 10 мусорных корзин, пронумеровал и расставил в ряд. Всего Саня забросил 51 бумажку. Сколько существует способов распределения бумажек по урнам?}

\problemItemSimple
{Ф}
{22 программиста из EPAM собираются провести футбольный матч. Чтобы не отставать от тактических трендов, обе команды будут играть по схеме 1-3-5-2. Сколько всего существует способов создать различные команды из этих игроков? 2 команды считаются различными, если выполняется одно из условий: 1) команды не состоят из одних и тех же 11 игроков; 2) существует игрок, который имеет разные позиции в каждой из команд.}

\problemItemSimple
{Ф}
{2 друга Коли, Слава и Влад, знают, что Коля фанат Avril Lavigne, и попросили его составить плейлисты из её песен. Коля решил познакомить друзей с новым творчеством певицы, поэтому он планирует составить из 3 последних альбомов, в которых 19 (14 + 5 бонусных), 13 и 12 песен соответственно, 2 плейлиста по 22 песни: один для Славы, второй для Пети "--- так, чтобы в каждом из них была как минимум одна песня из каждого альбома. Сколько вариантов есть у Коли? Порядок следования композиций внутри плейлиста не важен.}

\end{problemList}
    
\newpage 

\quizTitle{Логіка прэдыкатаў. Камбінаторыка.}
{Логика предикатов. Комбинаторика.}
{2}

\begin{problemList}

\problemItemSimple
{Ф}
{Получите высказывание $\log_{2}{6} = 3$ как значение некоторого предиката.}

\problemItemSimple
{Ф}
{Запишите предикат $L(y, z) = (y \geq z)$ с помощью предикатов $R(y, z) = (y < z)$, $S(y, z) = (y =z)$ и логических операций. Все предикаты определены на множестве вещественных чисел.}

\problemItemSimple
{Ф}
{Найдите и изобразите на плоскости область истинности предиката $S(x, y) = (y^2-x < 0 \wedge x > 0)$, определённого на множестве вещественных чисел.}

\problemItemSimple
{Ф}
{В магазине есть 7 видов пиджаков, 5 видов брюк и 4 вида галстуков. Сколькими способами можно купить комплект из пиджака, брюк и галстука?}

\problemItemSimple
{Ф}
{Контрольную работу по дискретной математике пишут 3 подгруппы, в которых 14, 10 и 12 человек. Преподавателю нужно сложить работы в одну стопку так, чтобы работы были отсортированы по подгруппам. Сколькими способами он может это сделать?}

\problemItemSimple
{Ф}
{Фруктовый дискаунтер <<Виноград зелёный>> проводит акцию: <<Любой фрукт стоит по 20 копеек за штуку>>. Студент Слава увидел рекламу, обнаружил у себя в кармане 2 рубля и решил потратить их все в <<Зелёном винограде>>. Он любит бананы, апельсины и яблоки. Сколько существует вариантов купить 10 любимых фруктов (предполагается, что в магазине фруктов хватит на любой вариант)?}

\problemItemSimple
{Ф}
{После ухода Егора Крида из Black Star Тимати задумался о распаде лейбла на два: в одном будет 6 исполнителей плюс Тимати в качестве лидера, и этот лейбл сохранит за собой текущее название; другой лейбл будет называться по-другому, и в нём будут оставшиеся 6 рэперов, среди которых будет выбран лидер. Тимати интересно считать только деньги, поэтому нужно ему помочь и посчитать количество способов такого переформирования Black Star.}

\problemItemSimple
{Ф}
{Андрей решил на выходных пересмотреть все любимые фильмы с актёром Мэттом Дэймоном. Оказалось, что среди них 4 фильма из 90-х, 5 фильмов из 00-х и 3 фильма из 10-х. Сколько вариантов подборок по 6 фильмов на каждый из двух дней он может составить, если не имеет значения, в каком порядке смотреть фильмы, и при этом в каждую подборку должны попасть фильмы каждого из десятилетий?}

\end{problemList}

\end{document}