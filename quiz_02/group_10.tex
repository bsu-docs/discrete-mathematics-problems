\documentclass[12pt, a4paper]{article}
\usepackage{import}

\subimport{../common/}{preamble}

\begin{document}

\quizTitle{Логіка прэдыкатаў. Камбінаторыка.}
{Логика предикатов. Комбинаторика.}
{1}

\begin{problemList}

\problemItemSimple
{Атрымайце выказванне $\sqrt[3]{27} = 3$ як значэнне некаторага прэдыката.}
{Получите высказывание $\sqrt[3]{27} = 3$ как значение некоторого предиката.}

\bigskip

\problemItemSimple
{Запішыце прэдыкат $K(x, y) = (x \le y)$ з дапамогай прэдыкатаў $P(x, y) = (x < y)$, $Q(x, y) = (x = y)$ і лагічных аперацый.
Усе прэдыкаты вызначаныя на мностве рэчаісных лікаў.}
{Запишите предикат $K(x, y) = (x \le y)$ с помощью предикатов $P(x, y) = (x < y)$, $Q(x, y) = (x = y)$ и логических операций.
Все предикаты определены на множестве вещественных чисел.}

\bigskip

\problemItemSimple
{Знайдзіце і адлюструйце на плоскасці вобласць праўдзівасці прэдыката $Q(x, y) = (x + y > 4 \wedge x - y = 2)$, вызначанага на мностве рэчаісных лікаў.}
{Найдите и изобразите на плоскости область истинности предиката $Q(x, y) = (x + y > 4 \wedge x - y = 2)$, определённого на множестве вещественных чисел.}

\bigskip

\problemItemSimple
{На паліцы стаяць дзесяць тамоў Пушкіна, чатыры тамы Лермантава і шэсць тамоў Гогаля.
Колькі існуе спосабаў выбраць з паліцы адну кнігу?}
{На полке стоят десять томов Пушкина, четыре тома Лермонтова и шесть томов Гоголя.
Сколькими способами можно выбрать с полки одну книгу?}

\bigskip

\problemItemSimple
{Коля любіць калекцыянаваць дыскі з музыкай.
У яго ёсць поўныя дыскаграфіі Avril Lavigne, Lady Gaga і Powerwolf, у якіх адпаведна па 6, 6, і 7 альбомаў,
якія трэба расставіць на паліцы так, каб альбомы кожнага выканаўцы стаялі запар.
Колькі існуе спосабаў зрабіць гэта?}
{Коля любит коллекционировать диски с музыкой.
У него есть полные дискографии Avril Lavigne, Lady Gaga и Powerwolf, в которых по 6, 6 и 7 альбомов соответственно,
которые нужно расставить на полке так, чтобы альбомы каждого исполнителя стояли подряд.
Сколькими способами это можно сделать?}

\bigskip

\problemItemSimple
{Офісны работнік Саня засумаваў і вырашыў пакідаць скамячаныя паперкі ў сметніцы.
Ён узяў 10 карзін, пранумараваў і расставіў у шэраг.
Сумарна Саня закінуў 51 паперку. Колькі існуе спосабаў размеркавання паперак па урнах?}
{Офисный работник Саня заскучал и решил побросать скомканные бумажки в урны.
Он взял 10 мусорных корзин, пронумеровал и расставил в ряд.
Всего Саня забросил 51 бумажку. Сколько существует способов распределения бумажек по урнам?}

\bigskip

\problemItemSimple
{22 праграмісты з ЕПАМ збіраюцца правесці футбольны матч.
Каб не адставаць ад тактычных трэндаў, абедзве каманды будуць гуляць па схеме 1-3-5-2.
Колькі ўсяго існуе спосабаў стварыць адрозныя каманды з гэтых гульцоў?
2 каманды лічацца адрознымі, калі выконваецца адна з умоў:
1)~каманды не складаюцца з адных і тых жа 11 гульцоў;
2)~існуе гулец, які мае розныя пазіцыі ў кожнай з каманд.}
{22 программиста из EPAM собираются провести футбольный матч.
Чтобы не отставать от тактических трендов, обе команды будут играть по схеме 1-3-5-2.
Сколько всего существует способов создать различные команды из этих игроков?
2 команды считаются различными, если выполняется одно из условий:
1)~команды не состоят из одних и тех же 11 игроков;
2)~существует игрок, который имеет разные позиции в каждой из команд.}

\bigskip

\problemItemSimple
{2 сябра Колі, Слава і Улад, ведаюць, што Коля фанат Avril Lavigne, і папрасілі яго скласці плэйлісты з ейных песень.
Коля вырашыў пазнаёміць сяброў з новай творчасцю спявачкі, таму ён плануе скласці з 3 апошніх альбомаў,
у якіх адпаведна 19 (14 + 5 бонусных), 13 і 12 песень, 2 плэйлісты па 22 песні: адзін для Славы, другі для Пеці "--- так,
каб у кожным з іх была прынамсі адна песня з кожнага альбома.
Колькі варыянтаў ёсць у Колі? Парадак следавання кампазіцый унутры плэйліста не важны.}
{2 друга Коли, Слава и Влад, знают, что Коля фанат Avril Lavigne, и попросили его составить плейлисты из её песен.
Коля решил познакомить друзей с новым творчеством певицы, поэтому он планирует составить из 3 последних альбомов,
в которых 19 (14 + 5 бонусных), 13 и 12 песен соответственно, 2 плейлиста по 22 песни: один для Славы, второй для Пети "--- так,
чтобы в каждом из них была как минимум одна песня из каждого альбома.
Сколько вариантов есть у Коли? Порядок следования композиций внутри плейлиста не важен.}

\end{problemList}
    
\newpage 

\quizTitle{Логіка прэдыкатаў. Камбінаторыка.}
{Логика предикатов. Комбинаторика.}
{2}

\begin{problemList}

\problemItemSimple
{Атрымайце выказванне $\log_{2}{6} = 3$ як значэнне некаторага прэдыката.}
{Получите высказывание $\log_{2}{6} = 3$ как значение некоторого предиката.}

\bigskip

\problemItemSimple
{Запішыце прэдыкат $L(y, z) = (y \ge z)$ з дапамогай прэдыкатаў $R(y, z) = (y < z)$, $S(y, z) = (y =z)$ і лагічных аперацый.
Усе прэдыкаты вызначаныя на мностве рэчаісных лікаў.}
{Запишите предикат $L(y, z) = (y \ge z)$ с помощью предикатов $R(y, z) = (y < z)$, $S(y, z) = (y =z)$ и логических операций.
Все предикаты определены на множестве вещественных чисел.}

\bigskip

\problemItemSimple
{Знайдзіце і адлюструйце на плоскасці вобласць праўдзівасці прэдыката $S(x, y) = (y^2-x < 0 \wedge x > 0)$, вызначанага на мностве рэчаісных лікаў.}
{Найдите и изобразите на плоскости область истинности предиката $S(x, y) = (y^2-x < 0 \wedge x > 0)$, определённого на множестве вещественных чисел.}

\bigskip

\problemItemSimple
{У краме ёсць 7 тыпаў пінжакоў, 5 тыпаў нагавіц і 4 тыпы гальштукаў.
Колькі існуе спосабаў набыць камплект з пінжака, нагавіц і гальштука?}
{В магазине есть 7 видов пиджаков, 5 видов брюк и 4 вида галстуков.
Сколькими способами можно купить комплект из пиджака, брюк и галстука?}

\bigskip

\problemItemSimple
{Кантрольную работу па дыскрэтнай матэматыцы пішуць 3 падгрупы, у якіх 14, 10 і 12 чалавек.
Выкладчыку трэба скласці работы ў адзін стос так, каб работы былі адсартаваныя па падгрупах.
Колькі спосабаў ёсць у выкладчыка?}
{Контрольную работу по дискретной математике пишут 3 подгруппы, в которых 14, 10 и 12 человек.
Преподавателю нужно сложить работы в одну стопку так, чтобы работы были отсортированы по подгруппам.
Сколькими способами он может это сделать?}

\bigskip

\problemItemSimple
{Фруктовы дыскаўнтэр <<Вінаград зялёны>> праводзіць акцыю: <<Любы фрукт каштуе па 20 капеек за штуку>>.
Студэнт Слава ўбачыў рэкламу, знайшоў у сваёй кішэні 2 рублі і вырашыў патраціць іх усе ў <<Вінаградзе зялёным>>.
Ён любіць бананы, апельсіны і яблыкі. Колькі існуе спосабаў набыць 10 любімых фруктаў
(дапускаецца, што ў краме садавіны хопіць на любы варыянт)?}
{Фруктовый дискаунтер <<Виноград зелёный>> проводит акцию: <<Любой фрукт стоит по 20 копеек за штуку>>.
Студент Слава увидел рекламу, обнаружил у себя в кармане 2 рубля и решил потратить их все в <<Винограде зелёном>>.
Он любит бананы, апельсины и яблоки. Сколько существует вариантов купить 10 любимых фруктов
(предполагается, что в магазине фруктов хватит на любой вариант)?}

\bigskip

\problemItemSimple
{Пасля сыходу Ягора Крыда з Black Star Цімаці задумаўся аб распадзе лэйбла на два:
у адным будзе 6 выканаўцаў плюс Цімаці ў якасці лідара, і гэты лэйбл захоўвае за сабой бягучую назву;
другі лэйбл будзе называцца па-іншаму, і ў ім будуць астатнія 6 рэпераў, сярод якіх будзе выбраны лідар.
Цімаці цікава лічыць толькі грошы, таму трэба яму дапамагчы і палічыць колькасць спосабаў такога разбіцця Black Star.}
{После ухода Егора Крида из Black Star Тимати задумался о распаде лейбла на два:
в одном будет 6 исполнителей плюс Тимати в качестве лидера, и этот лейбл сохранит за собой текущее название;
второй лейбл будет называться по-другому, и в нём будут оставшиеся 6 рэперов, среди которых будет выбран лидер.
Тимати интересно считать только деньги, поэтому нужно ему помочь и посчитать количество способов такого разбиения Black Star.}

\bigskip

\problemItemSimple
{Андрэй вырашыў на выходных пераглядзець усе любімыя фільмы з акцёрам Мэтам Дэйманам.
Высветлілася, што сярод іх 4 фільмы з 90-х, 5 фільмаў з 00-х і 3 фільмы з 10-х.
Колькі варыянтаў падборак па 6 фільмаў на кожны з двух дзён ён можа скласці,
калі не мае значэння, у якім парадку глядзець фільмы,
і пры гэтым у кожную падборку павінны патрапіць фільмы кожнага з дзесяцігоддзяў?}
{Андрей решил на выходных пересмотреть все любимые фильмы с актёром Мэттом Дэймоном.
Оказалось, что среди них 4 фильма из 90-х, 5 фильмов из 00-х и 3 фильма из 10-х.
Сколько вариантов подборок по 6 фильмов на каждый из двух дней он может составить,
если не имеет значения, в каком порядке смотреть фильмы,
и при этом в каждую подборку должны попасть фильмы каждого из десятилетий?}

\end{problemList}

\end{document}