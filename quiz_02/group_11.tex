\documentclass[12pt,twoside]{article}

\setlength{\textwidth}{166mm}
\setlength{\textheight}{232mm}
\setlength{\topmargin}{-10mm}
\setlength{\headsep}{5mm}
\oddsidemargin=3mm
\evensidemargin=3mm
\setlength{\baselineskip}{18pt}

\usepackage[utf8]{inputenc}
\usepackage[russian]{babel}
\usepackage{amsfonts,amssymb,amsmath}
\usepackage{epsfig}
\usepackage{mathrsfs}
\usepackage{mathabx}
\usepackage{xcolor}

\renewcommand\le{\leqslant}
\renewcommand\ge{\geqslant}

\newcommand{\ruText}[1]{
  {\footnotesize \textcolor{darkgray}{#1} \par}
}

\newcommand{\biLangHeader}[2]{
  \subsection*{%
  	#1 \\%
  	\ruText{#2}%
  }%
}

\newcommand{\quizTitle}[3]{%
\begin{center}
	\textbf{Кантрольная работа па тэме <<#1>> (варыянт #3)} \\
	\ruText{Контрольная работа по теме <<#2>> (вариант #3)}
\end{center}
}

\newcommand{\problemItemSimple}[2]{%
	\item #1 \\%
	\ruText{#2}%
}

\newcommand{\problemItemWithCommonPart}[3]{%
	\item #1 \\%
	\ruText{#2}%
	#3%
}

\newcommand{\problemItemWithCommonPartComplicated}[5]{%
	\item #1 \\%
	\ruText{#2}%
	#3 \\
	\noindent #4 \\%
	\noindent \ruText{#5}%
}

\makeatletter
\def\belarusianLetters#1{
  \expandafter\@belarusianLetters\csname c@#1\endcsname
}
\def\@belarusianLetters#1{
  (%
  \ifcase#1\or а\or б\or в\or г\or д\or е\or ж\or з\or і\or к\or л\or м\fi%
  )
}
\makeatother
\AddEnumerateCounter{\belarusianLetters}{\@belarusianLetters}{Ы}

\newenvironment{problemList}
  {\begin{enumerate}[leftmargin=*,topsep=0pt,itemsep=-1ex,partopsep=1ex,parsep=1ex]}
  {\end{enumerate}}

\newenvironment{belarusianEnumerate}
  {\begin{enumerate}[label=\belarusianLetters*, topsep=-7pt]}
  {\end{enumerate} \textbf{}\vspace{-8pt}}

\AddEnumerateCounter{\asbuk}{\@asbuk}{\cyrm}
\newenvironment{russianEnumerate}
  {\begin{enumerate}[label=(\asbuk*), topsep=-4pt, itemsep=-1ex]}
  {\end{enumerate} \textbf{}\vspace{-11pt}}


% Lines below are to avoid word breaks.
\tolerance=1
\emergencystretch=\maxdimen
\hyphenpenalty=10000
\hbadness=10000

\renewenvironment{itemize}
{\begin{list}
             {\labelitemi}%                     Old parameters:
             {\setlength{\labelwidth}{1.3em}%        1em
              \setlength{\labelsep}{0.7em}%          0.7em
              \setlength{\itemindent}{0em}%          0em
              \setlength{\listparindent}{3em}%       3em
              \setlength{\leftmargin}{2em}%          3em !
              \setlength{\rightmargin}{0em}%         0em
              \setlength{\parsep}{0ex}%              0ex
              \setlength{\topsep}{0.5ex}%            2ex !
              \setlength{\itemsep}{1ex}%             0ex
             }
}
{\end{list}}

\pagestyle{empty}


\begin{document}

\quizTitle{Логіка прэдыкатаў. Камбінаторыка.}
{Логика предикатов. Комбинаторика.}
{1}

\begin{problemList}

\problemItemSimple
{Знайдзіце вобласць праўдзіваці прэдыката $P(x, y) = (x + y = 5)$, вызначанага на мностве натуральных лікаў.}
{Найдите область истинности предиката $P(x, y) = (x + y = 5)$, определенного на множестве натуральных чисел.}

\bigskip

\problemItemSimple
{Надайце формуле логікі прэдыкатаў $P(x, y) \to (\forall{ z Q(x, y, z)})$ інтэрпрэтацыю $P(x, y) = (x < y)$ і $Q(x, y, z) = (x - y \cdot z < (1 - z) \cdot y)$ на мностве рэчаісных лікаў.
Ці праўдзівае выказванне, якое атрымліваецца ў выніку такой інтэрпрэтацыі?}
{Придайте формуле логики предикатов $P(x, y) \to (\forall{ z Q(x, y, z)})$ интерпретацию $P(x, y) = (x < y)$ и $Q(x, y, z) = (x - y \cdot z < (1 - z) \cdot y)$ на множестве вещественных чисел.
Истинно или ложно высказывание, которое получается в результате такой интерпретации?}

\bigskip

\problemItemSimple
{Вызначыце праўдзівае значэнне выказвання $\exists{x \forall{y {(x \cdot y = 10)}}}$, $x, y \in \mathbbmss{R}$.}
{Установите истинностное значение высказывания $\exists{x \forall{y {(x \cdot y = 10)}}}$, $x, y \in \mathbbmss{R}$.}

\bigskip

\problemItemSimple
{Маша збіраецца з'есці аблык, сліву і мандарын, але пакуль  не вырашыла, у якой паслядоўнасці.
Колькімі спосабамі Мажа можа выбраць гэту падпаслядоўнасць?}
{Маша собирается съесть яблоко, сливу и мандарин, но пока не решила, в какой последовательности.
Сколькими способами Маша может выбрать эту последовательность?}

\bigskip

\problemItemSimple
{Падлетак Маша зарэгістравалася на twitter.com і плануе пісаць мнагазначныя аднаслоўныя твіты.
Яна ведае, што сённяшні твітар ужо не той, які быў раней, таму будзе выкарыстоўваць старое абмежаванне ў 140 сімвалаў.
Палічыце колькасць адрозных запісаў, якія яна можа апублікаваць, карыстаючыся ўсімі 26 літарамі англійскага алфавіта,
прычым словы не абавязкова маюць сэнс (у гэтым і мнагазначнасць, мяркуе Маша).}
{Подросток Маша зарегистрировалась на twitter.com и планирует писать многозначительные однословные твиты.
Она знает, что сегодняшний твиттер уже не тот, что был раньше, поэтому будет использовать старое ограничение в 140 символов.
Посчитайте количество различных записей, которые она может опубликовать, используя все 26 букв английского алфавита,
причём слова не обязательно имеют смысл (в этом и многозначительность, по мнению Маши).}

\bigskip

\problemItemSimple
{Жора дужа любіць салодкае, асабліва батончыкі <<Mars>>, <<Snickers>> і <<Milky Way>>.
Ягоная сям'я збіраецца ў паход у лес з начоўкай.
Ведаючы пра цягу Жоры да салодкага, бацькі дазволілі сабраць заплечнік, запаўняючы яго толькі гэтымі батончыкамі.
Вядома, што ў заплечнік памяшчаецца толькі 45 батончыкаў.
Колькі адрозных спосабаў цалкам запоўніць заплечнік батончыкамі ёсць у Жоры?}
{Жора очень любит сладкое, особенно батончики <<Mars>>, <<Snickers>> и <<Milky Way>>.
Его семья собирается в поход в лес с ночёвкой.
Зная о пристастии Жоры, родители разрешили собрать рюкзак, заполнив его только этими батончиками.
Известно, что в рюкзак помещается лишь 45 батончиков.
Сколько различных способов полностью заполнить рюкзак батончиками есть у Жоры?}

\bigskip

\problemItemSimple
{Філосаф Джэйсан у краме стаў у чаргу.
Калі Джэйсан аказаўся ў сярэдзіне чаргі, у якой было $n \ge 3$ чалавек, ён задумаўся: а колькі існуе спосабаў перарасставіць людзей у чарзе так,
каб людзі, якія зараз з'яўляюцца суседзямі Джэйсана ў чарзе, стаялі як мінімум праз аднаго чалавека ад Джэйсана?
Пазіцыя Джэйсана у чарзе застаецца той жа для любой перастаноўкі.
Джэйсан не змог знайсці рашэнне самастойна, і ён задумаўся пра тленнасць быція, таму дапаможам і палічым за яго.}
{Философ Джейсон в магазине стал в очередь.
Когда Джейсон оказался в середине очереди, в которой было $n \ge 3$ человек, он задумался: а сколькими способами можно перерасставить людей в очереди так,
чтобы люди, которые сейчас стоят по соседству с Джейсоном, стояли как минимум через одного человека от Джейсона?
Позиция Джейсона в очереди остаётся прежней для любой перестановки.
Не сумев сосчитать самостоятельно, Джейсон задумался о бренности бытия, поэтому поможем и посчитаем за него.}

\bigskip

\problemItemSimple
{Сантэхнік Сцяпан даведаўся, што можна глядзець, як людзі гуляюць у кампутар, і зарэгістраваўся на twitch.tv. Сцяпан убачыў, што людзі пішуць штосьці незразумелае ў чат, і вырашыў падключыцца.
Для пачатку ён плануе напісаць 100 паведамленнняў, выкарыстаўваючы 2 самыя папулярныя паведамленні: <<OMEGALUL>> і <<POG CHAMP>>.
Колькі спосабаў існуе напісаць 60 паведамленняў <<OMEGALUL>> і 40 паведамленняў <<POG CHAMP>> так, каб усе паведамленні аднаго тыпу не ішлі запар?}
{Сантехник Степан узнал, что можно смотреть, как люди играют в компьютер, и зарегистрировался на twitch.tv.
Степан увидел, что люди пишут что-то непонятное в чате, и решил подключиться.
Для начала он планирует написать 100 сообщений, используя 2 самых популярных сообщения: <<OMEGALUL>> и <<POG CHAMP>>.
Сколько способов существует написать 60 сообщений <<OMEGALUL>> и 40 сообщений <<POG CHAMP>> так, чтобы все сообщения одного типа не шли подряд?}

\end{problemList}

\end{document}