\documentclass[12pt,twoside]{article}

\setlength{\textwidth}{166mm}
\setlength{\textheight}{232mm}
\setlength{\topmargin}{-10mm}
\setlength{\headsep}{5mm}
\oddsidemargin=3mm
\evensidemargin=3mm
\setlength{\baselineskip}{18pt}

\usepackage[utf8]{inputenc}
\usepackage[russian]{babel}
\usepackage{amsfonts,amssymb,amsmath}
\usepackage{epsfig}
\usepackage{mathrsfs}
\usepackage{mathabx}
\usepackage{xcolor}

\renewcommand\le{\leqslant}
\renewcommand\ge{\geqslant}

\newcommand{\ruText}[1]{
  {\footnotesize \textcolor{darkgray}{#1} \par}
}

\newcommand{\biLangHeader}[2]{
  \subsection*{%
  	#1 \\%
  	\ruText{#2}%
  }%
}

\newcommand{\quizTitle}[3]{%
\begin{center}
	\textbf{Кантрольная работа па тэме <<#1>> (варыянт #3)} \\
	\ruText{Контрольная работа по теме <<#2>> (вариант #3)}
\end{center}
}

\newcommand{\problemItemSimple}[2]{%
	\item #1 \\%
	\ruText{#2}%
}

\newcommand{\problemItemWithCommonPart}[3]{%
	\item #1 \\%
	\ruText{#2}%
	#3%
}

\newcommand{\problemItemWithCommonPartComplicated}[5]{%
	\item #1 \\%
	\ruText{#2}%
	#3 \\
	\noindent #4 \\%
	\noindent \ruText{#5}%
}

\makeatletter
\def\belarusianLetters#1{
  \expandafter\@belarusianLetters\csname c@#1\endcsname
}
\def\@belarusianLetters#1{
  (%
  \ifcase#1\or а\or б\or в\or г\or д\or е\or ж\or з\or і\or к\or л\or м\fi%
  )
}
\makeatother
\AddEnumerateCounter{\belarusianLetters}{\@belarusianLetters}{Ы}

\newenvironment{problemList}
  {\begin{enumerate}[leftmargin=*,topsep=0pt,itemsep=-1ex,partopsep=1ex,parsep=1ex]}
  {\end{enumerate}}

\newenvironment{belarusianEnumerate}
  {\begin{enumerate}[label=\belarusianLetters*, topsep=-7pt]}
  {\end{enumerate} \textbf{}\vspace{-8pt}}

\AddEnumerateCounter{\asbuk}{\@asbuk}{\cyrm}
\newenvironment{russianEnumerate}
  {\begin{enumerate}[label=(\asbuk*), topsep=-4pt, itemsep=-1ex]}
  {\end{enumerate} \textbf{}\vspace{-11pt}}


% Lines below are to avoid word breaks.
\tolerance=1
\emergencystretch=\maxdimen
\hyphenpenalty=10000
\hbadness=10000

\renewenvironment{itemize}
{\begin{list}
             {\labelitemi}%                     Old parameters:
             {\setlength{\labelwidth}{1.3em}%        1em
              \setlength{\labelsep}{0.7em}%          0.7em
              \setlength{\itemindent}{0em}%          0em
              \setlength{\listparindent}{3em}%       3em
              \setlength{\leftmargin}{2em}%          3em !
              \setlength{\rightmargin}{0em}%         0em
              \setlength{\parsep}{0ex}%              0ex
              \setlength{\topsep}{0.5ex}%            2ex !
              \setlength{\itemsep}{1ex}%             0ex
             }
}
{\end{list}}

\pagestyle{empty}


\begin{document}

\quizTitle{Логіка прэдыкатаў. Камбінаторыка.}
{Логика предикатов. Комбинаторика.}
{1}

\begin{problemList}

\problemItemSimple
{Ф}
{A}

\problemItemSimple
{Ф}
{A}

\problemItemSimple
{Ф}
{A}

\problemItemSimple
{Ф}
{A}

\problemItemSimple
{Ф}
{Подросток Маша зарегистрировалась на twitter.com и планирует писать многозначительные однословные твиты. Она знает, что сегодняшний твиттер уже не тот, что был раньше, поэтому будет использовать старое ограничение в 140 символов. Посчитайте количество различных записей, которые она может опубликовать, используя все 26 букв английского алфавита, причём слова не обязательно имеют смысл (в этом и многозначительность, по мнению Маши).}

\problemItemSimple
{Ф}
{Жора очень любит сладкое, особенно батончики <<Mars>>, <<Snickers>> и <<Milky Way>>. Его семья собирается в поход в лес с ночёвкой. Зная о пристастии Жоры, родители разрешили собрать рюкзак, заполнив его только этими батончиками. Известно, что в рюкзак помещается лишь 45 батончиков. Сколько различных способов собрать рюкзак в поход есть у Жоры?}

\problemItemSimple
{Ф}
{Философ Джейсон в магазине стал в очередь. Когда Джейсон оказался в середине очереди, в которой было $n \geq 4$ человек, он задумался: а сколькими способами можно перерасставить этих людей в очереди, чтобы люди, которые сейчас стоят по соседству с Джейсоном, стояли как минимум через одного человека от Джейсона? Не сумев сосчитать самостоятельно, Джейсон задумался о бренности бытия, поэтому поможем и посчитаем за него.}

\problemItemSimple
{Ф}
{Сантехник Степан узнал, что можно смотреть, как люди играют в компьютер, и зарегистрировался на twitch.tv. Степан увидел, что люди пишут что-то непонятное в чате, и решил подключиться. Для начала он планирует написать 100 сообщений, используя 2 самых популярных вида сообщений: <<OMEGALUL>> и <<POG CHAMP>>. Сколько способов существует написать эти сообщения в соотношении 60-40, причём все сообщения одного типа не должны идти подряд?}

\end{problemList}

\end{document}