\usepackage[utf8]{inputenc}
\usepackage[russian]{babel}
\usepackage{amsfonts,amssymb,amsmath}
\usepackage{float}
\usepackage{epsfig}
\usepackage{mathrsfs}
\usepackage{mathabx}
\usepackage{xcolor}
\usepackage{enumitem}
\usepackage{hyperref}
\usepackage{bbm}
\usepackage{geometry}
\usepackage{ifthen}

\geometry{left=2cm, right=1cm, top=1.5cm, bottom=1.5cm}
\newcommand{\ruText}[1]{
  {\scriptsize \textcolor{darkgray}{#1} \par}
}

\newcommand{\biLangHeader}[2]{
  \subsection*{
    {\normalsize #1} \\
    \indent \ruText{#2}
  }
}


% Lines below are to avoid word breaks.
\tolerance=1
\emergencystretch=\maxdimen
\hyphenpenalty=10000
\hbadness=10000

\pagestyle{empty}

\usepackage{titlesec}
\titleformat{\subsection}[display]{\bfseries\filright}{}{}{}



\begin{document}

\quizTitle{Логіка прэдыкатаў. Камбінаторыка.}
{Логика предикатов. Комбинаторика.}
{1}

\begin{problemList}

\problemItemSimple
{Ф}
{A}

\problemItemSimple
{Ф}
{A}

\problemItemSimple
{Ф}
{A}

\problemItemSimple
{Ф}
{A}

\problemItemSimple
{Ф}
{Подросток Маша зарегистрировалась на twitter.com и планирует писать многозначительные однословные твиты. Она знает, что сегодняшний твиттер уже не тот, что был раньше, поэтому будет использовать старое ограничение в 140 символов. Посчитайте количество различных записей, которые она может опубликовать, используя все 26 букв английского алфавита, причём слова не обязательно имеют смысл (в этом и многозначительность, по мнению Маши).}

\problemItemSimple
{Ф}
{Жора очень любит сладкое, особенно батончики <<Mars>>, <<Snickers>> и <<Milky Way>>. Его семья собирается в поход в лес с ночёвкой. Зная о пристастии Жоры, родители разрешили собрать рюкзак, заполнив его только этими батончиками. Известно, что в рюкзак помещается лишь 45 батончиков. Сколько различных способов собрать рюкзак в поход есть у Жоры?}

\problemItemSimple
{Ф}
{Философ Джейсон в магазине стал в очередь. Когда Джейсон оказался в середине очереди, в которой было $n \geq 4$ человек, он задумался: а сколькими способами можно перерасставить этих людей в очереди, чтобы люди, которые сейчас стоят по соседству с Джейсоном, стояли как минимум через одного человека от Джейсона? Не сумев сосчитать самостоятельно, Джейсон задумался о бренности бытия, поэтому поможем и посчитаем за него.}

\problemItemSimple
{Ф}
{Сантехник Степан узнал, что можно смотреть, как люди играют в компьютер, и зарегистрировался на twitch.tv. Степан увидел, что люди пишут что-то непонятное в чате, и решил подключиться. Для начала он планирует написать 100 сообщений, используя 2 самых популярных вида сообщений: <<OMEGALUL>> и <<POG CHAMP>>. Сколько способов существует написать эти сообщения в соотношении 60-40, причём все сообщения одного типа не должны идти подряд?}

\end{problemList}

\end{document}