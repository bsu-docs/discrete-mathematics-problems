\documentclass[12pt,twoside]{article}

\setlength{\textwidth}{166mm}
\setlength{\textheight}{232mm}
\setlength{\topmargin}{-10mm}
\setlength{\headsep}{5mm}
\oddsidemargin=3mm
\evensidemargin=3mm
\setlength{\baselineskip}{18pt}

\usepackage[utf8]{inputenc}
\usepackage[russian]{babel}
\usepackage{amsfonts,amssymb,amsmath}
\usepackage{epsfig}
\usepackage{mathrsfs}
\usepackage{mathabx}
\usepackage{xcolor}

\renewcommand\le{\leqslant}
\renewcommand\ge{\geqslant}

\newcommand{\ruText}[1]{
  {\footnotesize \textcolor{darkgray}{#1} \par}
}

\newcommand{\biLangHeader}[2]{
  \subsection*{%
  	#1 \\%
  	\ruText{#2}%
  }%
}

\newcommand{\quizTitle}[3]{%
\begin{center}
	\textbf{Кантрольная работа па тэме <<#1>> (варыянт #3)} \\
	\ruText{Контрольная работа по теме <<#2>> (вариант #3)}
\end{center}
}

\newcommand{\problemItemSimple}[2]{%
	\item #1 \\%
	\ruText{#2}%
}

\newcommand{\problemItemWithCommonPart}[3]{%
	\item #1 \\%
	\ruText{#2}%
	#3%
}

\newcommand{\problemItemWithCommonPartComplicated}[5]{%
	\item #1 \\%
	\ruText{#2}%
	#3 \\
	\noindent #4 \\%
	\noindent \ruText{#5}%
}

\makeatletter
\def\belarusianLetters#1{
  \expandafter\@belarusianLetters\csname c@#1\endcsname
}
\def\@belarusianLetters#1{
  (%
  \ifcase#1\or а\or б\or в\or г\or д\or е\or ж\or з\or і\or к\or л\or м\fi%
  )
}
\makeatother
\AddEnumerateCounter{\belarusianLetters}{\@belarusianLetters}{Ы}

\newenvironment{problemList}
  {\begin{enumerate}[leftmargin=*,topsep=0pt,itemsep=-1ex,partopsep=1ex,parsep=1ex]}
  {\end{enumerate}}

\newenvironment{belarusianEnumerate}
  {\begin{enumerate}[label=\belarusianLetters*, topsep=-7pt]}
  {\end{enumerate} \textbf{}\vspace{-8pt}}

\AddEnumerateCounter{\asbuk}{\@asbuk}{\cyrm}
\newenvironment{russianEnumerate}
  {\begin{enumerate}[label=(\asbuk*), topsep=-4pt, itemsep=-1ex]}
  {\end{enumerate} \textbf{}\vspace{-11pt}}


% Lines below are to avoid word breaks.
\tolerance=1
\emergencystretch=\maxdimen
\hyphenpenalty=10000
\hbadness=10000

\renewenvironment{itemize}
{\begin{list}
             {\labelitemi}%                     Old parameters:
             {\setlength{\labelwidth}{1.3em}%        1em
              \setlength{\labelsep}{0.7em}%          0.7em
              \setlength{\itemindent}{0em}%          0em
              \setlength{\listparindent}{3em}%       3em
              \setlength{\leftmargin}{2em}%          3em !
              \setlength{\rightmargin}{0em}%         0em
              \setlength{\parsep}{0ex}%              0ex
              \setlength{\topsep}{0.5ex}%            2ex !
              \setlength{\itemsep}{1ex}%             0ex
             }
}
{\end{list}}

\pagestyle{empty}


\begin{document}

\quizTitle
{Логіка~выказванняў.~Мноствы.~Адлюстраванні.}
{Логика высказываний.~Множества.~Отображения.}
{1}

\begin{problemList}

\problemItemSimple
{Што можна сказаць наконт праўдзівасці выказвання $(\overline{A} \cdot B) \to C$, калі вядома, што~$A = \mbox{П}$?}
{Что можно сказать об истинностном значении высказывания $(\overline{A} \cdot B) \to C$, если известно, что $A = \mbox{И}$?}

\bigskip

\problemItemSimple
{Дакажыце, што формула логікі выказванняў $(((A \to \overline{B}) \to B) \to B)$ з'яўляецца таўталогіяй.}
{Докажите, что формула логики высказываний $(((A \to \overline{B}) \to B) \to B)$ является тавтологией.}

\bigskip

\problemItemSimple
{Пры складанні раскладу заняткаў на некаторы дзень настаўнікі выказалі наступныя пажаданні:
\begin{belarusianEnumerate}
    \item матэматык прасіў паставіць ягоны занятак першым ці другім;
    \item гісторык жадае правесці свой занятак першым ці трэцім;
    \item настаўнік рускай мовы і літаратуры прасіў паставіць ягоны занятак другім ці~трэцім.
\end{belarusianEnumerate}

\vspace{-8pt}
Ці можна задаволіць просьбы ўсіх настаўнікаў?}
{При составлении расписания уроков на некоторый день учителя высказали следующие пожелания:
\begin{russianEnumerate}
    \item математик просил поставить его урок первым или вторым;
    \item историк желает провести свой урок первым или третьим;
    \item учитель русского языка и литературы просил поставить его урок вторым или третьим.
\end{russianEnumerate}\\
Можно ли удовлетворить просьбы всех учителей?}

\bigskip

\problemItemSimple
{Нядзеля не працоўны дзень. Сёння не працоўны дзень. Ці значыць гэта, што сёння нядзеля?}
{Воскресенье не рабочий день. Сегодня не рабочий день. Значит ли, что сегодня воскресенье?}

\bigskip

\problemItemSimple
{Дакажыце тэарэтыка-мноственную тоеснасць $\overline{A \cap B} = \overline{A} \cup \overline{B}$,
карыстаючыся азначэннем роўнасці мностваў.}
{Докажите теоретико-множественное тождество $\overline{A \cap B} = \overline{A} \cup \overline{B}$,
используя определение равенства множеств.}

\bigskip

\problemItemSimple
{Ці існуюць такія мноствы $A$, $B$ і $C$, што
$A \cap B \neq \emptyset$, $A \cap C = \emptyset$ і $(A \cap B) \setminus C = \emptyset$?}
{Существуют ли такие множества $A, B$ и $C$, что
$A \cap B \neq \emptyset$, $A \cap C = \emptyset$ и $(A \cap B) \setminus C = \emptyset$?}

\bigskip

\problemItemSimple
{Ці з'яўляецца адлюстраванне
$f: \mathbbmss{R} \setminus \{1\} \mapsto \mathbbmss{R} \setminus \{2\}$, $f(x) = \frac{2 x}{x - 1}$
біектыўным адлюстраваннем?}
{Является ли отображение
$f: \mathbbmss{R} \setminus \{1\} \mapsto \mathbbmss{R} \setminus \{2\}$, $f(x) = \frac{2 x}{x - 1}$
биективным отображением?}

\bigskip

\problemItemSimple
{Няхай для адлюстравання $f: \mathbbmss{R} \mapsto \mathbbmss{R}$ існуюць рэчаісныя лікі
$\alpha > 0$, $\beta > 0$, такія, што $|f(x) - f(y)| \geq \alpha \cdot |x - y|^{\beta}$
для любых $x, y \in \mathbbmss{R}$. Дакажыце, што адлюстраванне $f$ з'яўляецца ін'ектыўным адлюстраваннем.}
{Пусть для отображения $f: \mathbbmss{R} \mapsto \mathbbmss{R}$ существуют вещественные числа
$\alpha > 0$, $\beta > 0$, такие, что $|f(x) - f(y)| \geq \alpha \cdot |x - y|^{\beta}$
для любых $x, y \in \mathbbmss{R}$. Докажите, что отображение $f$ является инъективным отображением.}

\end{problemList}

\newpage

\quizTitle
{Логіка~выказванняў.~Мноствы.~Адлюстраванні.}
{Логика высказываний.~Множества.~Отображения.}
{2}

\begin{problemList}

\problemItemSimple
{Што можна сказаць наконт праўдзівасці выказвання $\overline{A} \to (B \vee C)$, калі вядома, што~$A = \mbox{П}$?}
{Что можно сказать об истинностном значении высказывания $\overline{A} \to (B \vee C)$, если известно, что $A = \mbox{И}$?}

\bigskip

\problemItemSimple
{Дакажыце, што формула логікі выказванняў $(A \to (B \to C)) \to (B \to (A \to C))$ з'яўляецца таўталогіяй.}
{Докажите, что формула логики высказываний $(A \to (B \to C)) \to (B \to (A \to C))$ является тавтологией.}

\bigskip

\problemItemSimple
{Трое сяброў Андрэй (А), Барбара (Б) і Варвара (В) "--- студэнты 3-й плыні ФПМІ БДУ, якія вучацца на розных спецыяльнасцях.
Вядома, што
\begin{belarusianEnumerate}
    \item калі В "--- студэнт КБ, то Б "--- студэнт ЭК;
    \item калі В "--- студэнт ЭК, то Б "--- студэнт-актуарый;
    \item калі Б "--- не студэнт КБ, то А "--- студэнт ЭК;
    \item калі А "--- студэнт–актуарый, то В "--- студэнт ЭК.
\end{belarusianEnumerate}

\vspace{-8pt}
Вызначыце спецыяльнасці студэнтаў А, Б і В.}
{Трое друзей Андрей (А), Барбара (Б) и Варвара (В) "--- студенты 3-го потока ФПМИ БГУ, обучающиеся на разных специальностях.
Известно, что
\begin{russianEnumerate}
    \item если В "--- студент КБ, то Б "--- студент ЭК;
    \item если В "--- студент ЭК, то Б "--- студент-актуарий;
    \item если Б "--- не студент КБ, то А "--- студент ЭК;
    \item если А "--- студент–актуарий, то В "--- студент ЭК.
\end{russianEnumerate}\\
Определите специальности студентов А, Б и В.}

\bigskip

\problemItemSimple
{Панядзелак "--- працоўны дзень. Сёння працоўны дзень. Ці значыць гэта, што сёння панядзелак?}
{Понедельник "--- рабочий день. Сегодня рабочий день. Значит ли, что сегодня понедельник?}

\bigskip

\problemItemSimple
{Дакажыце тэарэтыка-мноственную тоеснасць $\overline{A \cup B} = \overline{A} \cap \overline{B}$,
карыстаючыся азначэннем роўнасці мностваў.}
{Докажите теоретико-множественное тождество $\overline{A \cup B} = \overline{A} \cap \overline{B}$,
используя определение равенства множеств.}

\bigskip

\problemItemSimple
{Ці існуюць такія мноствы $A$, $B$ і $C$, што
$A \cap B \neq \emptyset$, $A \cap C = \emptyset$ і $(A \cap B) \setminus C = \emptyset$?}
{Существуют ли такие множества $A, B$ и $C$, что
$A \cap B \neq \emptyset$, $A \cap C = \emptyset$ и $(A \cap B) \setminus C = \emptyset$?}

\bigskip

\problemItemSimple
{Ці з'яўляецца адлюстраванне $f: [0, 1] \mapsto [2, 4]$, $f(x) = \frac{4}{x + 1}$ біектыўным адлюстраваннем?}
{Является ли отображение $f: [0, 1] \mapsto [2, 4]$, $f(x) = \frac{4}{x + 1}$ биективным отображением?}

\bigskip

\problemItemSimple
{Няхай ін'ектыўнае адлюстраванне $f: A \mapsto A$ такое, што $f \circ f = f$. Дакажыце, што~$f = id_{A}$.}
{Пусть инъективное отображение $f: A \mapsto A$ такое, что $f \circ f = f$. Докажите, что~$f = id_{A}$.}

\end{problemList}

\end{document}
