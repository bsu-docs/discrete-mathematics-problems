\documentclass[12pt, a4paper]{article}

\usepackage{import}
\subimport{../../common/}{preamble}

\begin{document}

\biLangHeader{4. Мноствы.}{Множества.}

\begin{problemList}

\problemItemWithCommonPart
{Дакажыце наступныя тоеснасці, карыстаючыся азначэннем роўнасці мностваў:}
{Докажите следующие тождества, используя определение равенства множеств:}
{%
\begin{belarusianEnumerate}
    \item $A \cup (B \cap C) = (A \cup B) \cap (A \cup C)$;
    \item $A \setminus (B \cup C) = (A \setminus B) \cap (A \setminus C)$;
    \item $\overline{A_1 \cup A_2 \cup \ldots \cup A_n} = \overline{A}_1 \cap \overline{A}_2 \cap \ldots \cap \overline{A}_n$;
    \item $A \setminus (B \cup C) = (A \setminus B) \setminus C$;
    \item $(A \setminus B) \times C = (A \times C) \setminus (B \times C)$.
\end{belarusianEnumerate}
}

\smallskip

\problemItemWithCommonPart
{Дакажыце наступныя тоеснасці, карыстаючыся раўназначнымі пераўтварэннямі:}
{Докажите следующие тождества, используя равносильные преобразования:}
{%
\begin{belarusianEnumerate}
    \item $(A \cap B) \cup (C \cap D) = (A \cup C) \cap (B \cup C) \cap (A \cup D) \cap (B \cup D)$;
    \item $A \oplus B \oplus (A \cap B) = A \cup B$;
    \item $A \setminus (B \setminus C) = (A \setminus B) \cup (A \cap C)$;
    \item $A \cap (B \oplus C) = (A \cap B) \oplus (A \cap C)$.
\end{belarusianEnumerate}
}

\smallskip

\problemItemWithCommonPart
{Дакажыце наступныя сцвярджэнні:}
{Докажите следующие утверждения:}
{%
\begin{belarusianEnumerate}
    \item $A \cup B \subseteq C$ $\,\,\,\Leftrightarrow\,\,\,$ $A \subseteq C$ і $B \subseteq C$;
    \item $A \subseteq B \cap C$ $\,\,\,\Leftrightarrow\,\,\,$ $A \subseteq B$ і $A \subseteq C$;
    \item $A \cap B \subseteq C$ $\,\,\,\Leftrightarrow\,\,\,$ $A \subseteq \overline{B} \cup C$;
    \item $A \subseteq B \cup C$ $\,\,\,\Leftrightarrow\,\,\,$ $A \cap \overline{B} \subseteq C$;
    \item $A \subseteq B$ $\,\,\,\Leftrightarrow\,\,\,$ $\overline{B} \subseteq \overline{A}$;
\end{belarusianEnumerate}
}

\smallskip

\problemItemWithCommonPart
{Рашыце наступныя сістэмы раўнанняў:}
{Решите следующие системы уравнений:}
{%
\begin{belarusianEnumerate}

    \item
    \begin{equation*}
        \hspace{-24mm}
        \left\{
        \begin{aligned}
        A \cap X &= B, \\
        A \cup X &= C, \\
        \end{aligned}
        \right.
        \quad \text{дзе $A$, $B$ і $C$ "--- дадзеныя мноствы і $B \subseteq A \subseteq C$.}
    \end{equation*}

    \item
    \begin{equation*}
        \hspace{-12mm}
        \left\{
        \begin{aligned}
        A \setminus X &= B, \\
        X \setminus A &= C, \\
        \end{aligned}
        \right.
        \quad \text{дзе $A$, $B$ і $C$ "--- дадзеныя мноствы і $B \subseteq A$, $A \cap C = \emptyset$.}
    \end{equation*}

\end{belarusianEnumerate}
}

\bigskip

\item
\begin{belarusianEnumerate}

\problemItemSimple
{Ёсць паслядоўнасць мностваў: $X_0 \supseteq X_1 \supseteq X_2 \supseteq \ldots \supseteq X_n \supseteq \ldots$ \\
Дакажыце, што перакрыжаванне любой бясконцай падпаслядоўнасці гэтых мностваў супадае з перасячэннем усёй паслядоўнасці.}
{Имеется последовательность множеств: $X_0 \supseteq X_1 \supseteq X_2 \supseteq \ldots \supseteq X_n \supseteq \ldots$ \\
Докажите, что пересечение любой бесконечной подпоследовательности этих множеств совпадает с пересечением всей последовательности.}

\problemItemSimple
{Ёсць паслядоўнасць мностваў: $X_0 \subseteq X_1 \subseteq X_2 \subseteq \ldots \subseteq X_n \subseteq \ldots$ \\
Дакажыце, што аб'яднанне любой бясконцай падпаслядоўнасці гэтых мностваў супадае з аб'яднаннем усёй паслядоўнасці.}
{Имеется последовательность множеств: $X_0 \subseteq X_1 \subseteq X_2 \subseteq \ldots \subseteq X_n \subseteq \ldots$ \\
Докажите, что объединение любой бесконечной подпоследовательности этих множеств совпадает с объединением всей последовательности.}

\end{belarusianEnumerate}

\newpage

\item
\begin{belarusianEnumerate}

\problemItemSimple
{Дакажыце, што $2^{A_1 \cap A_2 \cap\, \ldots\, \cap A_n} = 2^{A_1} \cap 2^{A_2} \cap \ldots \cap 2^{A_n}$.}
{Докажите, что $2^{A_1 \cap A_2 \cap\, \ldots\, \cap A_n} = 2^{A_1} \cap 2^{A_2} \cap \ldots \cap 2^{A_n}$.}

\problemItemSimple
{Пералічыце ўсе элементы мноства $2^A$, дзе $A = \{1, 2, \{\{1\}, 2, 3\}\}$.}
{Перечислите все элементы множества $2^A$, где $A = \{1, 2, \{\{1\}, 2, 3\}\}$.}

\problemItemSimple
{Пералічыце ўсе элементы мноства $2^A$, дзе
$A = \{\emptyset, \{\emptyset\}, \{1, \{1\}\}\} \setminus \{1, \{1\}\}$.}
{Перечислите все элементы множества $2^A$, где
$A = \{\emptyset, \{\emptyset\}, \{1, \{1\}\}\} \setminus \{1, \{1\}\}$.}

\end{belarusianEnumerate}

\end{problemList}

\end{document}
