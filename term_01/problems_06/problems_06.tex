\documentclass[12pt, a4paper]{article}

\usepackage{import}
\subimport{../../common/}{preamble}

\begin{document}

\biLangHeader{6. Прынцып Дзірыхле.}{Принцип Дирихле.}

\begin{problemList}

\problemItemSimple
{Дакажыце, што ў любым мностве з 52 цэлых лікаў знойдуцца прынамсі два лікі, сума ці розніца якіх дзеліцца на 100.}
{Докажите, что в любом множестве из 52 целых чисел найдутся по крайней мере два числа, сумма или разность которых делится на 100.}

\bigskip

\problemItemSimple
{Кропка $(x, y, z) \in \mathbbmss{R}^3$ называецца \emph{цэлай}, калі $x, y, z \in \mathbbmss{Z}$. Дакажыце, што сярод дзевяці цэлых кропак знойдуцца прынамсі дзве кропкі, для якіх сярэдзіна адрэзка з канцамі ў гэтых кропках таксама з'яўляецца цэлай кропкай.}
{Точка $(x, y, z) \in \mathbbmss{R}^3$ называется \emph{целой}, если $x, y, z \in \mathbbmss{Z}$. Докажите, что среди девяти целых точек найдутся по крайней мере две точки, для которых середина отрезка с концами в этих точках также является целой точкой.}

\bigskip

\problemItemSimple
{Дакажыце, што любое падмноства $S \subset \{1, 2, 3, \ldots, 200\}$ магутнасці $|S| = 101$ змяшчае прынамсі два ўзаемна простыя лікі $x$ и $y$, то-бок $\text{НАД}(x, y) = 1$.}
{Докажите, что любое подмножество $S \subset \{1, 2, 3, \ldots, 200\}$ мощности $|S| = 101$ содержит по крайней мере два взаимно простых числа $x$ и $y$, т.~е. $\text{НОД}(x, y) = 1$.}

\bigskip

\problemItemSimple
{Дакажыце, што любое падмноства $S \subset \{1, 2, 3, \ldots, 200\}$ магутнасці $|S| = 101$ змяшчае прынамсі два такія элементы $x$ і $y$, што альбо $x \mid y$, альбо $y \mid x$.}
{Докажите, что любое подмножество $S \subset \{1, 2, 3, \ldots, 200\}$ мощности $|S| = 101$ содержит по крайней мере два таких элемента $x$ и $y$, что либо $x \mid y$, либо $y \mid x$.}

\bigskip

\problemItemSimple
{Няхай $m$ "--- адвольны няцотны натуральны лік. Дакажыце, што існуе такі натуральны лік $n$, што $m \mid 2^n - 1$.}
{Пусть $m$ "--- произвольное нечетное натуральное число. Докажите, что существует такое натуральное число $n$, что $m \mid 2^n - 1$.}

\bigskip

\problemItemSimple
{Кожны дзень на працягу чатырохтыднёвага адпачынку чалавек гуляў прынамсі адну партыю ў шахматы, агульны лік згуляных партый не перавышае 40. Дакажыце, што знойдзецца прамежак часу, які складаецца з паслядоўных дзён, на працягу якіх было згуляна роўна 15 партый.}
{Каждый день на протяжении четырехнедельного отпуска отдыхающий играл
по крайней мере одну партию в шахматы. Общее число сыгранных партий не
превышает 40. Докажите, что найдется промежуток времени, состоящий из
последовательных дней, в течение которых было сыграно ровно 15 партий.}

\end{problemList}

\end{document}
