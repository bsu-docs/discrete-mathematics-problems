\documentclass[12pt, a4paper]{article}

\usepackage{import}
\subimport{../../common/}{preamble}

\begin{document}

\biLangHeader
{14. Рэкурэнтныя судачыненні.}
{Рекуррентные соотношения.}

\begin{problemList}

\problemItemSimple
{%
Знайдзіце колькасць падмностваў мноства $\{1, 2, \ldots, 11\}$, якія не змяшчаюць:
\begin{belarusianEnumerate}
    \item ніякіх двух паслядоўных лікаў;
    \item ніякіх трох паслядоўных лікаў.
\end{belarusianEnumerate}}
{%
Найдите число подмножеств множества $\{1, 2, \ldots, 11\}$, которые не содержат:
\begin{russianEnumerate}
    \item никаких двух подряд идущих чисел;
    \item никаких трёх подряд идущих чисел.
\end{russianEnumerate}}

\bigskip

\problemItemSimple
{Знайдзіце колькасць спосабаў, якімі можна наклеіць на паштовы канверт маркі на~агульную
суму 40 капеек, калі можна выкарыстоўваць маркі коштам 5, 10, 15 альбо 20 капеек,
пры гэтым размясціўшы іх у адну лінію (парадак размяшчэння марак "--- істотны).}
{Найдите число способов, которыми можно наклеить на почтовый конверт
марки на сумму 40 копеек, используя марки стоимостью в 5, 10, 15 и 20 копеек,
при этом расположив их в одну линию (порядок расположения марок учитывается).}

\bigskip

\problemItemSimple
{\emph{Лікі Фібаначы} $F(n)$, $n \ge 1$, вызначаюцца формуламі
$F(1) = 1$, $F(2) = 1$, $F(n) = F(n - 1) + F(n - 2)$ для $n \ge 3$.
Выразіце праз лікі Фібаначы колькасць паслядоўнасцяў $(x_1, x_2, \ldots, x_n)$,
якія складаюцца з нулёў і адзінак, для якіх: \\
$x_1 \le x_2 \ge x_3 \le x_4 \ge x_5 \le \ldots\,\,$.}
{\emph{Числа Фибоначчи} $F(n)$, $n \ge 1$, определяются формулами
$F(1) = 1$, $F(2) = 1$, $F(n) = F(n - 1)~+~F(n - 2)$ при $n \ge 3$. Выразите
через числа Фибоначчи число последовательностей $(x_1, x_2, \ldots, x_n)$,
состоящих из нулей и единиц, для которых: \\
$x_1 \le x_2 \ge x_3 \le x_4 \ge x_5 \le \ldots\,\,$.}

\bigskip

\problemItemSimple
{Флаг змяшчае $n \ge 1$ гарызантальных палосак. Кожная палоска расфарбаваная ў~адзін з трох колераў (чырвоны, белы, жоўты).
Ніякія дзве суседнія палоскі не расфарбаваныя ў адзін колер, а таксама крайнія палоскі маюць адрозныя колеры.
Знайдзіце колькасць усіх магчымых сцягоў.}
{Флаг состоит из $n \ge 1$ горизонтальных полосок. Каждая полоска
имеет один из трёх цветов (красный, белый, жёлтый). Никакие две соседние полоски
не имеют одинакового цвета, а также крайние полоски окрашены в разные цвета.
Найдите число возможных флагов.}

\medskip

\problemItemSimple
{Сярод усіх паслядоўнасцяў $(x_1, x_2, \ldots, x_n)$, якія складаюцца з лікаў
$0, 1, 2, 3$, знайдзіце колькасць паслядоўнасцяў, якія змяшчаюць цотную колькасць нулёў.}
{Рассматриваются все последовательности $(x_1, x_2, \ldots, x_n)$,
состоящие из чисел $0, 1, 2, 3$. Найдите число последовательностей,
содержащих чётное число нулей.}

\medskip

\problemItemSimple
{Вызначыце колькасць пароляў даўжыні 7, якія складаюцца з літар англійскага алфавіта
$\{a, b, c, \ldots, z\}$, у якіх літары $a$, $b$ і $c$ не стаяць побач у любым парадку.
Літары ў паролі могуць паўтарацца.}
{Определите число паролей длины 7, состоящих из букв английского
алфавита $\{a, b, c, \ldots, z\}$, в которых буквы $a$, $b$ и $c$ не
стоят рядом в любом порядке. Буквы в пароле могут повторяться.}

\medskip

\problemItemWithCommonPart
{Знайдзіце агульнае рашэнне лінейных аднародных рэкурэнтных судачыненняў з пастаяннымі каэфіцыентамі:}
{Найдите общее решение линейных однородных рекуррентных соотношений с
постоянными коэффициентами:}
{%
\begin{belarusianEnumerate}
    \item $f(n + 3) - 9f(n + 2) + 26f(n + 1) - 24f(n) = 0$, $n = 1, 2, \ldots\, $;
    \item $f(n + 3) - 4f(n + 2) - 3f(n + 1) + 18f(n) = 0$, $n = 1, 2, \ldots\, $;
    \item $f(n + 4) - 2f(n + 3) + 5f(n + 2) - 8f(n + 1) + 4f(n) = 0$, $n = 1, 2, \ldots\,\, $.
\end{belarusianEnumerate}
}

\end{problemList}

\end{document}
