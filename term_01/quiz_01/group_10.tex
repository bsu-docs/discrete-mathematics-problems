\documentclass[12pt, a4paper]{article}

\usepackage{import}
\subimport{../../common/}{preamble}

\begin{document}

\quizTitle
{Логіка~выказванняў.~Мноствы.~Адлюстраванні.}
{Логика высказываний.~Множества.~Отображения.}
{1}

\begin{problemList}

\problemItemSimple
{Што можна сказаць наконт праўдзівасці выказвання  $C \to (A \to B)$, калі вядома, што~$A~\to~B = \mbox{П}$?}
{Что можно сказать об истинностном значении высказывания $C \to (A \to B)$, если известно, что~$A~\to~B = \mbox{И}$?}

\bigskip

\problemItemSimple
{Дакажыце раўназначнасць логікі выказванняў $A \cdot B \cdot \overline{C} \vee A \cdot B \vee \overline{A} \cdot B \equiv B$,
карыстаючыся раўназначнымі пераўтварэннямі.}
{Докажите равносильность логики высказываний $A \cdot B \cdot \overline{C} \vee A \cdot B \vee \overline{A} \cdot B \equiv B$,
используя равносильные преобразования.}

\bigskip

\problemItemWithCommonPart
{Рашыце сістэму лагічных раўнанняў:}
{Решите систему логических уравнений:}
{\[\begin{cases} A \vee \overline{B} = \mbox{П}; \\ (\overline{A} \to B) \cdot A = \mbox{Н}. \end{cases}\]}

\smallskip

\problemItemSimple
{Разглядзім формулы логікі выказванняў $A \cdot B$, $A \to \overline{B}$, $\overline{A \cdot B}$, $A \to B$, $\overline{A \to B}$.
Знайдзіце ўсе пары формул такія, што другая ў пары формула з'яўляецца лагічным вынікам першай формулы.}
{Рассмотрим формулы логики высказываний $A \cdot B$, $A \to \overline{B}$, $\overline{A \cdot B}$, $A \to B$, $\overline{A \to B}$.
Найдите все пары формул такие, что вторая в паре формула является логическим следствием первой формулы.}

\bigskip

\problemItemSimple
{<<Гэта, канешне, Сава. Ці я не Віні-Пух. А я "--- ён \dots>> (<<Віні-Пух і ўсе-ўсе-ўсе>>, А.~Мілн).
Ці вынікае адсюль, што <<калі гэта не Сава, то я не Віні-Пух>>?}
{<<Это, конечно, Сова. Или я не Винни-Пух. А я "--- он $\ldots$>> (А. Милн, <<Винни-Пух и все-все-все>>).
Следует ли отсюда, что <<если это не Сова, то я не Винни-Пух>>?}

\bigskip

\problemItemSimple
{На суразмоўе прыйшлі 65 школьнікаў. Ім прапанавалі 3 кантрольныя работы. За кожную кантрольную работу ставілася адна з адзнак:
2, 3, 4 або 5. Ці праўда, што знойдуцца два школьнікі, якія атрымалі аднолькавыя адзнакі на ўсіх кантрольных?}
{На собеседование пришли 65 школьников. Им предложили 3 контрольные работы. За каждую контрольную работу ставилась одна из оценок:
2, 3, 4 или 5. Верно ли, что найдутся два школьника, получившие одинаковые оценки на всех контрольных?}

\bigskip

\problemItemSimple
{Даследуйце адлюстраванне $f: [0, 1] \mapsto [0, 1]$, $f(x) = \sqrt{1 - x^2}$ на ін'ектыўнасць, сюр'ектыўнасць і біектыўнасць.}
{Исследуйте отображение $f: [0, 1] \mapsto [0, 1]$, $f(x) = \sqrt{1 - x^2}$ на инъективность, сюръективность и биективность.}

\bigskip

\problemItemSimple
{Дакажыце, што $A \cap B \subseteq C$ тады і толькі тады, калі $A \subseteq \overline{B} \cup C$.}
{Докажите, что $A \cap B \subseteq C$ тогда и только тогда, когда $A \subseteq \overline{B} \cup C$.}

\end{problemList}

\newpage

\quizTitle
{Логіка~выказванняў.~Мноствы.~Адлюстраванні.}
{Логика высказываний.~Множества.~Отображения.}
{2}

\begin{problemList}

\problemItemSimple
{Што можна сказаць наконт праўдзівасці выказвання  $\overline{A \to B} \to B$, калі вядома, што~$A~\to~B = \mbox{П}$?}
{Что можно сказать об истинностном значении высказывания $\overline{A \to B} \to B$, если известно, что~$A~\to~B = \mbox{И}$?}

\bigskip

\problemItemSimple
{Дакажыце раўназначнасць логікі выказванняў $\overline{\overline{A} \cdot \overline{B}} \vee ((A \to B) \cdot A) \equiv A \vee B$,
карыстаючыся раўназначнымі пераўтварэннямі.}
{Докажите равносильность логики высказываний $\overline{\overline{A} \cdot \overline{B}} \vee ((A \to B) \cdot A) \equiv A \vee B$,
используя равносильные преобразования.}

\bigskip

\problemItemWithCommonPart
{Рашыце сістэму лагічных раўнанняў:}
{Решите систему логических уравнений:}
{\[\begin{cases} A \to \overline{B} = \mbox{П}; \\ A \cdot B = \mbox{Н}. \end{cases}\]}

\smallskip

\problemItemSimple
{Разглядзім формулы логікі выказванняў $A \vee B$, $\overline{A} \sim B$ и $\overline{A} \cdot B$.
Знайдзіце ўсе пары формул такія, што другая ў пары формула з'яўляецца лагічным вынікам першай формулы.}
{Рассмотрим формулы логики высказываний $A \vee B$, $\overline{A} \sim B$ и $\overline{A} \cdot B$.
Найдите все пары формул такие, что вторая в паре формула является логическим следствием первой формулы.}

\bigskip

\problemItemSimple
{<<Гэта, канешне, Сава. Ці я не Віні-Пух. А я "--- ён \dots>> (<<Віні-Пух і ўсе-ўсе-ўсе>>, А.~Мілн).
Ці вынікае адсюль, што <<калі гэта не Сава, то я не Віні-Пух>>?}
{<<Это, конечно, Сова. Или я не Винни-Пух. А я "--- он $\ldots$>> (А. Милн, <<Винни-Пух и все-все-все>>).
Следует ли отсюда, что <<если это не Сова, то я не Винни-Пух>>?}

\bigskip

\problemItemSimple
{На працягу семестра 28 студэнтаў напісалі тры кантрольныя работы па дыскрэтнай матэматыцы і матэматычнай логіцы,
кожная з якіх ацэньвалася ў 7, 8 або 9 балаў.
Ці праўда, што знойдуцца два студэнты, якія атрымалі аднолькавыя адзнакі на ўсіх кантрольных работах?}
{В течение семестра 28 студентов написали три контрольные работы по дискретной математике и математической логике,
каждая из которых была оценена в 7, 8 или 9 баллов.
Верно ли, что найдутся два студента, которые получили одинаковые оценки на всех контрольных работах?}

\bigskip

\problemItemSimple
{Даследуйце адлюстраванне $f:\  ]0, 1[\ \mapsto \mathbbmss{R}$, $f(x) = \frac{2 x - 1}{2 x - 2 x^2}$
на ін'ектыўнасць, сюр'ектыўнасць і біектыўнасць.}
{Исследуйте отображение $f:\  ]0, 1[\ \mapsto \mathbbmss{R}$, $f(x) = \frac{2 x - 1}{2 x - 2 x^2}$
на инъективность, сюръективность и биективность.}

\bigskip

\problemItemSimple
{Дакажыце, што $A \subseteq B \cup C$ тады і толькі тады, калі $A \cap \overline{B} \subseteq C$.}
{Докажите, что $A \subseteq B \cup C$ тогда и только тогда, когда $A \cap \overline{B} \subseteq C$.}

\end{problemList}

\end{document}
