\documentclass[12pt, a4paper]{article}

\usepackage{import}
\subimport{../../common/}{preamble}

\begin{document}

\biLangHeader
{3. Замкнёнасць і паўната. Асноўныя класы замкнёных функцый.}
{Замкнутость и полнота. Основные классы замкнутых функций.}

\begin{problemList}

\problemItemWithCommonPart
{Высветліце, ці з'яўляецца функцыя $f$ самадваістай:}
{Выясните, является ли функция $f$ самодвойственной:}
{%
\begin{belarusianEnumerateMulticol}
    \item $f=x_1\vee x_2$;
    \item $f=x_1x_2\vee x_2x_3\vee x_3x_1$;
    \item $f=x_1\oplus x_2\oplus x_3\oplus 1$;
    \item $f=(x\vee \overline{y}\vee z)t\vee x\overline{y}z$.
\end{belarusianEnumerateMulticol} 
}

\smallskip

\problemItemWithCommonPart
{Высветліце, ці з'яўляецца самадваістай функцыя $f$, зададзеная вектарна:}
{Выясните, является ли самодвойственной функция $f$, заданная векторно:}
{%
\begin{belarusianEnumerateMulticol}
    \item $\underline{w}(f)=(1010)$;
    \item $\underline{w}(f)=(1001)$;
    \item $\underline{w}(f)=(10110100)$;
    \item $\underline{w}(f)=(10110010)$.
\end{belarusianEnumerateMulticol}
}

\smallskip

\problemItemWithCommonPart
{Вызначыце, якія са зменных функцыі $f(\tilde x^n)$ трэба замяніць на $x$,
а якія на $\overline{x}$, каб атрымаць канстанту:}
{Определите, какие из переменных функции $f(\tilde x^n)$ следует заменить на $x$,
а какие на $\overline{x}$, чтобы получить константу:}
{%
\begin{belarusianEnumerateMulticol}
    \item $\underline{w}(f)=(01101101)$;
    \item $\underline{w}(f)=(10110110)$;
    \item $\underline{w}(f)=(11011000)$;
    \item $\underline{w}(f)=(10100100)$.
\end{belarusianEnumerateMulticol}
}

\smallskip

\problemItemWithCommonPart
{Функцыя $f$ прадстаўленая паліномам Жэгалкіна, высветліце, ці з'яўляецца яна лінейнай:}
{Представив функцию $f$ полиномом Жегалкина, выясните, является ли она линейной:}
{%
\begin{belarusianEnumerateMulticol}
    \item $f=x\rightarrow y$;
    \item $f=\overline{x\rightarrow y}\oplus\overline{x}y$;
    \item $f=x\overline{y}(x\sim y)$;
    \item $f=xy\vee\overline{x}\overline{y}\vee z$.
\end{belarusianEnumerateMulticol}
}

\problemItemWithCommonPart
{Высветліце, ці з'яўляецца лінейнай функцыя $f$, зададзеная вектарна:}
{Выясните, является ли линейной функция $f$, заданная векторно:}
{%
\begin{belarusianEnumerateMulticol}
    \item $\underline{w}(f)=(1001)$;
    \item $\underline{w}(f)=(1101)$;
    \item $\underline{w}(f)=(10010110)$;
    \item $\underline{w}(f)=(1001011010010110)$.
\end{belarusianEnumerateMulticol}
}

\smallskip

\problemItemWithCommonPart
{Падставіўшы на месцы зменных нелінейнай функцыі $f$ функцыі з мноства $\{0, 1, x, y\}$,
атрымайце хаця б адну з функцый $xy$, $x\overline{y}$, $\overline{xy}$:}
{Подставляя на места переменных нелинейной функции $f$ функции из множества $\{0, 1, x, y\}$,
получите хотя бы одну из функций $xy$, $x\overline{y}$, $\overline{xy}$:}
{%
\begin{belarusianEnumerateMulticol}
    \item $\underline{w}(f)=(01100111)$;
    \item $\underline{w}(f)=(11010101)$;
    \item $f=x_1x_2\vee x_2\overline{x}_3\vee \overline{x}_3x_1$;
    \item $f=(x_1\vee x_2\vee x_3)(\overline{x}_1\vee \overline{x}_2\vee \overline{x}_3\vee x_4)$.
\end{belarusianEnumerateMulticol}
}

\smallskip

\problemItemWithCommonPart
{Высветліце, ці належыць функцыя $f$ мноству $T_1\setminus T_0$:}
{Выясните, принадлежит ли функция $f$ множеству $T_1\setminus T_0$:}
{%
\begin{belarusianEnumerateMulticol}
    \item $\underline{w}(f)=(10010110)$;
    \item $\underline{w}(f)=(11011001)$;
    \item $f=(x_1\rightarrow x_2)(x_2\rightarrow x_3)(x_3 \rightarrow x_1)$;
    \item $f=x_1\rightarrow (x_2 \rightarrow (x_3 \rightarrow x_1))$.
\end{belarusianEnumerateMulticol}
}

\smallskip

\problemItemWithCommonPart
{Падлічыце колькасць функцый, якія залежаць ад зменных $x_1, x_2, \ldots, x_n$ і належаць мноству~$A$:}
{Подсчитайте число функций, зависящих от переменных $x_1, x_2, \ldots, x_n$ и принадлежащих множеству~$A$:}
{%
\begin{belarusianEnumerateMulticol}
    \item $A=(L\setminus T_0)\cap S$;
    \item $A=T_0\cap L$;
    \item $A=(S\cap L)\setminus T_1$;
    \item $A=S\setminus(T_0\cup T_1)$;
    \item $A=(L\setminus S)\cap T_1$;
    \item $A=T_0 \cup T_1 \cup S$.
\end{belarusianEnumerateMulticol}
}

\smallskip

\problemItemWithCommonPart
{Па вектары значэння $\tilde\alpha_f$ высветліце, ці з'яўляецца функцыя $f$ манатоннай:}
{По вектору значений $\tilde\alpha_f$ выясните, является ли функция $f$ монотонной:}
{%
\begin{belarusianEnumerateMulticol}
    \item $\underline{w}(f)=(0110)$;
    \item $\underline{w}(f)=(10011111)$;
    \item $\underline{w}(f)=(00110111)$;
    \item $\underline{w}(f)=(01010111)$.
\end{belarusianEnumerateMulticol}
}

\smallskip

\problemItemWithCommonPart
{Праверце, ці з'яўляецца функцыя $f$ манатоннай:}
{Проверьте, является ли функция $f$ монотонной:}
{%
\begin{belarusianEnumerate}
    \item $f=x\vee \overline{x}y\vee\overline{x}z\overline{y}$;
    \item $f=(x_1\oplus x_2)\cdot(x_1\sim x_2)$;
    \item $f=x_1\rightarrow (x_2 \rightarrow x_1)$;
    \item $f=x_1\overline{x}_2\overline{x}_3\vee x_1\overline{x}_2x_3\vee x_1x_2\overline{x}_3\vee x_1x_2x_3\vee \overline{x}_1x_2x_3$.
\end{belarusianEnumerate}
}

\smallskip

\problemItemWithCommonPart
{Для неманатоннай функцыі $f$ укажыце два суседнія наборы $\tilde\alpha$ і $\tilde\beta$ значэнняў зменных,
такіх, што $\tilde\alpha < \tilde\beta$ і $f(\tilde\alpha) > f(\tilde\beta)$:}
{Для немонотонной функции $f$ укажите два соседних набора $\tilde\alpha$ и $\tilde\beta$ значений переменных,
таких, что $\tilde\alpha < \tilde\beta$ и $f(\tilde\alpha) > f(\tilde\beta)$:}
{%
\begin{belarusianEnumerateMulticol}
    \item $f=x_1x_2x_3\vee \overline{x}_1x_2$;
    \item $f=x_1\oplus x_2\oplus x_3$;
    \item $f=x_1x_2\oplus x_3$;
    \item $f=x_1\vee x_2\overline{x}_3$.
\end{belarusianEnumerateMulticol}
}

\smallskip

\problemItemWithCommonPart
{Падлічыце колькасць булевых функцый ад $n$ зменных у кожным з наступных мностваў:}
{Подсчитайте число булевых функций от $n$ переменных в каждом из следующих множеств:}
{%
\begin{belarusianEnumerateMulticol}
    \item $M\setminus(T_1 \cup T_0)$;
    \item $M \cap L$.
\end{belarusianEnumerateMulticol}
}

\smallskip

\problemItemSimple[*]
{Ці справядліва, што колькасць самадваістых функцый, якія істотна залежаць ад зменных $x_1, \ldots, x_n$,
роўная колькасці функцый з $P_2^{n-1}$, якія істотна залежаць ад усіх сваіх зменных?}
{Верно ли, что число самодвойственных функций, существенно зависящих от переменных $x_1, \ldots, x_n$,
равно числу функций из $P_2^{n-1}$, существенно зависящих от всех своих переменных?}

\bigskip

\problemItemSimple[*]
{Дакажыце, што лінейная функцыя з'яўляецца самадваістай тады і толькі тады,
калі яна істотна залежыць ад няцотнай колькасці зменных.}
{Докажите, что линейная функция является самодвойственной тогда и только тогда,
когда она существенно зависит от нечётного числа переменных.}

\bigskip

\problemItemSimple[*]
{Знайдзіце колькасць лінейных функцый $f(\tilde x^n)$, якія істотна залежаць роўна ад $k$ зменных.}
{Найдите число линейных функций $f(\tilde x^n)$, существенно зависящих в точности от $k$ переменных.}

\bigskip

\problemItemSimple[*]
{Пакажыце, што $M$ не змяшчаецца ні ў адным з класаў $T_0$, $T_1$, $S$, $L$,
для гэтага прывядзіце манатонныя функцыі, якія не змяшчаюцца ў адпаведных класах.}
{Покажите, что $M$ не содержится ни в одном из классов $T_0$, $T_1$, $S$, $L$,
указав монотонные функции, не содержащиеся в соответствующих классах.}

\bigskip

\problemItemSimple[*]
{Падлічыце колькасць шэферавых функцый у $P_2^n$.}
{Подсчитайте число шефферовых функций в $P_2^n$.}

\end{problemList}

\end{document}