\documentclass[12pt, a4paper]{article}

\usepackage{import}
\subimport{../../common/}{preamble}

\begin{document}

\biLangHeader
{4. Тэарэма Поста. Мінімізацыя булевый функцый.}
{Теорема Поста. Минимизация булевых функций.}

\biLangHeader
{Тэарэма Поста.}
{Теорема Поста.}

\begin{problemList}
    
\problemItemWithCommonPart
{Высветліце, ці з'яўляецца поўнай сістэма функцый:}
{Выясните, полна ли система функций:}
{%
\begin{belarusianEnumerateTwocol}
    \item $A=\{xy\oplus z, x\oplus y\oplus 1\}$;
    \item $A=\{xy, x\vee y, x\oplus y, xy\vee yz\vee zx\}$;
    \item $A=\{xy, x\vee y, x\oplus y\oplus z\oplus 1\}$;
    \item $A=\{1, \overline {x}, x(y\sim z)\oplus\overline{x}(y\oplus z), x\sim y\}$.
\end{belarusianEnumerateTwocol}
}

\smallskip

\problemItemWithCommonPart
{Высветліце, ці з'яўляецца поўнай сістэма $A$ функцый, зададзеных вектарамі сваіх значэнняў:}
{Выясните, полна ли система $A$ функций, заданных векторами своих значений:}
{%
\begin{belarusianEnumerate}
    \item $A=\{f_1=(0110),\ f_2=(11000011),\ f_3=(10010110)\}$;
    \item $A=\{f_1=(0111),\ f_2=(01011010),\ f_3=(01111110)\}$;
    \item $A=\{f_1=(0111),\ f_2=(10010110)\}$;
    \item $A=\{f_1=(0101),\ f_2=(11101000),\ f_3=(01101001)\}$.
\end{belarusianEnumerate}
}

\smallskip

\problemItemWithCommonPart
{Высветліце, ці з'яўляецца сістэма $A$ поўнай:}
{Выясните, полна ли система $A$:}
{%
\begin{belarusianEnumerateTwocol}
    \item $A=(S\setminus M)\cup L\setminus(T_0\cup T_1)$;
    \item $A=(S\cap M)\cup(L\setminus M)$;
    \item $A=(L\cap T_1\cap T_0)\cup S\setminus(T_0\cup T_1)$;
    \item $A=(L\cap T_1)\cup(S\cap M)$.
\end{belarusianEnumerateTwocol}
}

\smallskip

\problemItemWithCommonPart[*]
{Высветліце, ці з'яўляецца поўнай сістэма функцый $A=\{f_1, f_2\}$:}
{Выясните, полна ли система функций $A=\{f_1, f_2\}$:}
{%
\begin{belarusianEnumerate}
    \item $f_1\in S\setminus M$, $f_2\not\in L\cup S$, $f_1 \rightarrow f_2 \equiv 1$;
    \item $f_1\not\in L \cup T_0 \cup T_1$, $f_2\in M\cap L$, $f_1 \rightarrow f_2 \equiv 1$.
\end{belarusianEnumerate}
}

\end{problemList}

\biLangHeader
{Мінімізацыя булевый функцый.}
{Минимизация булевых функций.}

\begin{problemList}

\problemItemWithCommonPart
{З зададзенага мноства $A$ элементарных кан'юнкцый вылучыце простыя імпліканты функцыі~$f$:}
{Из заданного множества $A$ элементарных конъюнкций выделите простые импликанты функции~$f$:}
{%
\begin{belarusianEnumerate}
    \item $A=\{x_1, \overline{x}_3, x_1x_2, x_2\overline{x}_3\}$, $\underline{w}(f)=(00101111)$;
    \item $A=\{x_1\overline{x}_2, x_2x_3, x_1x_2x_3\}$, $\underline{w}(f)=(01111110)$;
    \item $A=\{x_1, \overline{x}_4, x_2\overline{x}_3, \overline{x}_1\overline{x}_2\overline{x}_4\}$, $\underline{w}(f)=(1010111001011110)$;
    \item $A=\{x_1, x_2, x_1\overline{x}_2\}$, $\underline{w}(f)=(1011)$.
\end{belarusianEnumerate}
}

\smallskip

\problemItemWithCommonPart
{З дапамогай алгарытма Квайна пабудуйце скарочаную ДНФ для функцыі $f$,
зададзенай вектарам сваіх значэнняў:}
{С помощью алгоритма Квайна постройте сокращённую ДНФ для функции $f$, 
заданной вектором своих значений:}
{%
\begin{belarusianEnumerateTwocol}
    \item $\underline{w}(f)=(00011111)$;
    \item $\underline{w}(f)=(01110110)$;
    \item $\underline{w}(f)=(10111101)$;
    \item $\underline{w}(f)=(00101111)$.
\end{belarusianEnumerateTwocol}
}

\smallskip

\problemItemWithCommonPart
{Па зададзенай ДНФ $D$ з дапамогай метада Блэйка пабудуйце скарочаную ДНФ:}
{По заданной ДНФ $D$ с помощью метода Блейка постройте сокращённую ДНФ:}
{%
\begin{belarusianEnumerateTwocol}
    \item $D=x_1x_2\vee \overline{x}_1x_3\vee \overline{x}_2x_3$;
    \item $D=\overline{x}_1\overline{x}_2\vee x_1\overline{x}_2x_4\vee x_2\overline{x}_3x_4$;
    \item $D=x_1\overline{x}_2x_3\vee \overline{x}_1x_2\overline{x}_4\vee \overline{x}_2\overline{x}_3x_4$;
    \item $D=x_1\vee\overline{x}_1x_2\vee \overline{x}_1\overline{x}_2x_3\vee \overline{x}_1x_2x_3x_4$.
\end{belarusianEnumerateTwocol}
}

\smallskip

\problemItemWithCommonPart[*]
{Знайдзіце даўжыню скарочанай ДНФ функцыі $f$:}
{Найдите длину сокращённой ДНФ функции $f$:}
{%
\begin{belarusianEnumerate}
    \item $f=x_1\oplus x_2\oplus \ldots \oplus x_n$;
    \item $f=(x_1\vee x_2)(x_3\vee x_4)\ldots(x_{2n-1}\vee x_{2n})$.
\end{belarusianEnumerate}
}

\smallskip

\problemItemSimple[*]
{Пакажыце, што простая імпліканта манатоннай функцыі не змяшчае адмаўленняў зменных.}
{Покажите, что простая импликанта монотонной функции не содержит отрицаний переменных.}

\end{problemList}

\end{document}