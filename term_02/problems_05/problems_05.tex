\documentclass[12pt, a4paper]{article}

\usepackage{import}
\subimport{../../common/}{preamble}
\graphicspath{{figures/}}

\begin{document}

\biLangHeader
{5. Азбука тэорыі графаў.}
{Азбука теории графов.}

\begin{problemList}

\problemItemSimple
{Колькі рэбраў у поўным графе $K_n$?}
{Сколько рёбер в полном графе $K_n$?}

\bigskip

\problemItemWithCommonPart
{Знайдзіце акружэнні $N(u)$ і $N(v)$ вяршынь $u$ і $v$ у кожным з трох графаў, паказаных на малюнку:}
{Найдите окружения $N(u)$ и $N(v)$ вершин $u$ и $v$ в каждом из трёх графов, изображённых на рисунке:}
{%
\begin{center}
    \includegraphics[scale=0.7]{figure_01}
\end{center}
}

\problemItemSimple
{Дакажыце, што найбольшая колькасць рэбраў у графе парадку $n$, які не змяшчае трохкутнікаў, роўная $n^2/4$.}
{Докажите, что наибольшее число рёбер в графе порядка $n$, не содержащем треугольников, равно $n^2/4$.}

\bigskip

\problemItemSimple
{Знайдзіце ўсе двудольныя графы парадку $n$, колькасць рэбраў якіх роўная $n^2/4$.}
{Найдите все двудольные графы порядка $n$, число рёбер которых равно $n^2/4$.}

\bigskip

\problemItemWithCommonPart
{Дакажыце, што два графы, паказаныя на малюнку, ізаморфныя:}
{Докажите, что два графа, изображённые на рисунке, изоморфны:}
{%
\begin{center}
    \includegraphics[scale=0.7]{figure_02}
\end{center}
}

\problemItemWithCommonPart
{Дакажыце, што графы, паказаныя на малюнку, не ізаморфныя:}
{Докажите, что графы, изображённые на рисунке, не изоморфны:}
{%
\begin{center}
    \includegraphics[scale=0.7]{figure_03}
\end{center}
}

\problemItemWithCommonPart
{Ці існуе граф з дадзенай ступеннай паслядоўнасцю?
Калі так, то знайдзіце колькасць рэбраў у графе,
ступенная паслядоўнасць якога супадае з адной з наступных:}
{Существует ли граф с указанной степенной последовательностью?
Если да, то найдите число рёбер в графе,
степенная последовательность которого совпадает с одной из следующих:}
{%
\begin{belarusianEnumerateTwocol}
    \item (3, 3, 3, 3, 3, 3);
    \item (3, 3, 2, 2, 2, 2);
    \item (6, 5, 4, 4, 3, 3, 2, 2);
    \item (7, 6, 5, 4, 4, 3, 2, 1).
\end{belarusianEnumerateTwocol}
}

\smallskip

\problemItemWithCommonPartComplicated
{Знайдзіце дадатковы граф $\overline{G}$ для кожнага з графаў $G$:}
{Найдите дополнительный граф $\overline{G}$ для каждого из графов $G$:}
{%
\begin{belarusianEnumerateTwocol}
    \item $G=O_n$;
    \item $G=K_n$;
    \item $G=C_4$;
    \item $G=P_4$;
\end{belarusianEnumerateTwocol}
}
{Якія з прыведзеных графаў з'яўляюцца самададатковымі?}
{Какие из указанных графов являются самодополнительными?}

\bigskip

\problemItemSimple
{%
\begin{belarusianEnumerate}
    \item Для $(n,m)$-графа $G$ знайдзіце колькасць рэбраў у самададатковым графе $\overline{G}$.
    Колькі рэбраў у самададатковым графе парадку $n$?
    \item Дакажыце, што калі $n$ "--- парадак самададатковага графа,
    то $n\equiv 0\pmod 4$ ці $n \equiv 1 \pmod 4$;
    \item Дакажыце, што для кожнага $n\ge 1$, які адпавядае ўмове $n\equiv 0\pmod 4$
    ці $n \equiv 1 \pmod 4$, існуе самададатковы граф парадку $n$.
\end{belarusianEnumerate}\\[-30pt]
}
{%
\begin{russianEnumerate}
    \item Для $(n,m)$-графа $G$ найдите число рёбер в доплонительном графе $\overline{G}$.
    Сколько рёбер в самодополнительном графе порядка $n$?
    \item Докажите, что если $n$ "--- порядок самодополнительного графа, 
    то $n\equiv 0\pmod 4$ или $n \equiv 1 \pmod 4$;
    \item Докажите, что для каждого $n\ge 1$, удовлетворяющего условию $n\equiv 0\pmod 4$ 
    или $n \equiv 1 \pmod 4$, существует самодополнительный граф порядка $n$.
\end{russianEnumerate}
}

\bigskip

\problemItemSimple
{Пабудуйце граф $C_4\times P_3$.}
{Постройте граф $C_4\times P_3$.}

\bigskip

\problemItemSimple
{Няхай $d_1, d_2, \ldots, d_n$ "--- ступенная паслядоўнасць графа $G$,
а $d_1', d_2', \ldots, d_n'$ "--- ступенная паслядоўнасць графа $G'$.
Якая ступенная паслядоўнасць у графа $G\times G'$?}
{Пусть $d_1, d_2, \ldots, d_n$ "--- степенная последовательность графа $G$,
а $d_1', d_2', \ldots, d_n'$ "--- степенная последовательность графа $G'$.
Какая степенная последовательность у графа $G\times G'$?}

\bigskip

\problemItemSimple
{Знайдзіце простыя цыклы даўжыні 5, 6, 8 і 9 у графе Петэрсана.}
{Найдите простые циклы длины 5, 6, 8 и 9 в графе Петерсена.}

\bigskip

\problemItemSimple
{Дакажыце, што калі ў графе ёсць роўна дзве вяршыні няцотных ступеняў,
то ў ім ёсць ланцуг, які злучае гэтыя вяршыні.}
{Докажите, что если в графе есть ровно две вершины нечётных степеней,
то в нём есть соединяющая эти вершины цепь.}

\bigskip

\problemItemSimple
{Знайдзіце дыяметр графа $P^2_n$.}
{Найдите диаметр графа $P^2_n$.}

\bigskip

\problemItemWithCommonPart
{Знайдзіце радыюс, дыяметр, цэнтр і перыферыю наступных графаў:}
{Найдите радиус, диаметр, центр и периферию следующих графов:}
{%
\begin{belarusianEnumerateTwocol}
    \item $K_n$;
    \item $P_n$;
    \item $C_n$;
    \item граф Петэрсана.
\end{belarusianEnumerateTwocol}
}

\smallskip

\problemItemWithCommonPart[*]
{Паказаць, што для любога графа $G$ справядліва:}
{Показать, что для любого графа $G$ справедливо:}
{\[ \sum\limits_{\{u, v\}\in E(G)} (\deg u + \deg v) = \sum\limits_{w\in V(G)} \deg^2 w \]}

\medskip

\problemItemWithCommonPart[*]
{Знайдзіце групу аўтамарфізмаў для:}
{Найдите группу автоморфизмов для:}
{%
\begin{belarusianEnumerate}
    \item $O_n$;
    \item $K_n$;
    \item $P_n$;
    \item $C_n$.
\end{belarusianEnumerate}
}

\smallskip

\problemItemWithCommonPart[*]
{Колькі адрозных пазначаных простых цыклаў ёсць у графе $G$, калі}
{Сколько различных помеченных простых циклов имеется в графе $G$, если}
{%
\begin{belarusianEnumerate}
    \item $G=K_n$;
    \item $G=K_{p, q}$.
\end{belarusianEnumerate}
}

\smallskip

\problemItemSimple[*]
{Няхай $G$ і $\overline{G}$ адначасова звязныя і $d(G) \ge 3$.
Дакажыце, што $d(\overline{G}) \le 3$.}
{Пусть $G$ и $\overline{G}$ одновременно связны и $d(G) \ge 3$.
Докажите, что $d(\overline{G}) \le 3$.}

\end{problemList}

\end{document}