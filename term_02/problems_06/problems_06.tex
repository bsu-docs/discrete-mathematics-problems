\documentclass[12pt, a4paper]{article}

\usepackage{import}
\subimport{../../common/}{preamble}
\graphicspath{{figures/}}

\begin{document}

\biLangHeader
{6. Дрэвы: асноўныя ўласцівасці, ступенная паслядоўнасць, код Пруфера.}
{Деревья: основные свойства, степенная последовательность, код Прюфера.}

\begin{problemList}

\problemItemSimple
{Намалюйце ўсе папарна адрозныя пазначаныя дрэвы парадку 4.}
{Нарисуйте все попарно различные помеченные деревья порядка 4.}

\bigskip

\problemItemSimple
{%
Знайдзіце ўсе папарна неізаморфныя:
\begin{belarusianEnumerate}
    \item дрэвы шостага парадку;
    \item дрэвы, якія маюць пяць вяршынь ступені 1 і не змяшчаюць вяршынь ступені 2;
    \item самададатковыя дрэвы;
    \item графы, у якіх кожны спароджаны падграф парадку 3 з'яўляецца дрэвам.
\end{belarusianEnumerate}
}
{%
Найдите все попарно неизоморфные:
\begin{russianEnumerate}
    \item деревья шестого порядка;
    \item деревья, которые имеют пять вершин степени 1 и не содержат вершин степени 2;
    \item самодополнительные деревья;
    \item графы, в которых каждый порождённый подграф порядка 3 является деревом.
\end{russianEnumerate}
}

\bigskip

\problemItemSimple
{Дакажыце, што кожнае дрэва з'яўляецца двудольным графам.
Якія поўныя двудольныя графы з'яўляюцца дрэвамі?}
{Докажите, что каждое дерево является двудольным графом.
Какие полные двудольные графы являются деревьями?}

\bigskip

\problemItemSimple
{Няхай $S$ "--- адвольнае канечнае мноства натуральных лікаў, якое змяшчае 1.
Дакажыце, што існуе дрэва $T$, ступеннае мноства якога супадае з $S$.}
{Пусть $S$ "--- произвольное конечное множество натуральных чисел, содержащее 1.
Докажите, что существует дерево $T$, степенное множество которого совпадает с $S$.}

\bigskip

\problemItemSimple
{Няхай $T$ "--- дрэва парадку $n\ge 2$ і $V(T)=\{v_1, \ldots, v_n\}$. Пакажыце, што колькасць канцавых вяршынь дрэва $T$ вылічваецца па формуле $1+\sum\limits_{i=1}^n |\deg v_i - 2|/2$.}
{Пусть $T$ "--- дерево порядка $n\ge 2$ и $V(T)=\{v_1, \ldots, v_n\}$. Покажите, что число концевых вершин дерева $T$ вычисляется по формуле $1+\sum\limits_{i=1}^n |\deg v_i - 2|/2$.}

\medskip

\problemItemSimple
{Дакажыце, што радыюс $r(T)$ і дыяметр $d(T)$ любога дрэва $T$ звязаныя судачыненнем $r(T)=~\lceil d(T)/2 \rceil$.}
{Докажите, что радиус $r(T)$ и диаметр $d(T)$ любого дерева $T$ связаны соотношением $r(T)=\lceil d(T)/2 \rceil$.}

\bigskip

\problemItemWithCommonPart
{Пабудуйце коды Пруфера для кожнага дрэва, паказанага на малюнку:}
{Постройте коды Прюфера для каждого дерева, изображённого на рисунке:}
{%
\begin{center}
    \includegraphics[scale=0.9]{figure_01}
\end{center}
}

\problemItemWithCommonPart
{Аднавіце дрэва па ягоным кодзе Пруфера $P(T)$:}
{Восстановите дерево по его коду Прюфера $P(T)$:}
{%
\begin{belarusianEnumerate}
    \item $P(T)=(4, 5, 6, 7, 2, 1, 1, 6, 6, 7)$;
    \item $P(T)=(6, 6, 7, 7, 7, 8, 8, 9, 9, 8, 12, 13, 12)$;
    \item $P(T)=(i, i, \ldots, i)$, дзе $i\in\{1, 2, \ldots, n\}=V(T)$.
\end{belarusianEnumerate}
}

\smallskip

\problemItemWithCommonPart
{Знайдзіце ўсе каркасы наступных пазначаных графаў:}
{Найдите все остовы следующих помеченных графов:}
{%
\begin{belarusianEnumerateTwocol}
    \item $K_4$;
    \item $P_6$;
    \item $C_7$;
    \item $K_{1, 3}$.
\end{belarusianEnumerateTwocol}
}

\smallskip

\problemItemSimple
{Дакажыце, што ў любым нетрывіяльным графе ёсць каркас,
пасля выдалення ўсіх кантаў якога граф становіцца нязвязным.}
{Докажите, что в любом нетривиальном связном графе имеется остов,
после удаления всех рёбер которого граф становится несвязным.}

\bigskip

\problemItemSimple
{Ці існуе граф, які змяшчае роўна два каркасы?}
{Существует ли граф, содержащий ровно два остова?}

\bigskip

\problemItemSimple[*]
{Дакажыце, што ў дрэве $T$ усе вяршыні маюць няцотныя ступені тады і толькі тады, калі для кожнага канта $e\in E(T)$ абедзве кампаненты звязнасці графа $T-e$ маюць няцотны парадак.}
{Докажите, что в дереве $T$ все вершины имеют нечётные степени тогда и только тогда, когда для каждого ребра $e\in E(T)$ обе компоненты связности графа $T-e$ имеют нечётный порядок.}

\bigskip

\problemItemSimple[*]
{Няхай $T$ "--- дрэва парадку $k\ge 1$. Пакажыце, што адвольны граф $G$ з мінімальнай ступенню вяршыні не меншай за $k-1$, змяшчае падграф, ізаморфны $T$.}
{Пусть $T$ "--- дерево порядка $k\ge 1$. Покажите, что произвольный граф $G$ с минимальной степенью вершины не меньшей, чем $k-1$, содержит подграф, изоморфный $T$.}

\medskip

\problemItemSimple[*]
{Няхай $(d_1, \ldots, d_n)$ "---  паслядоўнасць дадатных лікаў такая,
што $n\ge 2$ і $\sum\limits_{i=1}^n d_i=2n-2$.
Знайдзіце колькасць пазначаных дрэваў на мностве вяршынь $\{1, 2, \ldots, n\}$, для якіх $\deg i=d_i$, $i=1, \ldots, n$.}
{Пусть $(d_1, \ldots, d_n)$ "--- последовательность положительных чисел такая,
что $n\ge 2$ и $\sum\limits_{i=1}^n d_i=2n-2$.
Найдите число помеченных деревьев на множестве вершин $\{1, 2, \ldots, n\}$, для которых $\deg i=d_i$, $i=1, \ldots, n$.}

\bigskip

\problemItemSimple[*]
{Даказаць, што паслядоўнасць цэлых лікаў $(d_1, d_2, \ldots, d_n)$,
дзе $d_1 \ge d_2 \ge \ldots \ge d_n \ge 1$ и $n \ge 2$,
з'яўляецца ступеннай паслядоўнасцю некаторага дрэва парадку $n$ тады і толькі тады,
калі $\sum\limits_{i = 1}^n d_i = 2n - 2$.}
{Доказать, что последовательность целых чисел $(d_1, d_2, \ldots, d_n)$,
где $d_1 \ge d_2 \ge \ldots \ge d_n \ge 1$ и $n \ge 2$,
является степенной последовательностью некоторого дерева порядка $n$ тогда и только тогда,
когда $\sum\limits_{i = 1}^n d_i = 2n - 2$.}

\medskip

\problemItemSimple[*]
{Няхай $T$ "--- дрэва, $v \in V(T)$ і $x$, $y$ "--- адрозныя вяршыні з акружэння вяршыні $v$.
Дакажыце, што $2e(v) \le e(x) + e(y)$. (Тут $e(u)$ "--- эксцэнтрысітэт вяршыні $u$.)}
{Пусть $T$ "--- дерево, $v \in V(T)$ и $x$, $y$ "--- различные вершины из окружения вершины $v$.
Докажите, что $2e(v) \le e(x) + e(y)$. (Здесь $e(u)$ "--- эксцентриситет вершины $u$.)}

\end{problemList}

\end{document}