\documentclass[12pt, a4paper]{article}

\usepackage{import}
\subimport{../../common/}{preamble}
\graphicspath{{figures/}}

\begin{document}
    
\biLangHeader
{7. Плоскія і планарныя графы: формула Эйлера, ацэнка на колькасць кантаў, крытэр планарнасці (Пантрагіна-Куратоўскага, Вагнэра).}
{Плоские и планарные графы: формула Эйлера, оценка на число рёбер, критерий планарности (Понтрягина-Куратовского, Вагнера).}

\begin{problemList}

\problemItemWithCommonPart
{Пабудуйце плоскія ўкладкі пералічаных ніжэй графаў, у якіх усе канты намаляваныя ў выглядзе адрэзкаў прамых ліній:}
{Постройте плоские укладки перечисленных ниже графов, в которых все рёбра изображены в виде отрезков прямых линий:}
{%
\begin{belarusianEnumerate}
    \item $K_5-e$;
    \item $K_{3, 3}-e$;
    \item $K_{2, n}$.
\end{belarusianEnumerate}
}

\smallskip

\problemItemSimple
{Дакажыце, што ў планарным графе з колькасцю вяршынь у долях $p\ge 2$, $q\ge 2$ колькасць кантаў не перавышае $2(p+q-2)$.}
{Докажите, что в планарном двудольном графе с числом вершин в долях $p\ge 2$, $q\ge 2$ число рёбер не превосходит $2(p+q-2)$.}

\bigskip

\problemItemSimple
{Дакажыце, што кожны планарны граф змяшчае вяршыню ступені не больш за 5.}
{Докажите, что каждый планарный граф содержит вершину степени, не превосходящей 5.}

\bigskip

\problemItemSimple
{Дакажыце, што ў максімальным планарным графе парадку $n\ge 4$ ступень кожнай вяршыні не менш за тры.}
{Докажите, что в максимальном планарном графе порядка $n\ge 4$ степень каждой вершины не менее трёх.}

\bigskip

\problemItemSimple
{Пры якіх $n$ квадрат цыкла $C_n$ з'яўляецца планарным графам?}
{При каких $n$ квадрат цикла $C_n$ является планарным графом?}

\bigskip

\problemItemSimple
{Прывядзіце прыклад звязнага планарнага графа $G$ з мінімальнай ступенню вяршыні не меншай за 4.
Дакажыце, што гэты граф не можа быць двудольным.}
{Приведите пример связного планарного графа $G$ с минимальной степенью вершины не меньшей 4.
Докажите, что такой граф не может быть двудольным.}

\bigskip

\problemItemSimple
{Прывядзіце прыклад такога графа $G$ парадку 8, што $G$ і $\overline{G}$ непланарныя.}
{Приведите пример такого графа $G$ порядка 8, что $G$ и $\overline{G}$ непланарны.}

\bigskip

\problemItemSimple
{$T$ "--- дрэва парадку $n\ge 4$ и $e_1, e_2, e_3\in E(\overline{T})$.
Дакажыце, што граф $T+e_1+e_2+e_3$ планарны.
Ці будзе справядлівым гэта сцвярджэнне, калі тры канты замяніць на чатыры?}
{$T$ "--- дерево порядка $n\ge 4$ и $e_1, e_2, e_3\in E(\overline{T})$.
Докажите, что граф $T+e_1+e_2+e_3$ планарен.
Будет ли верным это утверждение, если три ребра заменить на четыре?}

\bigskip

\problemItemWithCommonPart
{Якія з графаў, паказаных на малюнку, з'яўляюцца планарнымі?}
{Какие из графов, изображённых на рисунке, являются планарными?}
{%
\begin{center}
    \includegraphics[scale=0.85]{figure_01}
\end{center}
}

\problemItemSimple
{Дакажыце, што граф Петэрсана непланарны.}
{Докажите, что граф Петерсена непланарен.}

\bigskip

\problemItemSimple
{Дакажыце, што $n_1+m_2=n_2+m_1$, калі $(n_1, m_1)$-граф і $(n_2, m_2)$-граф гамеаморфныя.}
{Докажите, что $n_1+m_2=n_2+m_1$, если $(n_1, m_1)$-граф и $(n_2, m_2)$-граф гомеоморфны.}

\bigskip

\problemItemSimple[*]
{Якую мінімальную колькасць кантаў трэба выдаліць з графа Петэрсана, каб атрымаўся планарны граф?}
{Какое наименьшее число рёбер нужно удалить из графа Петерсена, чтобы получился планарный граф?}

\bigskip

\problemItemSimple[*]
{Прывядзіце прыклад 3-рэгулярнага планарнага графа, дыяметр якога роўны 3 і парадак роўны 12.
Ці існуюць такія графы парадку больш за 12?}
{Приведите пример 3-регулярного планарного графа, диаметр которого равен 3 и порядок равен 12.
Существуют ли такие графы порядка более 12?}

\bigskip

\problemItemSimple[*]
{Дакажыце, што існуе граф парадку $n \ge 3$ з $m > 3n - 6$ кантамі,
які не змяшчае графаў $K_5$ і $K_{3, 3}$ у якасці падграфаў.}
{Докажите, что существует граф порядка $n \ge 3$ с $m > 3n - 6$ рёбрами,
который не содержит графов $K_5$ и $K_{3, 3}$ в качестве подграфов.}

\bigskip

\problemItemSimple[*]
{Няхай $G$ "--- граф парадку $n \ge 11$.
Дакажыце, што граф $G$ і ягонае дапаўненне $\overline{G}$ не могуць быць адначасова планарнымі.}
{Пусть $G$ "--- граф порядка $n \ge 11$.
Докажите, что граф $G$ и его дополнение $\overline{G}$ не могут быть одновременно планарными.}

\bigskip

\problemItemSimple[*]
{%
Няхай $G$ "--- максімальны планарны граф парадку не менш за 4.
Дакажыце, што для графа $G$ справядлівае хаця б адно з наступных чатырох сцвярджэнняў:
\begin{belarusianEnumerate}
    \item $G$ змяшчае вяршыню ступені 3;
    \item $G$ змяшчае вяршыню ступені 4;
    \item $G$ змяшчае дзве сумежныя вяршыні ступені 5;
    \item $G$  змяшчае дзве сумежныя вяршыні, адна з якіх мае ступень 5, а другая "--- 6.
\end{belarusianEnumerate}
}
{%
Пусть $G$ "--- максимальный планарный граф порядка не меньше 4.
Докажите, что для графа $G$ верно хотя бы одно из следующих четырёх утверждений:
\begin{russianEnumerate}
    \item $G$ содержит вершину степени 3;
    \item $G$ содержит вершину степени 4;
    \item $G$ содержит две смежные вершины степени 5;
    \item $G$ содержит две смежные вершины, одна из которых имеет степень 5, а другая "--- 6.
\end{russianEnumerate}
}

\end{problemList}

\end{document}