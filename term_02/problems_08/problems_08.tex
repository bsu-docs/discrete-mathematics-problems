\documentclass[12pt, a4paper]{article}

\usepackage{import}
\subimport{../../common/}{preamble}
\graphicspath{{figures/}}

\begin{document}

\biLangHeader
{8. Абыходы ў графах: эйлеравыя графы, крытэр эйлеравасці, гамільтанавыя графы, дастатковая ўмова гамільтанавасці.}
{Обходы в графах: эйлеровы графы, критерий эйлеровости, гамильтоновы графы, достаточное условие гамильтоновости.}

\begin{problemList}

\problemItemSimple
{Пакажыце, што эйлеравы граф $G$ мае цотную колькасць кантаў тады і толькі тады,
калі $G$ змяшчае чотную колькасць вяршынь $v$, для якіх $\deg v \equiv 2\pmod{4}$.}
{Покажите, что эйлеров граф $G$ имеет чётное число рёбер тогда и только тогда,
когда $G$ содержит чётное число вершин $v$, для которых $\deg v \equiv 2\pmod{4}$.}

\bigskip

\problemItemSimple
{Няхай $G$ "--- дрэва. Пры якіх умовах граф $G^2$ з'яўляецца эйлеравым?}
{Пусть $G$ "--- дерево. При каких условиях граф $G^2$ является эйлеровым?}

\bigskip

\problemItemSimple
{Дакажыце, што эйлеравы граф змяшчае адзіны цыкл тады і толькі тады,
калі гэты граф з'яўляецца простым цыклам.}
{Докажите, что эйлеров граф содержит единственный цикл тогда и только тогда,
когда этот граф является простым циклом.}

\bigskip

\problemItemWithCommonPart
{Якія з графаў, паказаных на малюнку, з'яўляюцца гамільтанавымі?}
{Какие из графов, изображённых на рисунке, являются гамильтоновыми?}
{%
\begin{center}
    \includegraphics[scale=0.8]{figure_01}
\end{center}
}

\problemItemSimple
{Вызначыце колькасць адрозных гамільтанавых цыклаў у графах $K_n$ і $K_{n, n}$.}
{Определите число различных гамильтоновых циклов в графах $K_n$ и $K_{n, n}$.}

\bigskip

\problemItemSimple
{Няхай $G$ "--- звязны граф парадку $n\ge 4$, для любых трох адрозных вяршынь $u$, $v$ і $w$
якога прынамсі два з трох кантаў $uv$, $uw$ і $vw$ належаць $E(G)$.
Дакажыце, што $G$ "--- гамільтанавы граф.}
{Пусть $G$ "--- связный граф порядка $n\ge 4$, для любых трёх различных вершин $u$, $v$ и $w$
которого по крайней мере два из трёх рёбер $uv$, $uw$ и $vw$ принадлежат $E(G)$.
Докажите, что $G$ "--- гамильтонов граф.}

\bigskip

\problemItemSimple
{Граф называецца \textit{гусеніцай}, калі пасля выдалення ўсіх ягоных вісячых вяршынь атрымліваецца просты ланцуг.
Дакажыце, што калі $G$ "--- гусеніца, то $G^2$ гамільтанавы.}
{Граф называется \textit{гусеницей}, если после удаления всех его висячих вершин получается простая цепь.
Докажите, что если $G$ "--- гусеница, то $G^2$ гамильтонов.}

\bigskip

\problemItemSimple[*]
{Скажам, што набор неперасякальных па кантах ланцугоў (не абавязкова простых) пакрывае граф $G$,
калі кожнае кант графа $G$ уваходзіць у адзін з гэтых ланцугоў.
Дакажыце, што калі граф $G$ змяшчае роўна $2k$ вяршынь няцотнай ступені,
то найменшая колькасць неперасякальных па кантах ланцугоў, якія пакрываюць $G$, роўная $k$.}
{Скажем, что набор непересекающихся по рёбрам цепей (не обязательно простых) покрывает граф $G$,
если каждое ребро графа $G$ входит в одну из этих цепей.
Докажите, что если граф $G$ содержит ровно $2k$ вершин нечётной степени,
то наименьшее число покрывающих его непересекающихся по рёбрам цепей равно $k$.}

\bigskip

\problemItemSimple[*]
{Для графа $G$ вызначым ягоны кантавы граф $L(G)$ як граф,
вяршынямі якога з'яўляюцца канты графа $G$, і дзве вяршыні графа $L(G)$ сумежныя
тады і толькі тады, калі сумежныя адпаведныя ім канты графа $G$.
Дакажыце, што калі $G$ "--- эйлеравы граф, то граф $L(G)$ эйлеравы і гамільтанавы.}
{Для графа $G$ определим его рёберный граф $L(G)$ как граф,
вершинами которого являются рёбра графа $G$, и две вершины графа $L(G)$ смежны
тогда и только тогда, когда смежны соответствующие им рёбра графа $G$.
Докажите, что если $G$ "--- эйлеров граф, то граф $L(G)$ эйлеров и гамильтонов.}

\bigskip

\problemItemSimple[*]
{Пусть $G$ "--- $(n, m)$-граф парадку $n \ge 3$ і $m \ge \frac{n^2 - 3n + 6}{2}$.
Дакажыце, што $G$ "--- гамільтанавы.}
{Пусть $G$ "--- $(n, m)$-граф порядка $n \ge 3$ и $m \ge \frac{n^2 - 3n + 6}{2}$.
Докажите, что $G$ "--- гамильтонов.}

\bigskip

\problemItemSimple[*]
{Няхай звязны граф $G$ парадку $n \ge 3$ мае наступныя ўласцівасці:
(а) $G$ не змяшчае спароджанага  падграфа $K_{1, 3}$;
(б) акружэнне кожнай вяршыні $v \in V(G)$ спараджае ў графе $G$ звязны падграф.
Дакажыце, што $G$ з'яўляецца гамільтанавым.}
{Пусть связный граф $G$ порядка $n \ge 3$ удовлетворяет следующим свойствам:
(а) $G$ не содержит порождённого подграфа $K_{1, 3}$;
(б) окружение каждой вершины $v \in V(G)$ порождает в графе $G$ связный подграф.
Докажите, что $G$ является гамильтоновым.}

\bigskip

\problemItemSimple[*]
{Дакажыце, што кожнае кант 3-рэгулярнага графа змяшчаецца ў цотнай колькасці гамільтанавых цыклаў.
Выведзіце адсюль, што кожны гамільтанавы 3-рэгулярны граф мае прынамсі тры адрозныя гамільтанавыя цыклы.}
{Докажите, что каждое ребро 3-регулярного графа содержится в чётном числе гамильтоновых циклов.
Выведите отсюда, что каждый гамильтонов 3-регулярный граф имеет по меньшей мере три различных гамильтоновых цикла.}

\end{problemList}

\end{document}