\documentclass[12pt, a4paper]{article}

\usepackage{import}
\subimport{../../common/}{preamble}
\graphicspath{{figures/}}

\begin{document}

\biLangHeader
{10. Аўтаматныя граматыкі і рэгулярныя выразы. Лемы <<аб напампоўцы>>.}
{Автоматные грамматики и регулярные выражения. Леммы <<о накачке>>.}

\begin{center}
    \includegraphics[scale=0.7]{figure_01}
\end{center}    

\biLangHeader
{Аўтаматныя граматыкі і рэгулярныя выразы.}
{Автоматные грамматики и регулярные выражения.}

\begin{problemList}

\problemItemSimple
{%
Знайдзіце аўтаматную граматыку (і адпаведны рэгулярны выраз),
якая спараджае мову, што складаецца з:
\begin{belarusianEnumerate}
    \item дваічных уяўленняў лікаў, якія з'яўляюцца ступенямі 4;
    \item усіх радкоў з 0 і 1, якія змяшчаюць радок 001 у якасці падрадка;
    \item усіх радкоў з 0 і 1, якія не змяшчаюць радок 001 у якасці падрадка;
    \item усіх радкоў з 0 і 1, у якіх не больш за адну пару нулёў, якія ідуць запар;
    \item усіх радкоў з 0 і 1, у якіх у любым прэфіксе розніца паміж колькасцю 0 і колькасцю 1 не больш за адзін.
\end{belarusianEnumerate}
}
{%
Найдите автоматную грамматику (и соответствующее регулярное выражение),
которая порождает язык, состоящий из:
\begin{russianEnumerate}
    \item двоичных представлений чисел, являющихся степенями 4;
    \item всех строк из 0 и 1, содержащих строку 001 в качестве подстроки;
    \item всех строки из 0 и 1, не содержащих строку 001 в качестве подстроки;
    \item всех строки из 0 и 1, у которых не более одной пары подряд идущих 0;
    \item всех строки из 0 и 1, у которых в любом префиксе разница между количеством 0 и количеством 1 не больше одного.
\end{russianEnumerate}
}

\bigskip

\problemItemSimple[*]
{Пабудуйце аўтаматную граматыку, якая спараджае ўсе радкі з арабскіх лічбаў,
якія не змяшчаюць дзве аднолькавыя лічбы запар.}
{Постройте автоматную грамматику, которая порождает все строки из арабских цифр,
которые не содержат две одинаковые цифры подряд.}

\bigskip

\problemItemSimple[*]
{%
Пабудуйце аўтаматную граматыку над алфавітам
\[
\left\{
\begin{pmatrix} 0 \\ 0\\ 0 \end{pmatrix},
\begin{pmatrix} 1 \\ 0 \\ 0 \end{pmatrix},
\begin{pmatrix} 0 \\ 1 \\ 0 \end{pmatrix},
\begin{pmatrix} 0 \\ 0 \\ 1 \end{pmatrix},
\begin{pmatrix} 1 \\ 1 \\ 0 \end{pmatrix},
\begin{pmatrix} 1 \\ 0 \\ 1 \end{pmatrix},
\begin{pmatrix} 0 \\ 1 \\ 1 \end{pmatrix},
\begin{pmatrix} 1 \\ 1 \\ 1 \end{pmatrix}
\right\}
\]
Слова павінна належыць спараджальнай мове,
калі яно апісвае карэктнае прымяненне аперацыі складання над дваічнымі лікамі. Напрыклад, раз
\[
\begin{matrix}
& 0 & 1 & 1 & 0 \\
+ & 0& 1& 0& 1 \\
\hline
& 1& 0& 1& 1
\end{matrix}
\]
то радок 
\[
\begin{pmatrix} 0 \\ 0 \\ 1 \end{pmatrix}
\begin{pmatrix} 1 \\ 1 \\ 0 \end{pmatrix}
\begin{pmatrix} 1 \\ 0 \\ 1 \end{pmatrix}
\begin{pmatrix} 0 \\ 1 \\ 1 \end{pmatrix}
\]
належыць гэтай мове.
}
{%
Постройте автоматную грамматику над алфавитом
\[
\left\{
\begin{pmatrix} 0 \\ 0\\ 0 \end{pmatrix},
\begin{pmatrix} 1 \\ 0 \\ 0 \end{pmatrix},
\begin{pmatrix} 0 \\ 1 \\ 0 \end{pmatrix},
\begin{pmatrix} 0 \\ 0 \\ 1 \end{pmatrix},
\begin{pmatrix} 1 \\ 1 \\ 0 \end{pmatrix},
\begin{pmatrix} 1 \\ 0 \\ 1 \end{pmatrix},
\begin{pmatrix} 0 \\ 1 \\ 1 \end{pmatrix},
\begin{pmatrix} 1 \\ 1 \\ 1 \end{pmatrix}
\right\}
\]
Слово должно принадлежать порождаемому языку,
если оно описывает корректное применение операции сложения над двоичными числами. Например, т.к.
\[
\begin{matrix}
& 0 & 1 & 1 & 0 \\
+ & 0& 1& 0& 1 \\
\hline
& 1& 0& 1& 1
\end{matrix}
\]
то строка 
\[
\begin{pmatrix} 0 \\ 0 \\ 1 \end{pmatrix}
\begin{pmatrix} 1 \\ 1 \\ 0 \end{pmatrix}
\begin{pmatrix} 1 \\ 0 \\ 1 \end{pmatrix}
\begin{pmatrix} 0 \\ 1 \\ 1 \end{pmatrix}
\]
принадлежит этому языку.
}

\end{problemList}

\biLangHeader
{Лемы <<аб напампоўцы>>.}
{Леммы <<о накачке>>.}

\begin{problemList}

\problemItemWithCommonPart
{Дакажыце, што мова не з'яўляецца А-мовай:}
{Докажите, что язык не является А-языком:}
{%
\begin{belarusianEnumerateTwocol}
    \item $\{0^p\ |\ p - \text{просты лік}\}$
    \item $\{0^n1^n\ |\ n\ge 0\}$
    \item $\{\beta\beta^R\ |\ \beta\in\{0, 1\}^+\}$
    \item $\{\beta\beta^R\gamma\ |\ \beta\in\{0, 1\}^+, \gamma\in\{0, 1\}^*\}$
\end{belarusianEnumerateTwocol}
}

\problemItemWithCommonPart
{Дакажыце, што мова не з'яўляецца КС-мовай:}
{Докажите, что язык не является КС-языком:}
{%
\begin{belarusianEnumerateTwocol}
    \item $\{a^nb^nc^n\ |\ n\ge 0\}$
    \item $\{ww\ |\ w\in\{0, 1\}^*\}$
    \item $\{0^n1^m\ |\ m\le n^2\}$
    \item $\{a^ib^jc^k\ |\ k=\max\{i, j\}\}$
\end{belarusianEnumerateTwocol}
}

\problemItemSimple[*]
{Дакажыце, што мова, аналагічная мове з задачы 3* тэмы <<Аўтаматныя граматыкі і рэгулярныя выразы.>>, але вызначаная для аперацыі множання, не з'яўляецца A-мовай.}
{Докажите, что язык, аналогичный языку из задачи 3* темы <<Автоматные грамматики и регулярные выражения.>>, но определённый для операции умножения, не является А-языком.}

\end{problemList}

\end{document}