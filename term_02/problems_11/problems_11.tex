\documentclass[12pt, a4paper]{article}

\usepackage{import}
\subimport{../../common/}{preamble}
\graphicspath{{figures/}}

\begin{document}

\biLangHeader
{11. Дэтэрмінаваныя і недэтэрмінаваныя канечныя аўтаматы.}
{Детерминированные и недетерминированные конечные автоматы.}

\begin{problemList}

\problemItemWithCommonPartComplicated
{Няхай дадзены ДКА $M=(Q, \Sigma, \delta, q_0, F)$,
дзе $Q=\{q_0, q_1, q_2, q_3\}$, $\Sigma=\{0, 1\}$, $F=\{q_1, q_2\}$, а $\delta$ вызначаная наступнай табліцай:}
{Пусть дан ДКА $M=(Q, \Sigma, \delta, q_0, F)$,
где $Q=\{q_0, q_1, q_2, q_3\}$, $\Sigma=\{0, 1\}$, $F=\{q_1, q_2\}$, а $\delta$ определена следующей таблицей:}
{%
\[
\begin{matrix}
\delta & 0 & 1 \\
q_0 & q_1 & q_3 \\
q_1 & q_2 & q_3 \\
q_2 & q_2 & q_2 \\
q_3 & q_3 & q_3	
\end{matrix}
\]
}
{%
\begin{belarusianEnumerate}
    \item Пакажыце $M$ у выглядзе графа;
    \item Ці прымае $M$ радок 000? А радок 010?
    \item Якую мову $L(M)$ спараджае гэты ДКА? 
\end{belarusianEnumerate}\\[-15pt]
}
{%
\begin{russianEnumerate}
    \item Изобразите $M$ в виде графа;
    \item Принимает ли $M$ строку 000? А строку 010?
    \item Какой язык $L(M)$ порождает этот ДКА? 
\end{russianEnumerate}
}

\bigskip

\problemItemWithCommonPart
{Якую мову спараджае ДКА, паказаны на малюнку?}
{Какой язык порождает ДКА, изображённый на рисунке?}
{%
\begin{center}
    \includegraphics[scale=0.8]{figure_01}
\end{center}
}

\problemItemWithCommonPart
{Якую мову спараджае ДКА, паказаны на малюнку?}
{Какой язык порождает ДКА, изображённый на рисунке?}
{%
\begin{center}
    \includegraphics[scale=0.9]{figure_02}
\end{center}
}

\problemItemWithCommonPart
{Пабудуйце ДКА, які спараджае мову $L$:}
{Постройте ДКА, который порождает язык $L$:}
{%
\begin{belarusianEnumerate}
    \item $L_1=\{0, 1\}^*$;
    \item $L_2=\{01w|\ w\in\{0, 1\}^*\}$;
    \item $L_3=\{w|\ w\in\{0, 1\}^*, 00 \text{ з'яўляецца падрадком }w\}$;
    \item $L_4=\{w|w\in\{0, 1\}^*, w\ \text{"--- дваічнае ўяўленне ліку, які дзеліцца на 5}\}$;
    \item $L_5=L_2\cup L_3$;
    \item $L_6=L_2\cap L_3$.
\end{belarusianEnumerate}
}

\smallskip

\problemItemSimple
{%
Пабудуйце НДКА, які спараджае мову $L$:
\begin{belarusianEnumerate}
    \item $L$ складаецца з усіх бінарных радкоў, якія змяшчаюць $010$ у якасці падрадка;
    \item $L$ складаецца з усіх бінарных радкоў, якія пачынаюцца на $010$ ці сканчваюцца на $110$.
\end{belarusianEnumerate}
}
{%
Постройте НДКА, который порождает язык $L$:
\begin{russianEnumerate}
    \item $L$ состоит из всех бинарных строк, содержащих $010$ в качестве подстроки;
    \item $L$ состоит из всех бинарных строк, начинающихся на $010$ или оканчивающихся на $110$.
\end{russianEnumerate}
}

\bigskip

\problemItemWithCommonPartComplicated
{Няхай дадзены НДКА $M=(\{p, q, r\}, \{0, 1\}, \delta, p, \{q, r\})$, дзе}
{Пусть дан НДКА $M=(\{p, q, r\}, \{0, 1\}, \delta, p, \{q, r\})$, где}
{%
\[
\begin{matrix}
\delta & 0 & 1 \\
p & \{p, q\} & \{p\} \\
q & - & \{r\} \\
r & - & - \\
\end{matrix}
\]
}
{Пабудуйце эквівалентны яму ДКА.}
{Постройте эквивалентный ему ДКА.}

\bigskip

\problemItemWithCommonPartComplicated
{Няхай дадзены НДКА $M=(Q, \{0, 1\}, \delta, q_0, F)$, дзе $Q=\{q_0, q_1, q_2, q_3, q_4, q_5\}$, $F=\{q_3, q_4\}$ і}
{Пусть дан НДКА $M=(Q, \{0, 1\}, \delta, q_0, F)$, где $Q=\{q_0, q_1, q_2, q_3, q_4, q_5\}$, $F=\{q_3, q_4\}$ и}
{%
\[
\begin{matrix}
\delta & 0 & 1 & \epsilon \\
q_0 & \{q_0\} & \{q_0, q_2\} & \{q_1\} \\
q_1 & \{q_5\} & \{q_2\} & -\\
q_2 & \{q_3\} & - & - \\
q_3 & - & - & \{q_4\} \\
q_4 & \{q_3\} & - & - \\
q_5 & - & \{q_4\} & -
\end{matrix}
\]
}
{Пабудуйце эквівалентны яму ДКА.}
{Постройте эквивалентный ему ДКА.}

\bigskip

\problemItemWithCommonPart
{Пабудуйце ДКА, які спараджае тую ж мову, што і рэгулярны выраз:}
{Постройте ДКА, порождающий тот же язык, что и регулярное выражение:}
{%
\begin{belarusianEnumerate}
    \item $10+(0+11)0^*1$;
    \item $(0+10)^*(1+01)^*$.
\end{belarusianEnumerate}
}

\smallskip

\problemItemWithCommonPart
{Па НДКА, паказаным на малюнку, пабудуйце рэгулярны выраз, які прымае тую ж мову:}
{По НДКА, изображённому на рисунке, постройте регулярное выражение, которое принимает тот же язык:}
{%
\begin{belarusianEnumerateTwocol}
    \item \includegraphics[scale=0.92]{figure_03}
    \item \includegraphics[scale=1.05]{figure_04}
\end{belarusianEnumerateTwocol}
}

\smallskip

\problemItemWithCommonPartComplicated[*]
{Вызначым наступную аперацыю множання на мностве $\{a, b, c\}$:}
{Определим следующую операцию умножения на множестве $\{a, b, c\}$:}
{%
\[
\begin{matrix}
\times & a & b & c \\
a & c & a & b \\
b & b & c & a \\
c & c & b & c
\end{matrix}
\]
}
{Для кожнага радка з из $\{a, b, c\}^+$ вызначым аперацыю $value(w)$ як значэнне, якое атрымліваецца пры прымяненні паслядоўна аперацыі множання на сімвалах радка $w$ злева направа. Напрыклад, $value(abcb) = ((a\times b)\times c)\times b=c$. Пабудуйце НДКА, які спараджае мову $L$, які складаецца з усіх радкоў $w$ над алфавітам $\{a, b, c\}$ такіх, што $value(w)=value(w^R)$.}
{Для каждой строки из $\{a, b, c\}^+$ определим операцию $value(w)$ как значение, которое получается при применении последовательно операции умножения на символах строки $w$ слева направо. Например, $value(abcb) = ((a\times b)\times c)\times b=c$. Постройте НДКА, который порождает язык $L$, состоящий из всех строк $w$ над алфавитом $\{a, b, c\}$ таких, что $value(w)=value(w^R)$.}

\end{problemList}

\end{document}