\documentclass[12pt, a4paper]{article}

\usepackage{import}
\subimport{../../common/}{preamble}

\begin{document}

\biLangHeader
{12. Машыны Цьюрынга.}
{Машины Тьюринга.}

\begin{problemList}

\problemItemWithCommonPartComplicated
{Няхай $M_1$ "--- машына Цьюрынга, з $\Sigma=\{a\}$, $Q=\{q_0, q_1, q_2\}$ і $\delta$, якія апісваюцца наступнай табліцай:}
{Пусть $M_1$ "--- машина Тьюринга, с $\Sigma=\{a\}$, $Q=\{q_0, q_1, q_2\}$ и $\delta$, описывающееся следующей таблицей:}
{%    
\begin{center}
    \begin{tabular}{|c|c|c|}
        \hline
        & $q_1$ & $q_2$ \\ \hline
        $a$ & $\lambda q_2 +1$ & $aq_2-1$ \\ \hline
        $\lambda$ & $aq_00$ & $\lambda q_1 +1$ \\ \hline
    \end{tabular}
\end{center}
}
{Які будзе працэс вылічэння на словах $aaaa$ і $a\lambda\lambda a$?}
{Каким будет процесс вычисления на словах $aaaa$ и $a\lambda\lambda a$?}

\bigskip

\problemItemWithCommonPartComplicated
{Няхай $M_1$ "--- машына Цьюрынга, з $\Sigma=\{a\}$, $Q=\{q_0, q_1, q_2, q_3\}$ и $\delta$, якія апісваюцца наступнай табліцай:}
{Пусть $M_1$ "--- машина Тьюринга, с $\Sigma=\{a\}$, $Q=\{q_0, q_1, q_2, q_3\}$ и $\delta$, описывающееся следующей таблицей:}
{%
\begin{center}
    \begin{tabular}{|c|c|c|c|}
        \hline
        & $q_1$ & $q_2$ & $q_3$ \\ \hline
        $a$ & $aq_2+1$ & $aq_20$ & $aq_1+1$ \\ \hline
        $\lambda$ & $aq_2+1$ & $aq_3+1$ & $aq_00$\\ \hline
    \end{tabular}
\end{center}
}
{Які будзе працэс вылічэння на словах $aaa$, $a\lambda\lambda a$ і $aa\lambda\lambda\lambda a$?}
{Каким будет процесс вычисления на словах $aaa$, $a\lambda\lambda a$ и $aa\lambda\lambda\lambda a$?}

\bigskip

\problemItemSimple
{%
Пабудуйце машыну Цьюрынга, якая распазнае мову $L$:
\begin{belarusianEnumerate}
    \item $L$ складаецца з усіх цотных лікаў ва ўнарным запісе;
    \item $L=\{ww^R|w\in\{0, 1\}^*\}$;
    \item $L=\{a^mb^n|m\le n\}$;
    \item $L=\{a^mb^nc^{n+m}|n, m\ge 0\}$.
\end{belarusianEnumerate}
}
{%
Постройте машину Тьюринга, распознающую язык $L$:
\begin{russianEnumerate}
    \item $L$ состоит из всех чётных чисел в унарной записи;
    \item $L=\{ww^R|w\in\{0, 1\}^*\}$;
    \item $L=\{a^mb^n|m\le n\}$;
    \item $L=\{a^mb^nc^{n+m}|n, m\ge 0\}$.
\end{russianEnumerate}
}

\bigskip

\problemItemSimple
{Пабудуйце машыну Цьюрынга, якая пераводзіць лік з унарнай сістэмы злічэння ў траічную.}
{Постройте машину Тьюринга, переводящую число из унарной системы счисления в троичную.}

\bigskip

\problemItemSimple
{Пабудуйце машыну Цьюрынга $R_m$, якая вылічвае астачу ад дзялення ліку,
запісанага ва ўнарнай сістэме злічэння, на $m$.}
{Постройте машину Тьюринга $R_m$, вычисляющую остаток от деления числа,
записанного в унарной системе счисления, на $m$.}

\bigskip
 
\problemItemWithCommonPart
{Пабудуйце машыну Цьюрынга, якая вылічвае функцыю:}
{Постройте машину Тьюринга, вычисляющую функцию:}
{%
\begin{belarusianEnumerate}
    \item $f(n)=n+1$;
    \item $f(m, n)=max\{m, n\}$;
    \item $\pi_2^2(n_1, n_2)=n_2$;
    \item $\forall 1\ge i \ge k\ \pi_{i}^k(x_1, \ldots, x_k)=x_i$;
    \item $sub(n, m)=
    \begin{cases}
        n-m,\ n\ge m \ge 0\\
        0,\ m > n \ge 0
    \end{cases}$
\end{belarusianEnumerate}
}

\smallskip

\problemItemSimple[*]
{Пабудуйце машыну Цьюрынга, якая распазнае мову $L=\{ww|w\in\{0, 1\}^*\}$.}
{Постройте машину Тьюринга, распознающую язык $L=\{ww|w\in\{0, 1\}^*\}$.}

\bigskip

\problemItemSimple[*]
{Пабудуйце машыну Цьюрынга, якая пераводзіць лік з унарнай сістэмы злічэння ў дзесяцічную.}
{Постройте машину Тьюринга, переводящую число из унарной системы счисления в десятичную.}

\bigskip

\problemItemWithCommonPart[*]
{Пабудуйце машыну Цьюрынга, якая вылічвае функцыю:}
{Постройте машину Тьюринга, вычисляющую функцию:}
{%
\begin{belarusianEnumerate}
    \item $mult(n, m)=nm$;
    \item $insert_i^k(x_1, \ldots, x_k, y)=(x_1, \ldots, x_{i-1}, y, x_i, \ldots, x_k)$.
\end{belarusianEnumerate}
}

\end{problemList}

\end{document}